\chapter{Introduction}

\section{Motivation}

Coastal areas are attractive for human settlement, and up to 39\% of human population lives within 100km of the coast \parencite{Magdalena2021}, including many of the world's largest cities. 10\% of the world's population live in the an area that is less than 10m above sea level \parencite{Neumann2015,Lichter2011}, an area known as the Low Elevation Coastal Zone. All coastal populations face increasing vulnerability to coastal flooding hazards, due to increases in the sea level and changes in extreme weather patterns. To protect populations from these increasing risks, further investment in coastal protection is required; without more investment in coastal defense infrastructure, the population exposed to flooding risk is projected to increase by 50\% \Parencite{Kirezci2020}. One approach to mitigating these hazards is to employ the techniques of Building with Nature, using natural processes and ecosystem functions to create resilient coastal defenses. \todo{clean up transition} Mangroves provide coastal protection by reducing wave energy, while also adapting naturally to sea level rise, and providing other ecosystem services like increasing water quality and increasing fish spawning areas. Mangroves already provide flood protection to over 15 million people \parencite{Menendez2020}, but in the last 50 years between 30-50 percent of the world's mangrove have been lost \parencite{add}.

One current weakness of mangroves as a flood defense strategy is that the research from 

%% bullet points, delete below
\todo{Make motivation section into a paragraph instead of bullets}
\begin{itemize}
      \item 39\% of the world's population lives within 100km of the coast.\parencite{Magdalena2021}
      \item If further investment in coastal defence is not undertaken, the amount of people exposed to flooding will increase 52\%, and the value of assets at risk will increase by 46\% \parencite{Kirezci2020}.
      \item Currently 10\% of the world's population lives in a Low Elevation Coastal Zone (LECZ), defined as land areas less than 10m above sea level. These areas are especially vulnerable and population in these zones is only projected to increase, especially in African cities \parencite{Neumann2015,Lichter2011}
      \item mangroves currently provide flood protection to over 15 million people \parencite{Menendez2020}
      \item 30-50\% of the world's mangroves have been lost in the last 50 years \parencite{}. 
\end{itemize}
\subsection{Impact of nearshore data}

Nearshore bathymetry is crucial data for many aspects of coastal management and research. Bathymetric data is essential for many aspects of the blue economy, including aquaculture, marine energy, submarine cables, dredging operations, design of sea defenses, navigation, scientific research, and ecosystem preservation \parencite{Cesbron2021,Ashphaq2021}. However, the nearshore zone is notoriously difficult to survey. There is currently a global lack of data in the 5-10m zone \parencite{Albright2021} and up to 50\% of the world's shallow coastal zones remain unsurveyed \parencite{IHO/OHI2022}. Where surveys do exist, they can be decades out of date, especially in the 40\% of the world's coasts that are sandy and highly dynamic \parencite{Almar2021e}.  

There have been attempts at global bathymetric datasets, most notably GEBCO, which is an annually updated global bathymetric and topograhpic dataset. The bathymetry in GEBCO grids is derived by assimilation of acoustic soundings provided by ships and gravimetric bathymetry measurements \parencite{Cesbron2021}. While it provides very useful information about deep oceans, the resolution is not sufficient for detailed modeling, and because sonar data is limited in many shallow nearshore zones, GEBCO accuracy in the nearshore zone is limited \parencite{add}.

In the context of researching the impact of mangroves on flood risk, one of the most important aspects of a wave model is the nearshore bathymetry, since the exact shape and depth of the profile has a large impact on the hydrodynamic response of mangrove ecosystems \parencite{Horstman2014,Maza2019}. The current state of wave modeling of mangrove ecosystems either relies on in-situ survey data, or on a assumed profile shape. The further development of mangrove wave modeling will benefit from better quantification of bathymetry, topography, and mangrove structure \parencite{Menendez2020}. This detailed data is especially helpful given the heterogeneity between different mangrove forests \parencite{Mazda2013}.

\subsection{Limitations of conventional nearshore survey}

The nearshore zone is a very difficult environment to perform bathymetric surveying. \parencite{Parrish2019}. All current bathymetric survey techniques face limitations that restrict their use in the shallowest 5-10m of the nearshore zone. Most deep water survey is done using acoustic sounding devices attached to ships. This limits their operation in water shallower than 4-5 meters or in areas with naviagational hazards \parencite{Cesbron2021,add}. The current standard for high resolution survey in the nearshore zone is airborne lidar surveying. However, these surveys are extremely expensive to perform and require extensive post-processing effort to create a usable surface model.

\section{Identified Knowledge gaps}
A major restriction in modeling how mangroves and waves interact is the lack of global nearshore bathymetry data.  
\section{Research Question}
The primary question that this project intends to answer is: \\

\vspace{2mm}
\emph{How can spaceborne remote sensing data be used to improve estimates of nearshore bathymetry along mangrove-lined coasts?}

To answer this question, the following subquestions will be investigated:

\begin{itemize}
      \item Which mangrove sites in the world are spaceborne lidar and optical SDB applicable?
      \item How can ICESat-2 transects that contain bathymetry be identified algorithmically?
      \item Which of the established optical SDB methods are most reliable in mangrove ecosystems?
      \item Once transects with bathymetry are found, how can the seafloor elevation data be extracted?
      \item How can lidar photon return locations reflecting the seafloor be separated from background noise?
      \item How can gaps be filled in areas with missing lidar photons?
\end{itemize}

\chapter{Literature Review}


\section{Bathymetric Survey Methods}

\subsection[]{Airborne Lidar}
\subsection{Spaceborne Lidar}

The earliest attempts to use Light Detecting and Ranging (lidar) to survey the coastal zone date back to the late 1960s. \parencite{Bailly2016}.

\begin{itemize}
      \item 1064 and 1550nm lasers are preferred for topographic lidar
      \item 532nm laser better for bathymetric lidar
      \item photon-counting lidar allow low power consumption, so they are more practical from a satellite
      \item in waveform-resolving lidar, each point return is a pulse containing thousands of photons
\end{itemize}

\section{Mangroves as Coastal Defense}

Sea level rise poses a threat to all coastal communities, including many of the world's largest population centers. The risk is even more existential for low-lying tropical communities, who have done little to contribute to the current climate crisis, but are disproportionately affected by the sea level rise. Low-lying nations often do not have any available high ground to move cities to, and sometimes do not have access to the resources to fully mitigate these risks through hard infrastructure. For these reasons, mangroves forests are an important resource for these communities, providing coastal protection that can adapt to sea level rise by trapping sediment and expanding outward if conditions permit.

One way to adapt to the changing sea levels, and possible increases in extreme weather events is to promote the restoration and expansion of mangrove forests. It is well-established that mangrove forests can reduce wave energy \parencite{Maza2019,Menendez2020}\todo{add a few more citations here}, and to a certain extent, storm surges \parencite{Montgomery2019a,Chen2021,Mcivor2012}. Mangroves also offer many other ecosystem benefits including enhancing fish stocks, improving water quality, storing significant amounts of carbon, and attracting tourism \parencite{Atkinson2016b}

\begin{figure}[htbp]
      \centering
      \includegraphics[width = 0.7\textwidth]{figures/mangroves_reefs_Losada2018.png}
      \caption{The protective effects of Mangroves (showing reefs as well). From \parencite{Losada2018}}
      \label{mangrove-protection-diagram}
\end{figure}
\todo{fix this caption}

Shallow water bathymetry and mangrove height are difficult to survey manually, due to the difficulty of field work in dense mangrove environments \parencite{Gijsman2021}. Current study of the effects of mangroves on wave attenuation are limited by this lack of hydrodynamic data about these ecosystems \parencite{Horstman2014}.  Therefore, finding an automated way of extracting this anywhere in the world from and publicly available data is a massive improvement when modeling sites that do not have existing survey data, and can be used to as the basis for wave modeling, and process-based models. Improved models can allow better characterization of the morphologic response to mechanical factors like sea level rise and sediment supply that need further study to understand the system response to climate change.

One weakness of mangroves as a flood protection mechanism is that their response to long term changes in the biogeomorphic changes is not well known \parencite{Gijsman2021}.

\section{ICESat-2}

The ICESat-2 mission is intended to gather high resolution topographic data on a global scale. The satellite carries the Advanced Topographic Laser Altimeter System (ATLAS). ATLAS is a highly sensitive photon-counting, green-light lidar. The satellite instrument points at reference ground tracks (RGT) along the earth's surface with a repeat time of 91 days. Along the reference track, there are 3 beams, one pointing directly at the reference track, and two that are offset by approximately 3km on either side.

\begin{figure}
      \centering
      \includegraphics[width=0.5\textwidth]{./figures/ATLAS_beam_layout_from_user_guide.png}
      \caption{Layout of the ICESat-2 beams}
\end{figure}

Each of the 3 beams emits both a strong and weak beam, with the strong beam being approximately 4x more powerful \parencite{Neumann2019d}. Of the approximately \(10^{14}\) of photons emitted per pulse, up to  10 make it back to the sensor and are detected.\parencite{Neumann2019d}. The exact number of emitted photons that are subsequently detected at the sensor depends on the local atmospheric conditions and the reflectivity of the surface \parencite{Neumann2019e}.
\begin{figure}[htbp]
      \centering
      \includegraphics{./figures/3d_beam_view_from_atl03ATBD.png}
      \caption{The layout of the ICESat-2 beams in 3D space. from \cite{Neumann2019d}}
      \label{3d-beams}
\end{figure}


The main mission of the satellite is to gather data about mass and elevation changes in ice sheets and glaciers, and to study global global canopy height \parencite{Markus2017}. To increase the spatial coverage of the vegetation height, the observatory is pointed away from the reference ground track when flying over land to increase the spatial density of the observations of vegetation height \parencite{Markus2017}. The offpointing begins before the satellite begins to record data over land, so the nearshore coastal area is also included, and therefore bathymetric and hydrological applications also benefit from increased spatial coverage \parencite{Magruder2021}.

To locate the position of each photon in 3D space, the time of flight of the photon is calculated with a precision of 800 ps\parencite{Neumann2019d}. The location of the center of mass of the instrument is found using Global Positioning System (GPS) systems onboard the satellite. By combining the measured time of flight and satellite position, the geolocation of the photon can be estimated with high accuracy \parencite{Neumann2019d}.

The photon geolocation data is distributed by NASA as the ATL03 data product.

\subsection{Weak Vs. Strong Beams}

The beams are divided into weak and strong signals to enhance the radiometric dynamic range. The strong beams are expected to provide better signal-noise ratios over low-reflectively surfaces.\parencite{Neumann2019d} Therefore, these beams are expected to provide the best data for lidar bathymetry measurements.

\subsection{Refraction correction}


\subsection{Vertical Height Reference}
The ATL03 data product reports the photon heights relative to the WGS84 reference ellipsoid. These ellipsoidal heights already include corrections for:

\begin{itemize}
      \item The solid earth tides 
      \item Ocean loading 
      \item Ocean Poletide 
      \item Wet and dry atmospheric delays
\end{itemize}
 
The height provided in ATL03 is calculated by the following equation:

\[H_{GC} =  H_{P} - H_{OPT} - H_{OL} - H_{SEPT} - H_{SET} - H_{TCA}\]

Where:

\begin{itemize}

      \item \(H_{GC}\) is the geophysically corrected photon height above the WGS84 ellipsoid
      \item \(H_{P}\) is the raw photon height above the WGS84 ellipsoid
      \item \(H_{OPT}\) is the height of the Ocean Pole tide
      \item \(H_{OL}\) is the height of the ocean load tide
      \item \(H_{SEPT}\) is the height of the solid earth pole \textbf{tide}
      \item \(H_{SET}\) is the solid earth tide
      \item \(H_{TCA}\) is the height of the total column atmospheric delay
\end{itemize}

Included in the data are the height of the tide-free geoid, the height difference between the tide-free and mean-tide geoid, and the height of the tide relative to the mean tide geoid as calculated by the GOT4.8 model. This GOT4.8 model tidal height is a based on a low resolution model, and therefore is less accurate in nearshore coastal areas and within embayments \parencite{Neumann2019e}.

\subsection{Signal Photon Identification}

The ATL03 data product includes a calculated confidence that a given photon return is signal or noise. The probability is assigned for each of the 5 surface types. The default classification provided with the data are not useful for determining if a individual photon is from the seafloor, but is reliable for detecting ocean surface photons. Therefore, to calculate the bathymetry, all photon returns, including those classified as noise by the ATL03 signal/noise algorithm are filtered to remove points that are too high or low to be considered bathymetric returns. Then, a separate algorithm specifically calibrated to distinguish bathymetric signal from noise photons is applied to this data.

\section{Optical Satellite Derived bathymetry}

Cloud platforms like Google Earth Engine (GEE) \parencite{Gorelick2017a} allow significant speedups in SDB methods \parencite{Pike2019}. 


\subsection{Multispectral imagery}

There are many techniques to approximate depth by the spectral signatures of optical satellite data. The two classical ones are the band-ratio model and the linear band model. They work well in optically shallow waters \parencite{Salameh2019}.

\subsection{Machine learning methods for Optical Imagery}


\section{Lidar bathymetry}

Green light lasers can penetrate the water up to a certain depth based on the local water clarity. In very clear water depth detection of up to 40m has been achieved from spaceborne lasers \parencite{Parrish2019}. The laser path has to be corrected for refraction induced by the water before the bathymetry can be reliably estimated.

\section{Active-passive Sensor fusion: The best of both worlds}

The potential for bathymetric mapping using spaceborne laser observations has been known since before the advent of the ICESat-2 mission. The predecessor mission carried a lidar instrument called the Geoscience Laser Altimeter System (GLAS). While GLAS was a green-light laser, it was intended for measuring atmospheric aerosols \parencite{Abshire2005}. However, because of the laser architecture, GLAS was not able to penetrate the water column \parencite{Forfinski-Sarkozi2016}. However, a prototype of ATLAS, called the Multiple Altimeter Beam Experimental lidar (MABEL) instrument was tested with high-altitude aircraft missions, allowing a simulation of the data that would be transmitted by ATLAS \parencite{Mcgill2013}. Early experiments with MABEL showed good agreement between bathymetry measured with MABEL and high-quality airborne reference data \parencite{Jasinski2016,Forfinski-Sarkozi2016}. Since the launch of ICESat-2 in 2018, there have been several studies that evaluate the data for bathymetric mapping purposes, and experiment with different techniques to distinguish signal from noise. One of the first and most cited is Parrish et al.~(2019) \parencite{Parrish2019}. The paper addresses refraction correction by manually segmenting bathymetric signal points, estimating the water surface based on photons returns classified as water surface, using the provided angle of incidence of the satellite. The Parrish paper includes correction for both horizontal and vertical effects of refraction.

When the instrument is pointed directly at the RGT the laser beams are nearly pointing at the nadir. When directly on-nadir, the additional horizontal error induced by refraction is approximately 9cm \parencite{Parrish2019}, which for bathymetric purposes is negligible. However, by design ATLAS can point up to $5 \degree$ off-nadir (Equal to 43km away from the RGT) \parencite{Magruder2021}. The off-nadir pointing mode is used over land to increase the density of tracks in the mid-latitudes. This allows better spatial coverage of the vegetation canopy height and coasts at the mid-latitudes. When pointing off-nadir, the horizontal error is much more significant and must be corrected for accurate measurements. In all of these calculations, the water surface is assumed to be flat.

Parrish et al.~test their technique in 4 different regions: St.~Thomas, Turks and Caicos, North West Australia, and the Great Bahama Bank. In the Parrish paper, they suggest that the best use of lidar data for bathymetry is to combine it with optical/multispectral techniques. Because the lidar-derived data provides highly accurate point estimates along a certain track, and the multispectral approach allows the estimation for a 2D area, combining the two techniques provides a synergistic fusion of the strengths of both. The lidar-derived depths are used as training data for the multispectral models allowing an accurate 2d picture of the bathymetry.

Further studies have implemented the combination of multispectral SDB that is calibrated using satellite data. This technique has shown promising results.

\subsection{Summary of prior research in fusion of optical and lidar SDB}

The studies that have used some version of the proposed technique are summarized in table \ref{research-summary}.


\begin{landscape}
      \todo{Sort table by year}
      \todo{figure out page margins}
      \begin{table}
            \label{research-summary}
            \caption{Summary of SDB research that combines spaceborne lidar and optical data}
            % \centering
            \raggedright
            \begin{tabular}{lllp{3cm}p{3cm}ll}
                  \midrule
                  Author                                               & Year & Dataset & Refraction Correction Method  & S/N Classification method      & Tide Correction & Notes                        \\
                  \hline
                  \citeauthor{Forfinski-Sarkozi2016} & 2016 & MABEL   & First-order depth correction  & Manual                         & N/A             & non-tidal                    \\
                  \citeauthor{Parrish2019}                    & 2019 & ATLAS   & Parrish method                & Manual                         & N/A             & used  ellipsoidal height     \\
                  \citeauthor{Liu2021}                             & 2021 & ATL03   & Liu method                    & DBSCAN after Ma et al.         & TMD tidal model & -                            \\
                  \citeauthor{Ma2020}                             & 2020 & ATL03   & Parrish + sloping sea surface & Adaptive DBSCAN                & OTPS2           & -                            \\
                  \citeauthor{Xie2021}                            & 2021 & ATL03   & Parrish method                & Adaptive DBSCAN                &                 & DBSCAN  is used twice        \\
                  \citeauthor{Thomas2021d}                     & 2020 & ATL03   & Parrish method                & Manual                         & Not Specified   & -                            \\
                  \citeauthor{Albright2021}             & 2021 & ATL03   & First-order                   & Manual                         & N/A             & Converted to NAD83           \\
                  \citeauthor{Cao2021}                            & 2021 & ATL03   & First-order depth correction  & A-DRAGANN                      & OTPS2           &                              \\
                  \citeauthor{Lee2021}                           & 2021 & ATL03   & Not specified                 & not specified                  & T\_TIDE         & -                            \\
                  \citeauthor{LeQuilleuc2022b}            & 2022 & ATL03   & Parrish                       & DBSCAN with manual corrrection & N/A             & Compared ellipsoidal heights \\
                  \citeauthor{Xu2022a}                            & 2021 & ATL03   &                               &                                &                 & looked above water           \\
                  \bottomrule
            \end{tabular}
      \end{table}
      \todo{maybe remove last study}
\end{landscape}
% \end{sidewaystable}

\chapter{Proposed Methodology}
The following method is proposed to evaluate the effectiveness of hybrid SDB along mangrove coasts.

\section{Find areas for global analysis}
There are several datasets of the global distribution of mangrove forests. \citeauthor{Giri2011b} derived the first globally consistent remote-sensing-based global extent analysis \parencite{Worthington2020}. Another paper \parencite{Worthington2020a} created a globally consistent database of mangrove types. Each mangrove area is classified as either carbonate or terrigenous, and the coast type (Delta, Estuary, Lagoon, Open Coast). This data can provide valuable information on which types of mangrove environments might be favorable to automated bathymetric extraction. Certain mangrove environments might have higher average turbidity and therefore might be more difficult to extract spaceborne lidar from. The local turbidity at the time of the sattelite pass can be found using the GlobColour global database of ocean color \parencite{Garnesson2019} \todo{find academic citation for globcolour dataset}. 

To create areas of analysis for this project, first the \citeauthor{Giri2011b} dataset will be buffered by a distance such that it includes the offshore area, and these polygons will be generalized to snap to a rectangular grid. Polygons with an area that is too small can be dropped from the dataset. The remaining polygons will be used as locations for the global analysis.

\section{Download ICESat-2 ATL03 tracks within polygons of interest}
Using a python script that interacts with the NASA National Snow and Ice Data Center (NSIDC) data download API, the ATL03 variables of interest can be downloaded for each polygon. 

\section{Derive Signal/noise classification algorithm}
Using the DBSCAN algorithm with adaptive parameters, find bathymetric points (if there are any). Different methods of adaptive parameter choice can be tested at test sites to find a way to reliably distinguish transects with no useful signal from those with bathymetric signal, that are applicable anywhere in the world.

\section{Correction for refraction and tides}
\todo{create separate sections for refraction and tides}
Find the tidal correction from a tidal model, and then use this to transform the ellipsoidal height provided by the ICESat-2 data into a depth relative to local MSL.

The photon location data will be corrected for refraction by the Parrish method (possibly using one of the adaptions for wave height if needed)

\section{Perform Sentinel-2 SDB classification on sites with ICESat-2 Bathymetric data}
Some sites, either due to turbidity, depth, or instrument data quality issues, will not provide reliable bathymetry photon points. Any sites without lidar reference data will be removed from the dataset. 

Remaining polygons can then be analyzed in GEE to find the sentinel-2 images that are as temporally close to the date of the ICESat-2 track. An empirical optical SDB method will be used on these sites. 

\section{Summarize and analyze results}


\section{Proposed method flowchart}
\todo{Update figure based on rewritten methodology}
\begin{figure}
      \centering
      \includegraphics[width=\textwidth]{./figures/approach.png}
      \caption{Proposed methodology}
      \label{method-flowchart}
\end{figure}

\chapter{Practicalities}

\section{Planned Schedule}

Figure \ref{project-schedule} shows the tenative schedule for the project. 
\todo{update schedule figure to reflect the methodology section}
\begin{landscape}

      \begin{figure}
            \centering
            \includegraphics{./figures/thesis-schedule.png}
            \caption{Tenative Schedule}
            \label{project-schedule}
      \end{figure}

\end{landscape}
\section{Planned final document outline}
