Nearshore bathymetry needs 

Accurate nearshore bathymetry data is essential for navigational safety, shore protection, and coastal engineering research. Inaccurate bathymetry is one of the most severe limitations on the accuracy of numerical wave modeling. Because high resolution data is not available in most of the world, nearshore wave modeling studies often rely on global bathymetry datasets like GEBCO. The ~500m horizontal resolution is not sufficient for accurate wave modeling. One potential source of higher resolution nearshore data is other spaceborne remote sensing, such as lidar satellites. This study investigates how additional data sources can be combined with GEBCO to create a higher-resolution, higher accuracy product.

Satellite-Derived bathymetry

Lidar satellites, like ICESat-2, have been shown to be able to penetrate the first few meters of the nearshore zone. This can provide bathymetric data coverage along limited areas after some processing to isolate bathymetric signal. Bathymetry can also be estimated using the spectral response from optical remote sensing satellites. The topic under investigation is if these remote sensing sources, and any additional public data sources, can be combined to produce a statistically upscaled version of GEBCO that would provide higher resolution coverage in the nearshore zone. To combine them, a Kalman updating technique is proposed. The Kalman update technique allows a combination of the measurements at each discrete point, which is weighted based on the margin of error of each measurement.

Method of data assimilation 

To implement the Kalman upscaling, first global data from GEBCO is clipped to an area of interest, and then interpolated bilinearly to 50m resolution. Then, the ICESat-2 photon data for the area is processed to find a series of point measurements containing bathymetric data. To fill in the gaps between these point measurements, the bathymetric points are subsampled and interpolated to the same resolution as the GEBCO data using ordinary kriging, which results in a raster of uncertainty, and a raster of the interpolated depth value. To update the interpolated GEBCO raster, the Kalman gain is calculated for each raster cell in an elementwise manner, and using the Kalman state equation a new bathymetry grid is produced. If other data is available, the process can be applied recursively with other depth and uncertainty grids, with the output depth and uncertainty of the first iteration used the input to the second iteration, and so on. The process can allow the accuracy and resolution of the global dataset to be increased, without needing any additional a priori data.

Validation

To validate the method, the RMSE error is calculated between the resulting bathymetry grid and previously validated, high accuracy survey data. The validation will be applied at several global test sites to verify that the method generalizable to other regions.

Outlook

The results of this research could allow easier methods of characterizing nearshore bathymetry in remote areas, and could also be extended to include temporal variation – as more bathymetric data becomes available, it could be used measure the dynamic changes coastal systems. Data with this temporal dimension would provide valuable validation data for coastal dynamics models.
