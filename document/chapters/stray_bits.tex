\chapter*{stray bits}

\section*{error discussion}
\subsection*{Misclassified photons in ATL03 data}

Correcting filtering of sea surface returns is very important to the accuracy of the bathymetry signal finding, because the sea surface signal is often several orders of magnitude more dense than the bathymetric signal, and the location of the sea surface is also used to calculate the depth of a photon for refraction correction. The default photon classification provided in the ATL03 data is used to identify the sea surface within the filtering algorithm. Errors in the default ocean signal classification can result in sea surface signal being inadvertently included in the subsurface data and biasing the KDE results.

The default classification is often reliable, but when there is a large area where the bathymetric surface is shallow and nearly parallel to the sea surface, there can be misclassifications. An example of this is shown in figure \ref{fig:ageeba_bad_classes}. In this example, actual sea surface is not classified as a high confidence ocean surface return, and some areas that appear to be bathymetric signal are classified as ocean surface. This causes the bathymetric signal to be thrown out because it is incorrectly considered to be the sea surface.

This could potentially be mitigated by a different filtering strategy calculates the local sea surface based on the local geoid  This could be very feasible in microtidal areas where the tidal signal has a smaller impacts.

\pdfcomment{temp figure, replace with matplotlib figure with axis labels. Maybe also include a plan view map with scale bar to understand the geographic context}
\begin{figure}[htbp]
    \centering
    \includegraphics[width=\textwidth]{figures/ageeba_beach_example.png}
    \caption{Classification of photons from 2021-07-19, Beam gt3r, reference ground track 396. The two parallel straight lines from 200 to 800 are the sea surface and the bathymetric signal. The NASA photon classification algorithm misclassifies the bathymetric points as ocean surface returns}
    \label{fig:ageeba_bad_classes}
\end{figure}


\subsection{Limited spatial coverage in some islands}

In support of the vegetation mission of ICESat-2, the instrument is sometimes pointed up to several degrees to the side of the reference ground tracks when the satellite passes over land. This increases the spatial density of points at the expense of the temporal resolution. For bathymetric purposes the increased spatial resolution gives a more even  coverage of nearshore zone bathymetry. 

However, the land mask that is used to determine the off-pointing strategy has a limited resolution, and therefore some island nations do not benefit from the increased spatial density. This was noted when trying to collect data from Fiji and the Maldives. Due to apparently being located within the off-pointing zones, both of the aforementioned islands only have tracks which are 3km apart. They can still potentially collect bathymetry data if conditions are otherwise good, but the further reduction in spatial coverage limits the accuracy of the kriging method. This is unfortunate because many of the states that are at the highest need of detailed bathymetry for numerical studies are big ocean island nations. The tradeoff for this scenario is that the temporal resolution is significantly better, so the spaceborne lidar could be useful for studying the changes over time. 

\subsection{Inherent uncertainty of KDE Method}
There are a number of input parameters to the filtering and the density-based bathymetry finding methods. These parameters can be optimized for each site to reduce the RMSE error as much as possible if there is some validation data available. However since the end goal of the project is to be able to improve estimates without using any in situ data, ideally there would be no need for optimization based on the site.

Currently the globally-set parameters are sufficient to extract bathymetry without any tuning for all of the case studies that are investigated. However, the inability to tune in advance is a limitation. 

One possible future step would be to gather even more validation sites, and explore which other variables might influence the best parameter setting. It is possible that there are certain site variables which predict the optimal parameter options. Even so, upscaling of validation sites would allow better insight into which variables predict the presence of valid data.  

\subsection{ATL03 data quality issues}\label{sec:discussion-photon-issues}

There are a number of known issues with the ICESat-2 data. They are either due to atmospheric and environmental conditions, or due to limitations of the instrument. Many of them can be detected in advance, and then the effected granule data can be thrown out or the issue otherwise corrected for. However, there might be some edge cases related to these issues that cause either false bathymetric signal points, or cause the algorithm to miss valid bathymetric data. These data issues could present an issue for the scaling up the signal finding without any manual intervention. Currently the process is run without any intervention, but the sites are small enough to manually check several of the transects.

The following known data issues could effect the results of the KDE signal finding algorithm:

\subsubsection{Clouds}

The presence of some clouds along a single granule can cause the loss of data that might otherwise be valid

Clouds reflect sunlight which causes a higher background photon rate, and this can create issues with the telemetry bands and cause the telemetry bands to not include the surface. Even if the actual earth surface is included in the telemetry band, the clouds can affect the travel times and create inaccurate readings \parencite{atl03knownissues}.

During processing from L0 to L1, if the elevation from NASA reference DEM is not within the telemetry bands, no photons will be classified as signal. Therefore, if the entire granule is affected by this issue, there will be no sea surface found and therefore the entire granule will be filtered out. This can cause a significant loss of data but it is an issue inherent to nearly any remote-sensing based approach. One possible way to mitigate this would be to combine the ICESat-2 bathymetry data with synthetic aperture radar (SAR) remote sensing data. SAR remote sensing data can penetrate clouds because it uses radiometry outside of the visible spectrum. Although the spectrum used in SAR sensing cannot directly penetrate even clear water, the data can be used to estimate wave conditions, and the bathymetry can be estimated based on the transformation of the wave (called \emph{wave kinematic satellite-derived bathymetry}). The bathymetry estimates from SAR-derived wave kinematic SDB could be incorporated into the Kalman filtering step. 

In situations where the photons are able to pass through clouds, the changes to travel time through the clouds can affect the accuracy. This could potentially be something that is hard to detect and affect the accuracy of the bathymetric points if any are found.
 

\subsubsection{Multiple Telemetry Bands}

If the signal detection on board the satellite cannot determine where the primary surface is located, it will open another telemetry band to try to collect more signal. This can create other areas of photons that are significantly above or below the surface. The effect of is shown in figure \ref{fig:multiple_tel_bands}

