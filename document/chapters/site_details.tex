\newcommand{\sitesummary}[1]{

\section{#1 test site}

\subsection{Site conditions summary}

\subsection{Validation data}

\subsection{ICESat-2 Transects within AOI}
The AOI used the validate the method is shown in figure \ref{fig:#1_transects} below.
\begin{figure}[h]
    \centering
    \includegraphics[width=0.5\textwidth]{figures/#1_tracklines.pdf}
    \caption{Location of ICESat-2 Transects in #1 study area}
    \label{fig:#1_transects}
\end{figure}
\subsection{Error in lidar photon data}
After extracting the bathymetry from the ICESat-2 transects over the study site, the RMS error between the measured data from the USGS and the ICESat-2 data is
\begin{figure}[h]
    \centering
    \includegraphics[width=0.5\textwidth]{figures/#1_lidar_estimated_vs_truth.pdf}
    \caption{Comparison between depth calculated with ICESat-2 and true bathy}
    \label{fig:#1_truth_vs_measured_points}
\end{figure}
\subsection{Distribution of detected bathymetric photons}

Figure \ref{fig:#1-bathyphotonmap} shows the location and estimated bathymetric depths of individual ICESat-2 photon returns.

\begin{figure}[h]
    \centering
    \includegraphics[width=\textwidth]{figures/Florida_keys_photon_map.pdf}
    \caption{Geographic distribution of Bathymetric photon data}
    \label{fig:#1-bathyphotonmap}
\end{figure}

\subsection{Lidar updated GEBCO vs. Simple bilinear Interpolation}

\input{tables/#1_kalman_improvement.tex}
}

% \sitesummary{florida\_keys}