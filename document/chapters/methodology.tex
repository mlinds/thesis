\chapter{Data Sources and Methods}
\pdfcomment{provide information about ocean color data}
\pdfcomment{Explain how my processing chain differs from the NASA one}
\pdfcomment{make sure to explain how I add the ATL03 parameters to find the geoidal height}
\pdfcomment{mention the UTM grid reprojection steps if they're not mentioned.}
\section{Data Sources}
The following data sources were used as input to the 
\subsection{NASA ATL03}

The main source of data is NASA's ATL03 V005 data product \parencite{icesat2data}. ATL03 is a level 1 data product that consists of the precise latitude, longitude, and elevation for each received photon. As a level 1 data product, it has already undergone some processing by the Atlas Science Algorithm Software (ASAS) to correct for instrument errors, to classify photons as likely signal or noise for different surface types, and to correct for some geophysical effects including earth tides to provide provide measurements relative to the WGS-84 ellipsoid. 

Additionally, the data includes variables that allow for further corrections and adjustments to the tide-free geoid reference system. These additional variables include correction factors for the tide, ocean surface depression due to atmospheric pressure, and factors to convert the ellipsoidal elevation to a height relative to the tide-free geoid.

\subsection{GEBCO Global Grid}

The General Bathymetric Chart of the Ocean (GEBCO) is a global grid of topography and bathymetry at a 30 arc-second mesh resolution \parencite{gebco2021griddata}. GEBCO is assembled by compiling many different data sources, including mutli-beam sonar data, nautical charts, and satellite gravimetric measurements for deep-ocean bathymetry \parencite{gebcocookbook}. The elevation data is referenced to a vaguely-defined 'mean sea level'. The various data sets included in gebco are all assumed to be referenced to MSL, but some datasets referenced to chart datum are included. 

GEBCO has a limited accuracy and resolution, but it is the only available data source in many places in the world, so it is sometimes used as the best-estimate in very data-poor sites. However, the accuracy of GEBCO varies depending on the input data sources. In this project, the GEBCO elevation is used both to filter locations that \emph{may} contain valid bathymetry, and used as a prior guess to the bayesian updating approach. 

\subsection{GlobColour Daily Secchi Disk Depth Data}

To investigate the relationship between the water clarity and the availablity and quality of the bathymetric data from the spaceborne lidar, data from \citeauthor{Garnesson2019} is also linked to each transect to investigate the relationship between the Secchi Disk Depth \pdfcomment{does this need to be defined in the background section} along the transect. The exact data product used is \cite{dailysecchidata}. The data is accessed via the OPeNDAP protocol hosted by the Copernicus Marine Service, using the Xarray python package \parencite{hoyer_stephan_2022_6323468,hoyer2017xarray}.



\section{Methodology}

\pdfcomment{add a figure for each image}

To reach the end goal of incorporating ICESat-2 into GEBCO grids, first the lidar photon data for the area of interest is downloaded, processed into geoidal heights, subset to only include subsurface photons, find bathymetric signal within the subsurface data, interpolate into a 2D grid, then finally combine the interpolated ICESat-2 data with the resampled GEBCO grid within the area of interest. Then, for the test sites, the change in in the various error metrics between the naive bilinear interpolation is calculated. 


\begin{figure}[h!]
    \centering
    \import{figures/drawio}{Methodology_overview.drawio.svg.pdf_tex}
    \caption{Overview of the basic methodology}
    \label{fig:methodology-overview}
\end{figure}

\subsection{Processing ATL03}
To download data for a specific site geographic area of interest is created, and this area is passed to the NSIDC download API to use for spatial subsetting. This allows spatial subsetting of the data download which reduces file size. The NSIDC API also allows subsetting by the variable name, so only the ATL03 variables that are relevant for this research are downloaded, which further reduces the file size for practical download and storage of the ATL03 data. 

The three variables that define the 3D location in the WGS-84 reference frame of each photon are \emph{h\_ph}, \emph{lat\_ph}, and \emph{lon\_ph} are located in the \emph{heights} group within the ATL03 data structure. To use these variables for the purpose of bathymetric measurement, several other variables are required for processing. To transform the ellipsoidal elevation to the geoidal elevation, the two additive factors \emph{geoid} and \emph{geoid\_free2mean} are included in the download.

These correction factors are not provided for every photon but are provided for each 20m segment because they vary at scales longer than the nominal 0.7m between each photon. To find the correct adjustments factors for each photon, we need to match the segment-rate variables to the photon rate variables. This is done using the python package Pandas \parencite{jeff_reback_2022_6408044,mckinney-proc-scipy-2010} which has functions for joining time series data.  The segment-rate variable for each photon is determined using the Pandas dataframe \emph{asof()} method to find the closest segment rate variables in time to each photon. 

\begin{figure}[h!]
    \centering
    \includegraphics[width=0.75\textwidth]{figures/reference_photon_plot.jpg}
    \caption{Relation between the regular photons, and the 20m segment rate variables}
    \label{fig:reference-photon_match} 
\end{figure}

\subsection{Filtering ATL03 to subsurface returns}

The bathymetric signal that we are seeking to find is located in the shallow-water nearshore zone. Therefore, photons outside that zone need to be removed reduce processing time and eliminate false positives as much as possible. To reduce the downloaded transect data within the area of interest, the following filtering steps are applied to each transect \pdfcomment{Add figures For each of these steps}:

\begin{enumerate}
    \item For every photon, find the GEBCO depth at that location. Any photons with a GEBCO elevation between -50 and 3m are selected, and those outside of this region are culled from the data set. Points that are outside this range are assumed to be deeper than the maximum known depth detectable by ICESat-2 (38m per \citeauthor{Parrish2019}), or assumed to be on land. This provides a horizontal filtering along the transect. An example of this process can be seen in figure \ref{fig:gebco_filtering}
    
    \begin{figure}[h!]
        \centering
        \includegraphics[width=\textwidth]{figures/methodology_gebco_filtering.jpg}
        \caption{The GEBCO data for the example transect, and the photons that are removed due to the GEBCO depth}
        \label{fig:gebco_filtering}
    \end{figure}

    \item To remove any high noise, cloud returns, or any remaining high land points not removed in step 1, any points more than 5m above the geoid are removed, based on the approach in \citeauthor{Ranndal2021}.
    \item The local sea-surface elevation $h_{sea}$ is calculated by taking the median elevation of photons that are classified as high-confidence sea surface photons. The water depth for each photon is then calculated. The standard deviation of the elevation high-confidence photons $\sigma_{h_{sea}}$ is also calculated to estimate the magnitude of the wave height at the time of the observation.
    \item Any points with a water depth greater than 40 meters, and points with an geoidal height  less than -40m are removed, based on the same assumption that they are too deep to by bathymetric points. 
    \item Any points that are higher than $h_{sea} + max(\sigma_{h_{sea}},1)$ are removed. 
    
    
    The results of steps 3-5 \pdfcomment{check} are shown in figure \ref{fig:vert_filtering}
    \pdfcomment{fix legend on this graph}
    \begin{figure}[h!]
        \centering
        \includegraphics[width=\textwidth]{figures/methodology_sealvl_filtering.jpg}
        \caption{Vertical point filtering based on the local sea surface elevation}
        \label{fig:vert_filtering}
    \end{figure}
\end{enumerate}

After these filtering steps, the resulting subsurface photons for this example transect are shown in figure \ref{fig:remaing_photons}. The bathymetric signal can be seen clearly throughout the entire transect.

\begin{figure}[h!]
    \centering
    \includegraphics[width=\textwidth]{figures/methodology_reminaing_after_filtering.jpg}
    \caption{Subsurface photons found resulting after the filtering process}
    \label{fig:remaing_photons}
\end{figure}

An overview of the entire filtering chain is shown in figure \ref{fig:filtering-flowchart}

\begin{figure}[h!]
    \centering
    % %LaTeX with PSTricks extensions
%%Creator: Inkscape 1.2 (1:1.2.1+202207142221+cd75a1ee6d)
%%Please note this file requires PSTricks extensions
\psset{xunit=.5pt,yunit=.5pt,runit=.5pt}
\begin{pspicture}(391,511)
{
\newrgbcolor{curcolor}{0 0 0}
\pscustom[linewidth=1,linecolor=curcolor]
{
\newpath
\moveto(90.5,480.5)
\lineto(89.5,480.5)
\lineto(89.5,467.47)
}
}
{
\newrgbcolor{curcolor}{0 0 0}
\pscustom[linestyle=none,fillstyle=solid,fillcolor=curcolor]
{
\newpath
\moveto(89.5,462.22)
\lineto(86,469.22)
\lineto(89.5,467.47)
\lineto(93,469.22)
\closepath
}
}
{
\newrgbcolor{curcolor}{0 0 0}
\pscustom[linewidth=1,linecolor=curcolor]
{
\newpath
\moveto(89.5,462.22)
\lineto(86,469.22)
\lineto(89.5,467.47)
\lineto(93,469.22)
\closepath
}
}
{
\newrgbcolor{curcolor}{1 1 1}
\pscustom[linestyle=none,fillstyle=solid,fillcolor=curcolor]
{
\newpath
\moveto(0.5,480.5)
\lineto(20.5,510.5)
\lineto(120.5,510.5)
\lineto(100.5,480.5)
\closepath
}
}
{
\newrgbcolor{curcolor}{0 0 0}
\pscustom[linewidth=1,linecolor=curcolor]
{
\newpath
\moveto(0.5,480.5)
\lineto(20.5,510.5)
\lineto(120.5,510.5)
\lineto(100.5,480.5)
\closepath
}
}
{
\newrgbcolor{curcolor}{0 0 0}
\pscustom[linestyle=none,fillstyle=solid,fillcolor=curcolor]
{
\newpath
\moveto(28.44007901,494.628)
\lineto(29.34007901,492)
\lineto(30.58807901,492)
\lineto(27.51607901,500.748)
\lineto(26.07607901,500.748)
\lineto(22.95607901,492)
\lineto(24.14407901,492)
\lineto(25.06807901,494.628)
\closepath
\moveto(28.12807901,495.564)
\lineto(25.34407901,495.564)
\lineto(26.78407901,499.548)
\closepath
}
}
{
\newrgbcolor{curcolor}{0 0 0}
\pscustom[linestyle=none,fillstyle=solid,fillcolor=curcolor]
{
\newpath
\moveto(33.88807187,499.764)
\lineto(36.75607187,499.764)
\lineto(36.75607187,500.748)
\lineto(29.89207187,500.748)
\lineto(29.89207187,499.764)
\lineto(32.77207187,499.764)
\lineto(32.77207187,492)
\lineto(33.88807187,492)
\closepath
}
}
{
\newrgbcolor{curcolor}{0 0 0}
\pscustom[linestyle=none,fillstyle=solid,fillcolor=curcolor]
{
\newpath
\moveto(38.90403726,500.748)
\lineto(37.78803726,500.748)
\lineto(37.78803726,492)
\lineto(43.22403726,492)
\lineto(43.22403726,492.984)
\lineto(38.90403726,492.984)
\closepath
}
}
{
\newrgbcolor{curcolor}{0 0 0}
\pscustom[linestyle=none,fillstyle=solid,fillcolor=curcolor]
{
\newpath
\moveto(46.80003433,500.676)
\curveto(46.00803433,500.676)(45.28803433,500.316)(44.84403433,499.728)
\curveto(44.29203433,498.96)(44.01603433,497.808)(44.01603433,496.2)
\curveto(44.01603433,493.272)(44.97603433,491.724)(46.80003433,491.724)
\curveto(48.60003433,491.724)(49.58403433,493.272)(49.58403433,496.128)
\curveto(49.58403433,497.808)(49.32003433,498.936)(48.75603433,499.728)
\curveto(48.31203433,500.328)(47.60403433,500.676)(46.80003433,500.676)
\closepath
\moveto(46.80003433,499.74)
\curveto(47.94003433,499.74)(48.50403433,498.576)(48.50403433,496.224)
\curveto(48.50403433,493.752)(47.95203433,492.6)(46.77603433,492.6)
\curveto(45.66003433,492.6)(45.09603433,493.8)(45.09603433,496.188)
\curveto(45.09603433,498.576)(45.66003433,499.74)(46.80003433,499.74)
\closepath
}
}
{
\newrgbcolor{curcolor}{0 0 0}
\pscustom[linestyle=none,fillstyle=solid,fillcolor=curcolor]
{
\newpath
\moveto(52.8240314,495.996)
\lineto(52.9560314,495.996)
\lineto(53.4000314,495.996)
\curveto(54.5520314,495.996)(55.1640314,495.456)(55.1640314,494.412)
\curveto(55.1640314,493.32)(54.5040314,492.66)(53.4120314,492.66)
\curveto(52.2480314,492.66)(51.6840314,493.248)(51.6120314,494.52)
\lineto(50.5560314,494.52)
\curveto(50.6040314,493.824)(50.7240314,493.368)(50.9280314,492.984)
\curveto(51.3720314,492.144)(52.2000314,491.724)(53.3640314,491.724)
\curveto(55.1160314,491.724)(56.2440314,492.78)(56.2440314,494.424)
\curveto(56.2440314,495.528)(55.8240314,496.128)(54.8040314,496.488)
\curveto(55.5960314,496.812)(55.9920314,497.412)(55.9920314,498.288)
\curveto(55.9920314,499.776)(55.0200314,500.676)(53.4000314,500.676)
\curveto(51.6840314,500.676)(50.7720314,499.716)(50.7360314,497.88)
\lineto(51.7920314,497.88)
\curveto(51.8040314,498.408)(51.8520314,498.708)(51.9840314,498.972)
\curveto(52.2240314,499.464)(52.7520314,499.752)(53.4120314,499.752)
\curveto(54.3480314,499.752)(54.9120314,499.188)(54.9120314,498.252)
\curveto(54.9120314,497.64)(54.6960314,497.268)(54.2280314,497.064)
\curveto(53.9400314,496.944)(53.5560314,496.896)(52.8240314,496.884)
\closepath
}
}
{
\newrgbcolor{curcolor}{0 0 0}
\pscustom[linestyle=none,fillstyle=solid,fillcolor=curcolor]
{
\newpath
\moveto(61.00802701,498.288)
\lineto(61.00802701,492)
\lineto(62.01602701,492)
\lineto(62.01602701,495.264)
\curveto(62.02802701,496.776)(62.65202701,497.448)(64.03202701,497.412)
\lineto(64.03202701,498.432)
\curveto(63.86402701,498.456)(63.76802701,498.468)(63.64802701,498.468)
\curveto(63.00002701,498.468)(62.50802701,498.084)(61.93202701,497.148)
\lineto(61.93202701,498.288)
\closepath
}
}
{
\newrgbcolor{curcolor}{0 0 0}
\pscustom[linestyle=none,fillstyle=solid,fillcolor=curcolor]
{
\newpath
\moveto(70.21200357,494.808)
\curveto(70.21200357,495.768)(70.14000357,496.344)(69.96000357,496.812)
\curveto(69.55200357,497.844)(68.59200357,498.468)(67.41600357,498.468)
\curveto(65.66400357,498.468)(64.53600357,497.136)(64.53600357,495.06)
\curveto(64.53600357,492.984)(65.61600357,491.724)(67.39200357,491.724)
\curveto(68.83200357,491.724)(69.82800357,492.54)(70.08000357,493.908)
\lineto(69.07200357,493.908)
\curveto(68.79600357,493.08)(68.23200357,492.648)(67.42800357,492.648)
\curveto(66.79200357,492.648)(66.25200357,492.936)(65.91600357,493.464)
\curveto(65.67600357,493.824)(65.59200357,494.184)(65.58000357,494.808)
\closepath
\moveto(65.60400357,495.624)
\curveto(65.68800357,496.788)(66.39600357,497.544)(67.40400357,497.544)
\curveto(68.42400357,497.544)(69.13200357,496.752)(69.13200357,495.624)
\closepath
}
}
{
\newrgbcolor{curcolor}{0 0 0}
\pscustom[linestyle=none,fillstyle=solid,fillcolor=curcolor]
{
\newpath
\moveto(73.6679913,498.288)
\lineto(72.6359913,498.288)
\lineto(72.6359913,500.016)
\lineto(71.6399913,500.016)
\lineto(71.6399913,498.288)
\lineto(70.7879913,498.288)
\lineto(70.7879913,497.472)
\lineto(71.6399913,497.472)
\lineto(71.6399913,492.72)
\curveto(71.6399913,492.072)(72.0719913,491.724)(72.8519913,491.724)
\curveto(73.1159913,491.724)(73.3319913,491.748)(73.6679913,491.808)
\lineto(73.6679913,492.648)
\curveto(73.5239913,492.612)(73.3919913,492.6)(73.1879913,492.6)
\curveto(72.7559913,492.6)(72.6359913,492.72)(72.6359913,493.164)
\lineto(72.6359913,497.472)
\lineto(73.6679913,497.472)
\closepath
}
}
{
\newrgbcolor{curcolor}{0 0 0}
\pscustom[linestyle=none,fillstyle=solid,fillcolor=curcolor]
{
\newpath
\moveto(79.69197043,492)
\lineto(79.69197043,498.288)
\lineto(78.69597043,498.288)
\lineto(78.69597043,494.724)
\curveto(78.69597043,493.44)(78.02397043,492.6)(76.97997043,492.6)
\curveto(76.18797043,492.6)(75.68397043,493.08)(75.68397043,493.836)
\lineto(75.68397043,498.288)
\lineto(74.68797043,498.288)
\lineto(74.68797043,493.44)
\curveto(74.68797043,492.396)(75.46797043,491.724)(76.69197043,491.724)
\curveto(77.61597043,491.724)(78.20397043,492.048)(78.79197043,492.876)
\lineto(78.79197043,492)
\closepath
}
}
{
\newrgbcolor{curcolor}{0 0 0}
\pscustom[linestyle=none,fillstyle=solid,fillcolor=curcolor]
{
\newpath
\moveto(81.4079675,498.288)
\lineto(81.4079675,492)
\lineto(82.4159675,492)
\lineto(82.4159675,495.264)
\curveto(82.4279675,496.776)(83.0519675,497.448)(84.4319675,497.412)
\lineto(84.4319675,498.432)
\curveto(84.2639675,498.456)(84.1679675,498.468)(84.0479675,498.468)
\curveto(83.3999675,498.468)(82.9079675,498.084)(82.3319675,497.148)
\lineto(82.3319675,498.288)
\closepath
}
}
{
\newrgbcolor{curcolor}{0 0 0}
\pscustom[linestyle=none,fillstyle=solid,fillcolor=curcolor]
{
\newpath
\moveto(85.41594681,498.288)
\lineto(85.41594681,492)
\lineto(86.42394681,492)
\lineto(86.42394681,495.468)
\curveto(86.42394681,496.752)(87.09594681,497.592)(88.12794681,497.592)
\curveto(88.91994681,497.592)(89.42394681,497.112)(89.42394681,496.356)
\lineto(89.42394681,492)
\lineto(90.41994681,492)
\lineto(90.41994681,496.752)
\curveto(90.41994681,497.796)(89.63994681,498.468)(88.42794681,498.468)
\curveto(87.49194681,498.468)(86.89194681,498.108)(86.33994681,497.232)
\lineto(86.33994681,498.288)
\closepath
}
}
{
\newrgbcolor{curcolor}{0 0 0}
\pscustom[linestyle=none,fillstyle=solid,fillcolor=curcolor]
{
\newpath
\moveto(96.50394388,496.536)
\curveto(96.49194388,497.772)(95.67594388,498.468)(94.22394388,498.468)
\curveto(92.75994388,498.468)(91.81194388,497.712)(91.81194388,496.548)
\curveto(91.81194388,495.564)(92.31594388,495.096)(93.80394388,494.736)
\lineto(94.73994388,494.508)
\curveto(95.43594388,494.34)(95.71194388,494.088)(95.71194388,493.644)
\curveto(95.71194388,493.044)(95.12394388,492.648)(94.24794388,492.648)
\curveto(93.70794388,492.648)(93.25194388,492.804)(92.99994388,493.068)
\curveto(92.84394388,493.248)(92.77194388,493.428)(92.71194388,493.872)
\lineto(91.65594388,493.872)
\curveto(91.70394388,492.42)(92.51994388,491.724)(94.16394388,491.724)
\curveto(95.74794388,491.724)(96.75594388,492.504)(96.75594388,493.716)
\curveto(96.75594388,494.652)(96.22794388,495.168)(94.97994388,495.468)
\lineto(94.01994388,495.696)
\curveto(93.20394388,495.888)(92.85594388,496.152)(92.85594388,496.596)
\curveto(92.85594388,497.184)(93.37194388,497.544)(94.18794388,497.544)
\curveto(94.99194388,497.544)(95.42394388,497.196)(95.44794388,496.536)
\closepath
}
}
{
\newrgbcolor{curcolor}{0 0 0}
\pscustom[linewidth=1,linecolor=curcolor]
{
\newpath
\moveto(110.53,400.5)
\lineto(110.53,380.5)
\lineto(110.53,390.5)
\lineto(110.53,376.87)
}
}
{
\newrgbcolor{curcolor}{0 0 0}
\pscustom[linestyle=none,fillstyle=solid,fillcolor=curcolor]
{
\newpath
\moveto(110.53,371.62)
\lineto(107.03,378.62)
\lineto(110.53,376.87)
\lineto(114.03,378.62)
\closepath
}
}
{
\newrgbcolor{curcolor}{0 0 0}
\pscustom[linewidth=1,linecolor=curcolor]
{
\newpath
\moveto(110.53,371.62)
\lineto(107.03,378.62)
\lineto(110.53,376.87)
\lineto(114.03,378.62)
\closepath
}
}
{
\newrgbcolor{curcolor}{1 1 1}
\pscustom[linestyle=none,fillstyle=solid,fillcolor=curcolor]
{
\newpath
\moveto(50.5,460.5)
\lineto(170.5,460.5)
\lineto(170.5,400.5)
\lineto(50.5,400.5)
\closepath
}
}
{
\newrgbcolor{curcolor}{0 0 0}
\pscustom[linewidth=1,linecolor=curcolor]
{
\newpath
\moveto(50.5,460.5)
\lineto(170.5,460.5)
\lineto(170.5,400.5)
\lineto(50.5,400.5)
\closepath
}
}
{
\newrgbcolor{curcolor}{0 0 0}
\pscustom[linestyle=none,fillstyle=solid,fillcolor=curcolor]
{
\newpath
\moveto(50.30611829,429.628)
\lineto(51.20611829,427)
\lineto(52.45411829,427)
\lineto(49.38211829,435.748)
\lineto(47.94211829,435.748)
\lineto(44.82211829,427)
\lineto(46.01011829,427)
\lineto(46.93411829,429.628)
\closepath
\moveto(49.99411829,430.564)
\lineto(47.21011829,430.564)
\lineto(48.65011829,434.548)
\closepath
}
}
{
\newrgbcolor{curcolor}{0 0 0}
\pscustom[linestyle=none,fillstyle=solid,fillcolor=curcolor]
{
\newpath
\moveto(57.84211298,431.536)
\curveto(57.83011298,432.772)(57.01411298,433.468)(55.56211298,433.468)
\curveto(54.09811298,433.468)(53.15011298,432.712)(53.15011298,431.548)
\curveto(53.15011298,430.564)(53.65411298,430.096)(55.14211298,429.736)
\lineto(56.07811298,429.508)
\curveto(56.77411298,429.34)(57.05011298,429.088)(57.05011298,428.644)
\curveto(57.05011298,428.044)(56.46211298,427.648)(55.58611298,427.648)
\curveto(55.04611298,427.648)(54.59011298,427.804)(54.33811298,428.068)
\curveto(54.18211298,428.248)(54.11011298,428.428)(54.05011298,428.872)
\lineto(52.99411298,428.872)
\curveto(53.04211298,427.42)(53.85811298,426.724)(55.50211298,426.724)
\curveto(57.08611298,426.724)(58.09411298,427.504)(58.09411298,428.716)
\curveto(58.09411298,429.652)(57.56611298,430.168)(56.31811298,430.468)
\lineto(55.35811298,430.696)
\curveto(54.54211298,430.888)(54.19411298,431.152)(54.19411298,431.596)
\curveto(54.19411298,432.184)(54.71011298,432.544)(55.52611298,432.544)
\curveto(56.33011298,432.544)(56.76211298,432.196)(56.78611298,431.536)
\closepath
}
}
{
\newrgbcolor{curcolor}{0 0 0}
\pscustom[linestyle=none,fillstyle=solid,fillcolor=curcolor]
{
\newpath
\moveto(63.8660769,431.536)
\curveto(63.8540769,432.772)(63.0380769,433.468)(61.5860769,433.468)
\curveto(60.1220769,433.468)(59.1740769,432.712)(59.1740769,431.548)
\curveto(59.1740769,430.564)(59.6780769,430.096)(61.1660769,429.736)
\lineto(62.1020769,429.508)
\curveto(62.7980769,429.34)(63.0740769,429.088)(63.0740769,428.644)
\curveto(63.0740769,428.044)(62.4860769,427.648)(61.6100769,427.648)
\curveto(61.0700769,427.648)(60.6140769,427.804)(60.3620769,428.068)
\curveto(60.2060769,428.248)(60.1340769,428.428)(60.0740769,428.872)
\lineto(59.0180769,428.872)
\curveto(59.0660769,427.42)(59.8820769,426.724)(61.5260769,426.724)
\curveto(63.1100769,426.724)(64.1180769,427.504)(64.1180769,428.716)
\curveto(64.1180769,429.652)(63.5900769,430.168)(62.3420769,430.468)
\lineto(61.3820769,430.696)
\curveto(60.5660769,430.888)(60.2180769,431.152)(60.2180769,431.596)
\curveto(60.2180769,432.184)(60.7340769,432.544)(61.5500769,432.544)
\curveto(62.3540769,432.544)(62.7860769,432.196)(62.8100769,431.536)
\closepath
}
}
{
\newrgbcolor{curcolor}{0 0 0}
\pscustom[linestyle=none,fillstyle=solid,fillcolor=curcolor]
{
\newpath
\moveto(66.41005402,433.288)
\lineto(65.41405402,433.288)
\lineto(65.41405402,427)
\lineto(66.41005402,427)
\closepath
\moveto(66.41005402,435.748)
\lineto(65.40205402,435.748)
\lineto(65.40205402,434.488)
\lineto(66.41005402,434.488)
\closepath
}
}
{
\newrgbcolor{curcolor}{0 0 0}
\pscustom[linestyle=none,fillstyle=solid,fillcolor=curcolor]
{
\newpath
\moveto(72.1100163,433.288)
\lineto(72.1100163,432.376)
\curveto(71.6060163,433.132)(71.0540163,433.468)(70.2620163,433.468)
\curveto(68.7380163,433.468)(67.6820163,432.052)(67.6820163,430.036)
\curveto(67.6820163,428.98)(67.9340163,428.212)(68.4740163,427.576)
\curveto(68.9420163,427.024)(69.5420163,426.724)(70.1900163,426.724)
\curveto(70.9460163,426.724)(71.4860163,427.06)(72.0140163,427.852)
\lineto(72.0140163,427.528)
\curveto(72.0140163,425.848)(71.5460163,425.224)(70.2980163,425.224)
\curveto(69.4460163,425.224)(69.0020163,425.56)(68.9060163,426.28)
\lineto(67.8860163,426.28)
\curveto(67.9820163,425.116)(68.9060163,424.384)(70.2740163,424.384)
\curveto(71.1980163,424.384)(71.9660163,424.684)(72.3740163,425.188)
\curveto(72.8540163,425.776)(73.0340163,426.556)(73.0340163,428.032)
\lineto(73.0340163,433.288)
\closepath
\moveto(70.3580163,432.544)
\curveto(71.4140163,432.544)(72.0140163,431.656)(72.0140163,430.06)
\curveto(72.0140163,428.536)(71.4020163,427.648)(70.3580163,427.648)
\curveto(69.3260163,427.648)(68.7260163,428.548)(68.7260163,430.096)
\curveto(68.7260163,431.632)(69.3260163,432.544)(70.3580163,432.544)
\closepath
}
}
{
\newrgbcolor{curcolor}{0 0 0}
\pscustom[linestyle=none,fillstyle=solid,fillcolor=curcolor]
{
\newpath
\moveto(74.77401337,433.288)
\lineto(74.77401337,427)
\lineto(75.78201337,427)
\lineto(75.78201337,430.468)
\curveto(75.78201337,431.752)(76.45401337,432.592)(77.48601337,432.592)
\curveto(78.27801337,432.592)(78.78201337,432.112)(78.78201337,431.356)
\lineto(78.78201337,427)
\lineto(79.77801337,427)
\lineto(79.77801337,431.752)
\curveto(79.77801337,432.796)(78.99801337,433.468)(77.78601337,433.468)
\curveto(76.85001337,433.468)(76.25001337,433.108)(75.69801337,432.232)
\lineto(75.69801337,433.288)
\closepath
}
}
{
\newrgbcolor{curcolor}{0 0 0}
\pscustom[linestyle=none,fillstyle=solid,fillcolor=curcolor]
{
\newpath
\moveto(92.45000897,431.62)
\lineto(88.80200897,431.62)
\lineto(88.80200897,430.636)
\lineto(91.46600897,430.636)
\lineto(91.46600897,430.396)
\curveto(91.46600897,428.836)(90.31400897,427.708)(88.71800897,427.708)
\curveto(87.83000897,427.708)(87.02600897,428.032)(86.51000897,428.596)
\curveto(85.93400897,429.22)(85.58600897,430.264)(85.58600897,431.344)
\curveto(85.58600897,433.492)(86.81000897,434.908)(88.65800897,434.908)
\curveto(89.99000897,434.908)(90.95000897,434.224)(91.19000897,433.096)
\lineto(92.33000897,433.096)
\curveto(92.01800897,434.872)(90.67400897,435.892)(88.67000897,435.892)
\curveto(87.60200897,435.892)(86.73800897,435.616)(86.05400897,435.052)
\curveto(85.03400897,434.212)(84.47000897,432.856)(84.47000897,431.284)
\curveto(84.47000897,428.596)(86.11400897,426.724)(88.47800897,426.724)
\curveto(89.66600897,426.724)(90.60200897,427.168)(91.46600897,428.116)
\lineto(91.74200897,426.952)
\lineto(92.45000897,426.952)
\closepath
}
}
{
\newrgbcolor{curcolor}{0 0 0}
\pscustom[linestyle=none,fillstyle=solid,fillcolor=curcolor]
{
\newpath
\moveto(95.47400751,430.984)
\lineto(100.23800751,430.984)
\lineto(100.23800751,431.968)
\lineto(95.47400751,431.968)
\lineto(95.47400751,434.764)
\lineto(100.41800751,434.764)
\lineto(100.41800751,435.748)
\lineto(94.35800751,435.748)
\lineto(94.35800751,427)
\lineto(100.63400751,427)
\lineto(100.63400751,427.984)
\lineto(95.47400751,427.984)
\closepath
}
}
{
\newrgbcolor{curcolor}{0 0 0}
\pscustom[linestyle=none,fillstyle=solid,fillcolor=curcolor]
{
\newpath
\moveto(102.23000531,427)
\lineto(106.17800531,427)
\curveto(107.00600531,427)(107.61800531,427.228)(108.08600531,427.732)
\curveto(108.51800531,428.188)(108.75800531,428.812)(108.75800531,429.496)
\curveto(108.75800531,430.552)(108.27800531,431.188)(107.16200531,431.62)
\curveto(107.96600531,431.992)(108.37400531,432.628)(108.37400531,433.528)
\curveto(108.37400531,434.176)(108.13400531,434.728)(107.67800531,435.136)
\curveto(107.21000531,435.556)(106.62200531,435.748)(105.78200531,435.748)
\lineto(102.23000531,435.748)
\closepath
\moveto(103.34600531,431.98)
\lineto(103.34600531,434.764)
\lineto(105.50600531,434.764)
\curveto(106.13000531,434.764)(106.47800531,434.68)(106.77800531,434.452)
\curveto(107.09000531,434.212)(107.25800531,433.852)(107.25800531,433.372)
\curveto(107.25800531,432.892)(107.09000531,432.532)(106.77800531,432.292)
\curveto(106.47800531,432.064)(106.13000531,431.98)(105.50600531,431.98)
\closepath
\moveto(103.34600531,427.984)
\lineto(103.34600531,430.996)
\lineto(106.07000531,430.996)
\curveto(107.05400531,430.996)(107.64200531,430.432)(107.64200531,429.484)
\curveto(107.64200531,428.548)(107.05400531,427.984)(106.07000531,427.984)
\closepath
}
}
{
\newrgbcolor{curcolor}{0 0 0}
\pscustom[linestyle=none,fillstyle=solid,fillcolor=curcolor]
{
\newpath
\moveto(117.12199377,433.036)
\curveto(116.77399377,434.956)(115.66999377,435.892)(113.74999377,435.892)
\curveto(112.57399377,435.892)(111.62599377,435.52)(110.97799377,434.8)
\curveto(110.18599377,433.936)(109.75399377,432.688)(109.75399377,431.272)
\curveto(109.75399377,429.832)(110.19799377,428.596)(111.01399377,427.744)
\curveto(111.69799377,427.048)(112.56199377,426.724)(113.70199377,426.724)
\curveto(115.83799377,426.724)(117.03799377,427.876)(117.30199377,430.192)
\lineto(116.14999377,430.192)
\curveto(116.05399377,429.592)(115.93399377,429.184)(115.75399377,428.836)
\curveto(115.39399377,428.116)(114.64999377,427.708)(113.71399377,427.708)
\curveto(111.97399377,427.708)(110.86999377,429.1)(110.86999377,431.284)
\curveto(110.86999377,433.528)(111.91399377,434.908)(113.61799377,434.908)
\curveto(114.32599377,434.908)(114.98599377,434.704)(115.34599377,434.356)
\curveto(115.66999377,434.056)(115.84999377,433.696)(115.98199377,433.036)
\closepath
}
}
{
\newrgbcolor{curcolor}{0 0 0}
\pscustom[linestyle=none,fillstyle=solid,fillcolor=curcolor]
{
\newpath
\moveto(122.41395642,435.892)
\curveto(119.90595642,435.892)(118.20195642,434.044)(118.20195642,431.308)
\curveto(118.20195642,428.56)(119.89395642,426.724)(122.42595642,426.724)
\curveto(123.49395642,426.724)(124.42995642,427.048)(125.13795642,427.648)
\curveto(126.08595642,428.452)(126.64995642,429.808)(126.64995642,431.236)
\curveto(126.64995642,434.056)(124.98195642,435.892)(122.41395642,435.892)
\closepath
\moveto(122.41395642,434.908)
\curveto(124.30995642,434.908)(125.53395642,433.48)(125.53395642,431.26)
\curveto(125.53395642,429.148)(124.27395642,427.708)(122.42595642,427.708)
\curveto(120.55395642,427.708)(119.31795642,429.148)(119.31795642,431.308)
\curveto(119.31795642,433.468)(120.55395642,434.908)(122.41395642,434.908)
\closepath
}
}
{
\newrgbcolor{curcolor}{0 0 0}
\pscustom[linestyle=none,fillstyle=solid,fillcolor=curcolor]
{
\newpath
\moveto(133.83795349,427)
\lineto(136.24995349,433.288)
\lineto(135.12195349,433.288)
\lineto(133.34595349,428.188)
\lineto(131.66595349,433.288)
\lineto(130.53795349,433.288)
\lineto(132.74595349,427)
\closepath
}
}
{
\newrgbcolor{curcolor}{0 0 0}
\pscustom[linestyle=none,fillstyle=solid,fillcolor=curcolor]
{
\newpath
\moveto(142.63392822,427.588)
\curveto(142.52592822,427.564)(142.47792822,427.564)(142.41792822,427.564)
\curveto(142.06992822,427.564)(141.87792822,427.744)(141.87792822,428.056)
\lineto(141.87792822,431.752)
\curveto(141.87792822,432.868)(141.06192822,433.468)(139.51392822,433.468)
\curveto(138.58992822,433.468)(137.85792822,433.204)(137.42592822,432.736)
\curveto(137.13792822,432.412)(137.01792822,432.052)(136.99392822,431.428)
\lineto(138.00192822,431.428)
\curveto(138.08592822,432.196)(138.54192822,432.544)(139.47792822,432.544)
\curveto(140.38992822,432.544)(140.88192822,432.208)(140.88192822,431.608)
\lineto(140.88192822,431.344)
\curveto(140.86992822,430.912)(140.65392822,430.756)(139.83792822,430.648)
\curveto(138.42192822,430.468)(138.20592822,430.42)(137.82192822,430.264)
\curveto(137.08992822,429.952)(136.71792822,429.4)(136.71792822,428.584)
\curveto(136.71792822,427.444)(137.50992822,426.724)(138.78192822,426.724)
\curveto(139.57392822,426.724)(140.20992822,427)(140.91792822,427.648)
\curveto(140.98992822,427)(141.30192822,426.724)(141.94992822,426.724)
\curveto(142.16592822,426.724)(142.29792822,426.748)(142.63392822,426.832)
\closepath
\moveto(140.88192822,428.98)
\curveto(140.88192822,428.644)(140.78592822,428.44)(140.48592822,428.164)
\curveto(140.07792822,427.792)(139.58592822,427.6)(138.99792822,427.6)
\curveto(138.21792822,427.6)(137.76192822,427.972)(137.76192822,428.608)
\curveto(137.76192822,429.268)(138.19392822,429.604)(139.27392822,429.76)
\curveto(140.34192822,429.904)(140.54592822,429.952)(140.88192822,430.108)
\closepath
}
}
{
\newrgbcolor{curcolor}{0 0 0}
\pscustom[linestyle=none,fillstyle=solid,fillcolor=curcolor]
{
\newpath
\moveto(144.70992529,435.748)
\lineto(143.70192529,435.748)
\lineto(143.70192529,427)
\lineto(144.70992529,427)
\closepath
}
}
{
\newrgbcolor{curcolor}{0 0 0}
\pscustom[linestyle=none,fillstyle=solid,fillcolor=curcolor]
{
\newpath
\moveto(151.33390387,427)
\lineto(151.33390387,433.288)
\lineto(150.33790387,433.288)
\lineto(150.33790387,429.724)
\curveto(150.33790387,428.44)(149.66590387,427.6)(148.62190387,427.6)
\curveto(147.82990387,427.6)(147.32590387,428.08)(147.32590387,428.836)
\lineto(147.32590387,433.288)
\lineto(146.32990387,433.288)
\lineto(146.32990387,428.44)
\curveto(146.32990387,427.396)(147.10990387,426.724)(148.33390387,426.724)
\curveto(149.25790387,426.724)(149.84590387,427.048)(150.43390387,427.876)
\lineto(150.43390387,427)
\closepath
}
}
{
\newrgbcolor{curcolor}{0 0 0}
\pscustom[linestyle=none,fillstyle=solid,fillcolor=curcolor]
{
\newpath
\moveto(158.37790094,429.808)
\curveto(158.37790094,430.768)(158.30590094,431.344)(158.12590094,431.812)
\curveto(157.71790094,432.844)(156.75790094,433.468)(155.58190094,433.468)
\curveto(153.82990094,433.468)(152.70190094,432.136)(152.70190094,430.06)
\curveto(152.70190094,427.984)(153.78190094,426.724)(155.55790094,426.724)
\curveto(156.99790094,426.724)(157.99390094,427.54)(158.24590094,428.908)
\lineto(157.23790094,428.908)
\curveto(156.96190094,428.08)(156.39790094,427.648)(155.59390094,427.648)
\curveto(154.95790094,427.648)(154.41790094,427.936)(154.08190094,428.464)
\curveto(153.84190094,428.824)(153.75790094,429.184)(153.74590094,429.808)
\closepath
\moveto(153.76990094,430.624)
\curveto(153.85390094,431.788)(154.56190094,432.544)(155.56990094,432.544)
\curveto(156.58990094,432.544)(157.29790094,431.752)(157.29790094,430.624)
\closepath
}
}
{
\newrgbcolor{curcolor}{0 0 0}
\pscustom[linestyle=none,fillstyle=solid,fillcolor=curcolor]
{
\newpath
\moveto(165.27789655,433.288)
\lineto(164.24589655,433.288)
\lineto(164.24589655,435.016)
\lineto(163.24989655,435.016)
\lineto(163.24989655,433.288)
\lineto(162.39789655,433.288)
\lineto(162.39789655,432.472)
\lineto(163.24989655,432.472)
\lineto(163.24989655,427.72)
\curveto(163.24989655,427.072)(163.68189655,426.724)(164.46189655,426.724)
\curveto(164.72589655,426.724)(164.94189655,426.748)(165.27789655,426.808)
\lineto(165.27789655,427.648)
\curveto(165.13389655,427.612)(165.00189655,427.6)(164.79789655,427.6)
\curveto(164.36589655,427.6)(164.24589655,427.72)(164.24589655,428.164)
\lineto(164.24589655,432.472)
\lineto(165.27789655,432.472)
\closepath
}
}
{
\newrgbcolor{curcolor}{0 0 0}
\pscustom[linestyle=none,fillstyle=solid,fillcolor=curcolor]
{
\newpath
\moveto(167.66588611,428.248)
\lineto(166.41788611,428.248)
\lineto(166.41788611,427)
\lineto(167.66588611,427)
\closepath
}
}
{
\newrgbcolor{curcolor}{0 0 0}
\pscustom[linestyle=none,fillstyle=solid,fillcolor=curcolor]
{
\newpath
\moveto(171.00188464,428.248)
\lineto(169.75388464,428.248)
\lineto(169.75388464,427)
\lineto(171.00188464,427)
\closepath
}
}
{
\newrgbcolor{curcolor}{0 0 0}
\pscustom[linestyle=none,fillstyle=solid,fillcolor=curcolor]
{
\newpath
\moveto(174.33788318,428.248)
\lineto(173.08988318,428.248)
\lineto(173.08988318,427)
\lineto(174.33788318,427)
\closepath
}
}
{
\newrgbcolor{curcolor}{0 0 0}
\pscustom[linewidth=1,linecolor=curcolor]
{
\newpath
\moveto(149.25,480.5)
\lineto(149.79,469.14)
}
}
{
\newrgbcolor{curcolor}{0 0 0}
\pscustom[linestyle=none,fillstyle=solid,fillcolor=curcolor]
{
\newpath
\moveto(150.05,463.9)
\lineto(146.22,470.72)
\lineto(149.79,469.14)
\lineto(153.21,471.06)
\closepath
}
}
{
\newrgbcolor{curcolor}{0 0 0}
\pscustom[linewidth=1,linecolor=curcolor]
{
\newpath
\moveto(150.05,463.9)
\lineto(146.22,470.72)
\lineto(149.79,469.14)
\lineto(153.21,471.06)
\closepath
}
}
{
\newrgbcolor{curcolor}{1 1 1}
\pscustom[linestyle=none,fillstyle=solid,fillcolor=curcolor]
{
\newpath
\moveto(120.5,480.5)
\lineto(140.5,510.5)
\lineto(235.5,510.5)
\lineto(215.5,480.5)
\closepath
}
}
{
\newrgbcolor{curcolor}{0 0 0}
\pscustom[linewidth=1,linecolor=curcolor]
{
\newpath
\moveto(120.5,480.5)
\lineto(140.5,510.5)
\lineto(235.5,510.5)
\lineto(215.5,480.5)
\closepath
}
}
{
\newrgbcolor{curcolor}{0 0 0}
\pscustom[linestyle=none,fillstyle=solid,fillcolor=curcolor]
{
\newpath
\moveto(151.69605695,496.62)
\lineto(148.04805695,496.62)
\lineto(148.04805695,495.636)
\lineto(150.71205695,495.636)
\lineto(150.71205695,495.396)
\curveto(150.71205695,493.836)(149.56005695,492.708)(147.96405695,492.708)
\curveto(147.07605695,492.708)(146.27205695,493.032)(145.75605695,493.596)
\curveto(145.18005695,494.22)(144.83205695,495.264)(144.83205695,496.344)
\curveto(144.83205695,498.492)(146.05605695,499.908)(147.90405695,499.908)
\curveto(149.23605695,499.908)(150.19605695,499.224)(150.43605695,498.096)
\lineto(151.57605695,498.096)
\curveto(151.26405695,499.872)(149.92005695,500.892)(147.91605695,500.892)
\curveto(146.84805695,500.892)(145.98405695,500.616)(145.30005695,500.052)
\curveto(144.28005695,499.212)(143.71605695,497.856)(143.71605695,496.284)
\curveto(143.71605695,493.596)(145.36005695,491.724)(147.72405695,491.724)
\curveto(148.91205695,491.724)(149.84805695,492.168)(150.71205695,493.116)
\lineto(150.98805695,491.952)
\lineto(151.69605695,491.952)
\closepath
}
}
{
\newrgbcolor{curcolor}{0 0 0}
\pscustom[linestyle=none,fillstyle=solid,fillcolor=curcolor]
{
\newpath
\moveto(154.72005548,495.984)
\lineto(159.48405548,495.984)
\lineto(159.48405548,496.968)
\lineto(154.72005548,496.968)
\lineto(154.72005548,499.764)
\lineto(159.66405548,499.764)
\lineto(159.66405548,500.748)
\lineto(153.60405548,500.748)
\lineto(153.60405548,492)
\lineto(159.88005548,492)
\lineto(159.88005548,492.984)
\lineto(154.72005548,492.984)
\closepath
}
}
{
\newrgbcolor{curcolor}{0 0 0}
\pscustom[linestyle=none,fillstyle=solid,fillcolor=curcolor]
{
\newpath
\moveto(161.47605328,492)
\lineto(165.42405328,492)
\curveto(166.25205328,492)(166.86405328,492.228)(167.33205328,492.732)
\curveto(167.76405328,493.188)(168.00405328,493.812)(168.00405328,494.496)
\curveto(168.00405328,495.552)(167.52405328,496.188)(166.40805328,496.62)
\curveto(167.21205328,496.992)(167.62005328,497.628)(167.62005328,498.528)
\curveto(167.62005328,499.176)(167.38005328,499.728)(166.92405328,500.136)
\curveto(166.45605328,500.556)(165.86805328,500.748)(165.02805328,500.748)
\lineto(161.47605328,500.748)
\closepath
\moveto(162.59205328,496.98)
\lineto(162.59205328,499.764)
\lineto(164.75205328,499.764)
\curveto(165.37605328,499.764)(165.72405328,499.68)(166.02405328,499.452)
\curveto(166.33605328,499.212)(166.50405328,498.852)(166.50405328,498.372)
\curveto(166.50405328,497.892)(166.33605328,497.532)(166.02405328,497.292)
\curveto(165.72405328,497.064)(165.37605328,496.98)(164.75205328,496.98)
\closepath
\moveto(162.59205328,492.984)
\lineto(162.59205328,495.996)
\lineto(165.31605328,495.996)
\curveto(166.30005328,495.996)(166.88805328,495.432)(166.88805328,494.484)
\curveto(166.88805328,493.548)(166.30005328,492.984)(165.31605328,492.984)
\closepath
}
}
{
\newrgbcolor{curcolor}{0 0 0}
\pscustom[linestyle=none,fillstyle=solid,fillcolor=curcolor]
{
\newpath
\moveto(176.36804175,498.036)
\curveto(176.02004175,499.956)(174.91604175,500.892)(172.99604175,500.892)
\curveto(171.82004175,500.892)(170.87204175,500.52)(170.22404175,499.8)
\curveto(169.43204175,498.936)(169.00004175,497.688)(169.00004175,496.272)
\curveto(169.00004175,494.832)(169.44404175,493.596)(170.26004175,492.744)
\curveto(170.94404175,492.048)(171.80804175,491.724)(172.94804175,491.724)
\curveto(175.08404175,491.724)(176.28404175,492.876)(176.54804175,495.192)
\lineto(175.39604175,495.192)
\curveto(175.30004175,494.592)(175.18004175,494.184)(175.00004175,493.836)
\curveto(174.64004175,493.116)(173.89604175,492.708)(172.96004175,492.708)
\curveto(171.22004175,492.708)(170.11604175,494.1)(170.11604175,496.284)
\curveto(170.11604175,498.528)(171.16004175,499.908)(172.86404175,499.908)
\curveto(173.57204175,499.908)(174.23204175,499.704)(174.59204175,499.356)
\curveto(174.91604175,499.056)(175.09604175,498.696)(175.22804175,498.036)
\closepath
}
}
{
\newrgbcolor{curcolor}{0 0 0}
\pscustom[linestyle=none,fillstyle=solid,fillcolor=curcolor]
{
\newpath
\moveto(181.66000439,500.892)
\curveto(179.15200439,500.892)(177.44800439,499.044)(177.44800439,496.308)
\curveto(177.44800439,493.56)(179.14000439,491.724)(181.67200439,491.724)
\curveto(182.74000439,491.724)(183.67600439,492.048)(184.38400439,492.648)
\curveto(185.33200439,493.452)(185.89600439,494.808)(185.89600439,496.236)
\curveto(185.89600439,499.056)(184.22800439,500.892)(181.66000439,500.892)
\closepath
\moveto(181.66000439,499.908)
\curveto(183.55600439,499.908)(184.78000439,498.48)(184.78000439,496.26)
\curveto(184.78000439,494.148)(183.52000439,492.708)(181.67200439,492.708)
\curveto(179.80000439,492.708)(178.56400439,494.148)(178.56400439,496.308)
\curveto(178.56400439,498.468)(179.80000439,499.908)(181.66000439,499.908)
\closepath
}
}
{
\newrgbcolor{curcolor}{0 0 0}
\pscustom[linestyle=none,fillstyle=solid,fillcolor=curcolor]
{
\newpath
\moveto(195.60400146,500.748)
\lineto(194.60800146,500.748)
\lineto(194.60800146,497.496)
\curveto(194.18800146,498.132)(193.51600146,498.468)(192.67600146,498.468)
\curveto(191.04400146,498.468)(189.97600146,497.16)(189.97600146,495.156)
\curveto(189.97600146,493.032)(191.00800146,491.724)(192.71200146,491.724)
\curveto(193.57600146,491.724)(194.17600146,492.048)(194.71600146,492.828)
\lineto(194.71600146,492)
\lineto(195.60400146,492)
\closepath
\moveto(192.84400146,497.532)
\curveto(193.92400146,497.532)(194.60800146,496.584)(194.60800146,495.072)
\curveto(194.60800146,493.62)(193.91200146,492.66)(192.85600146,492.66)
\curveto(191.75200146,492.66)(191.02000146,493.632)(191.02000146,495.096)
\curveto(191.02000146,496.56)(191.75200146,497.532)(192.84400146,497.532)
\closepath
}
}
{
\newrgbcolor{curcolor}{0 0 0}
\pscustom[linestyle=none,fillstyle=solid,fillcolor=curcolor]
{
\newpath
\moveto(202.70797913,492.588)
\curveto(202.59997913,492.564)(202.55197913,492.564)(202.49197913,492.564)
\curveto(202.14397913,492.564)(201.95197913,492.744)(201.95197913,493.056)
\lineto(201.95197913,496.752)
\curveto(201.95197913,497.868)(201.13597913,498.468)(199.58797913,498.468)
\curveto(198.66397913,498.468)(197.93197913,498.204)(197.49997913,497.736)
\curveto(197.21197913,497.412)(197.09197913,497.052)(197.06797913,496.428)
\lineto(198.07597913,496.428)
\curveto(198.15997913,497.196)(198.61597913,497.544)(199.55197913,497.544)
\curveto(200.46397913,497.544)(200.95597913,497.208)(200.95597913,496.608)
\lineto(200.95597913,496.344)
\curveto(200.94397913,495.912)(200.72797913,495.756)(199.91197913,495.648)
\curveto(198.49597913,495.468)(198.27997913,495.42)(197.89597913,495.264)
\curveto(197.16397913,494.952)(196.79197913,494.4)(196.79197913,493.584)
\curveto(196.79197913,492.444)(197.58397913,491.724)(198.85597913,491.724)
\curveto(199.64797913,491.724)(200.28397913,492)(200.99197913,492.648)
\curveto(201.06397913,492)(201.37597913,491.724)(202.02397913,491.724)
\curveto(202.23997913,491.724)(202.37197913,491.748)(202.70797913,491.832)
\closepath
\moveto(200.95597913,493.98)
\curveto(200.95597913,493.644)(200.85997913,493.44)(200.55997913,493.164)
\curveto(200.15197913,492.792)(199.65997913,492.6)(199.07197913,492.6)
\curveto(198.29197913,492.6)(197.83597913,492.972)(197.83597913,493.608)
\curveto(197.83597913,494.268)(198.26797913,494.604)(199.34797913,494.76)
\curveto(200.41597913,494.904)(200.61997913,494.952)(200.95597913,495.108)
\closepath
}
}
{
\newrgbcolor{curcolor}{0 0 0}
\pscustom[linestyle=none,fillstyle=solid,fillcolor=curcolor]
{
\newpath
\moveto(205.87595715,498.288)
\lineto(204.84395715,498.288)
\lineto(204.84395715,500.016)
\lineto(203.84795715,500.016)
\lineto(203.84795715,498.288)
\lineto(202.99595715,498.288)
\lineto(202.99595715,497.472)
\lineto(203.84795715,497.472)
\lineto(203.84795715,492.72)
\curveto(203.84795715,492.072)(204.27995715,491.724)(205.05995715,491.724)
\curveto(205.32395715,491.724)(205.53995715,491.748)(205.87595715,491.808)
\lineto(205.87595715,492.648)
\curveto(205.73195715,492.612)(205.59995715,492.6)(205.39595715,492.6)
\curveto(204.96395715,492.6)(204.84395715,492.72)(204.84395715,493.164)
\lineto(204.84395715,497.472)
\lineto(205.87595715,497.472)
\closepath
}
}
{
\newrgbcolor{curcolor}{0 0 0}
\pscustom[linestyle=none,fillstyle=solid,fillcolor=curcolor]
{
\newpath
\moveto(212.55994598,492.588)
\curveto(212.45194598,492.564)(212.40394598,492.564)(212.34394598,492.564)
\curveto(211.99594598,492.564)(211.80394598,492.744)(211.80394598,493.056)
\lineto(211.80394598,496.752)
\curveto(211.80394598,497.868)(210.98794598,498.468)(209.43994598,498.468)
\curveto(208.51594598,498.468)(207.78394598,498.204)(207.35194598,497.736)
\curveto(207.06394598,497.412)(206.94394598,497.052)(206.91994598,496.428)
\lineto(207.92794598,496.428)
\curveto(208.01194598,497.196)(208.46794598,497.544)(209.40394598,497.544)
\curveto(210.31594598,497.544)(210.80794598,497.208)(210.80794598,496.608)
\lineto(210.80794598,496.344)
\curveto(210.79594598,495.912)(210.57994598,495.756)(209.76394598,495.648)
\curveto(208.34794598,495.468)(208.13194598,495.42)(207.74794598,495.264)
\curveto(207.01594598,494.952)(206.64394598,494.4)(206.64394598,493.584)
\curveto(206.64394598,492.444)(207.43594598,491.724)(208.70794598,491.724)
\curveto(209.49994598,491.724)(210.13594598,492)(210.84394598,492.648)
\curveto(210.91594598,492)(211.22794598,491.724)(211.87594598,491.724)
\curveto(212.09194598,491.724)(212.22394598,491.748)(212.55994598,491.832)
\closepath
\moveto(210.80794598,493.98)
\curveto(210.80794598,493.644)(210.71194598,493.44)(210.41194598,493.164)
\curveto(210.00394598,492.792)(209.51194598,492.6)(208.92394598,492.6)
\curveto(208.14394598,492.6)(207.68794598,492.972)(207.68794598,493.608)
\curveto(207.68794598,494.268)(208.11994598,494.604)(209.19994598,494.76)
\curveto(210.26794598,494.904)(210.47194598,494.952)(210.80794598,495.108)
\closepath
}
}
{
\newrgbcolor{curcolor}{0 0 0}
\pscustom[linewidth=1,linecolor=curcolor]
{
\newpath
\moveto(170.5,340.5)
\lineto(190.53,340.5)
\lineto(180.53,340.5)
\lineto(194.13,340.5)
}
}
{
\newrgbcolor{curcolor}{0 0 0}
\pscustom[linestyle=none,fillstyle=solid,fillcolor=curcolor]
{
\newpath
\moveto(199.38,340.5)
\lineto(192.38,337)
\lineto(194.13,340.5)
\lineto(192.38,344)
\closepath
}
}
{
\newrgbcolor{curcolor}{0 0 0}
\pscustom[linewidth=1,linecolor=curcolor]
{
\newpath
\moveto(199.38,340.5)
\lineto(192.38,337)
\lineto(194.13,340.5)
\lineto(192.38,344)
\closepath
}
}
{
\newrgbcolor{curcolor}{1 1 1}
\pscustom[linestyle=none,fillstyle=solid,fillcolor=curcolor]
{
\newpath
\moveto(50.5,370.5)
\lineto(170.5,370.5)
\lineto(170.5,310.5)
\lineto(50.5,310.5)
\closepath
}
}
{
\newrgbcolor{curcolor}{0 0 0}
\pscustom[linewidth=1,linecolor=curcolor]
{
\newpath
\moveto(50.5,370.5)
\lineto(170.5,370.5)
\lineto(170.5,310.5)
\lineto(50.5,310.5)
\closepath
}
}
{
\newrgbcolor{curcolor}{0 0 0}
\pscustom[linestyle=none,fillstyle=solid,fillcolor=curcolor]
{
\newpath
\moveto(49.35812534,340.768)
\lineto(52.23812534,340.768)
\curveto(53.23412534,340.768)(53.67812534,340.288)(53.67812534,339.208)
\lineto(53.66612534,338.428)
\curveto(53.66612534,337.888)(53.76212534,337.36)(53.91812534,337)
\lineto(55.27412534,337)
\lineto(55.27412534,337.276)
\curveto(54.85412534,337.564)(54.77012534,337.876)(54.74612534,339.04)
\curveto(54.73412534,340.48)(54.50612534,340.912)(53.55812534,341.32)
\curveto(54.54212534,341.812)(54.93812534,342.4)(54.93812534,343.408)
\curveto(54.93812534,344.92)(54.00212534,345.748)(52.27412534,345.748)
\lineto(48.24212534,345.748)
\lineto(48.24212534,337)
\lineto(49.35812534,337)
\closepath
\moveto(49.35812534,341.752)
\lineto(49.35812534,344.764)
\lineto(52.05812534,344.764)
\curveto(52.68212534,344.764)(53.04212534,344.668)(53.31812534,344.428)
\curveto(53.61812534,344.176)(53.77412534,343.78)(53.77412534,343.264)
\curveto(53.77412534,342.22)(53.24612534,341.752)(52.05812534,341.752)
\closepath
}
}
{
\newrgbcolor{curcolor}{0 0 0}
\pscustom[linestyle=none,fillstyle=solid,fillcolor=curcolor]
{
\newpath
\moveto(61.82610117,339.808)
\curveto(61.82610117,340.768)(61.75410117,341.344)(61.57410117,341.812)
\curveto(61.16610117,342.844)(60.20610117,343.468)(59.03010117,343.468)
\curveto(57.27810117,343.468)(56.15010117,342.136)(56.15010117,340.06)
\curveto(56.15010117,337.984)(57.23010117,336.724)(59.00610117,336.724)
\curveto(60.44610117,336.724)(61.44210117,337.54)(61.69410117,338.908)
\lineto(60.68610117,338.908)
\curveto(60.41010117,338.08)(59.84610117,337.648)(59.04210117,337.648)
\curveto(58.40610117,337.648)(57.86610117,337.936)(57.53010117,338.464)
\curveto(57.29010117,338.824)(57.20610117,339.184)(57.19410117,339.808)
\closepath
\moveto(57.21810117,340.624)
\curveto(57.30210117,341.788)(58.01010117,342.544)(59.01810117,342.544)
\curveto(60.03810117,342.544)(60.74610117,341.752)(60.74610117,340.624)
\closepath
}
}
{
\newrgbcolor{curcolor}{0 0 0}
\pscustom[linestyle=none,fillstyle=solid,fillcolor=curcolor]
{
\newpath
\moveto(63.18209824,343.288)
\lineto(63.18209824,337)
\lineto(64.19009824,337)
\lineto(64.19009824,340.948)
\curveto(64.19009824,341.86)(64.85009824,342.592)(65.66609824,342.592)
\curveto(66.41009824,342.592)(66.83009824,342.136)(66.83009824,341.332)
\lineto(66.83009824,337)
\lineto(67.83809824,337)
\lineto(67.83809824,340.948)
\curveto(67.83809824,341.86)(68.49809824,342.592)(69.31409824,342.592)
\curveto(70.04609824,342.592)(70.47809824,342.124)(70.47809824,341.332)
\lineto(70.47809824,337)
\lineto(71.48609824,337)
\lineto(71.48609824,341.716)
\curveto(71.48609824,342.844)(70.83809824,343.468)(69.66209824,343.468)
\curveto(68.82209824,343.468)(68.31809824,343.216)(67.73009824,342.508)
\curveto(67.35809824,343.18)(66.85409824,343.468)(66.03809824,343.468)
\curveto(65.19809824,343.468)(64.63409824,343.156)(64.10609824,342.4)
\lineto(64.10609824,343.288)
\closepath
}
}
{
\newrgbcolor{curcolor}{0 0 0}
\pscustom[linestyle=none,fillstyle=solid,fillcolor=curcolor]
{
\newpath
\moveto(75.60207755,343.468)
\curveto(73.82607755,343.468)(72.77007755,342.208)(72.77007755,340.096)
\curveto(72.77007755,337.972)(73.82607755,336.724)(75.61407755,336.724)
\curveto(77.39007755,336.724)(78.45807755,337.984)(78.45807755,340.048)
\curveto(78.45807755,342.232)(77.42607755,343.468)(75.60207755,343.468)
\closepath
\moveto(75.61407755,342.544)
\curveto(76.74207755,342.544)(77.41407755,341.62)(77.41407755,340.06)
\curveto(77.41407755,338.572)(76.71807755,337.648)(75.61407755,337.648)
\curveto(74.49807755,337.648)(73.81407755,338.572)(73.81407755,340.096)
\curveto(73.81407755,341.62)(74.49807755,342.544)(75.61407755,342.544)
\closepath
}
}
{
\newrgbcolor{curcolor}{0 0 0}
\pscustom[linestyle=none,fillstyle=solid,fillcolor=curcolor]
{
\newpath
\moveto(82.21405594,337)
\lineto(84.62605594,343.288)
\lineto(83.49805594,343.288)
\lineto(81.72205594,338.188)
\lineto(80.04205594,343.288)
\lineto(78.91405594,343.288)
\lineto(81.12205594,337)
\closepath
}
}
{
\newrgbcolor{curcolor}{0 0 0}
\pscustom[linestyle=none,fillstyle=solid,fillcolor=curcolor]
{
\newpath
\moveto(90.71002756,339.808)
\curveto(90.71002756,340.768)(90.63802756,341.344)(90.45802756,341.812)
\curveto(90.05002756,342.844)(89.09002756,343.468)(87.91402756,343.468)
\curveto(86.16202756,343.468)(85.03402756,342.136)(85.03402756,340.06)
\curveto(85.03402756,337.984)(86.11402756,336.724)(87.89002756,336.724)
\curveto(89.33002756,336.724)(90.32602756,337.54)(90.57802756,338.908)
\lineto(89.57002756,338.908)
\curveto(89.29402756,338.08)(88.73002756,337.648)(87.92602756,337.648)
\curveto(87.29002756,337.648)(86.75002756,337.936)(86.41402756,338.464)
\curveto(86.17402756,338.824)(86.09002756,339.184)(86.07802756,339.808)
\closepath
\moveto(86.10202756,340.624)
\curveto(86.18602756,341.788)(86.89402756,342.544)(87.90202756,342.544)
\curveto(88.92202756,342.544)(89.63002756,341.752)(89.63002756,340.624)
\closepath
}
}
{
\newrgbcolor{curcolor}{0 0 0}
\pscustom[linestyle=none,fillstyle=solid,fillcolor=curcolor]
{
\newpath
\moveto(95.21002316,334.384)
\lineto(96.21802316,334.384)
\lineto(96.21802316,337.66)
\curveto(96.74602316,337.012)(97.33402316,336.724)(98.15002316,336.724)
\curveto(99.78202316,336.724)(100.83802316,338.032)(100.83802316,340.036)
\curveto(100.83802316,342.148)(99.80602316,343.468)(98.13802316,343.468)
\curveto(97.28602316,343.468)(96.60202316,343.084)(96.13402316,342.34)
\lineto(96.13402316,343.288)
\lineto(95.21002316,343.288)
\closepath
\moveto(97.97002316,342.532)
\curveto(99.07402316,342.532)(99.79402316,341.56)(99.79402316,340.06)
\curveto(99.79402316,338.632)(99.06202316,337.66)(97.97002316,337.66)
\curveto(96.90202316,337.66)(96.21802316,338.62)(96.21802316,340.096)
\curveto(96.21802316,341.572)(96.90202316,342.532)(97.97002316,342.532)
\closepath
}
}
{
\newrgbcolor{curcolor}{0 0 0}
\pscustom[linestyle=none,fillstyle=solid,fillcolor=curcolor]
{
\newpath
\moveto(104.49802023,343.468)
\curveto(102.72202023,343.468)(101.66602023,342.208)(101.66602023,340.096)
\curveto(101.66602023,337.972)(102.72202023,336.724)(104.51002023,336.724)
\curveto(106.28602023,336.724)(107.35402023,337.984)(107.35402023,340.048)
\curveto(107.35402023,342.232)(106.32202023,343.468)(104.49802023,343.468)
\closepath
\moveto(104.51002023,342.544)
\curveto(105.63802023,342.544)(106.31002023,341.62)(106.31002023,340.06)
\curveto(106.31002023,338.572)(105.61402023,337.648)(104.51002023,337.648)
\curveto(103.39402023,337.648)(102.71002023,338.572)(102.71002023,340.096)
\curveto(102.71002023,341.62)(103.39402023,342.544)(104.51002023,342.544)
\closepath
}
}
{
\newrgbcolor{curcolor}{0 0 0}
\pscustom[linestyle=none,fillstyle=solid,fillcolor=curcolor]
{
\newpath
\moveto(109.7060173,343.288)
\lineto(108.7100173,343.288)
\lineto(108.7100173,337)
\lineto(109.7060173,337)
\closepath
\moveto(109.7060173,345.748)
\lineto(108.6980173,345.748)
\lineto(108.6980173,344.488)
\lineto(109.7060173,344.488)
\closepath
}
}
{
\newrgbcolor{curcolor}{0 0 0}
\pscustom[linestyle=none,fillstyle=solid,fillcolor=curcolor]
{
\newpath
\moveto(111.40999588,343.288)
\lineto(111.40999588,337)
\lineto(112.41799588,337)
\lineto(112.41799588,340.468)
\curveto(112.41799588,341.752)(113.08999588,342.592)(114.12199588,342.592)
\curveto(114.91399588,342.592)(115.41799588,342.112)(115.41799588,341.356)
\lineto(115.41799588,337)
\lineto(116.41399588,337)
\lineto(116.41399588,341.752)
\curveto(116.41399588,342.796)(115.63399588,343.468)(114.42199588,343.468)
\curveto(113.48599588,343.468)(112.88599588,343.108)(112.33399588,342.232)
\lineto(112.33399588,343.288)
\closepath
}
}
{
\newrgbcolor{curcolor}{0 0 0}
\pscustom[linestyle=none,fillstyle=solid,fillcolor=curcolor]
{
\newpath
\moveto(120.28999295,343.288)
\lineto(119.25799295,343.288)
\lineto(119.25799295,345.016)
\lineto(118.26199295,345.016)
\lineto(118.26199295,343.288)
\lineto(117.40999295,343.288)
\lineto(117.40999295,342.472)
\lineto(118.26199295,342.472)
\lineto(118.26199295,337.72)
\curveto(118.26199295,337.072)(118.69399295,336.724)(119.47399295,336.724)
\curveto(119.73799295,336.724)(119.95399295,336.748)(120.28999295,336.808)
\lineto(120.28999295,337.648)
\curveto(120.14599295,337.612)(120.01399295,337.6)(119.80999295,337.6)
\curveto(119.37799295,337.6)(119.25799295,337.72)(119.25799295,338.164)
\lineto(119.25799295,342.472)
\lineto(120.28999295,342.472)
\closepath
}
}
{
\newrgbcolor{curcolor}{0 0 0}
\pscustom[linestyle=none,fillstyle=solid,fillcolor=curcolor]
{
\newpath
\moveto(125.83399149,341.536)
\curveto(125.82199149,342.772)(125.00599149,343.468)(123.55399149,343.468)
\curveto(122.08999149,343.468)(121.14199149,342.712)(121.14199149,341.548)
\curveto(121.14199149,340.564)(121.64599149,340.096)(123.13399149,339.736)
\lineto(124.06999149,339.508)
\curveto(124.76599149,339.34)(125.04199149,339.088)(125.04199149,338.644)
\curveto(125.04199149,338.044)(124.45399149,337.648)(123.57799149,337.648)
\curveto(123.03799149,337.648)(122.58199149,337.804)(122.32999149,338.068)
\curveto(122.17399149,338.248)(122.10199149,338.428)(122.04199149,338.872)
\lineto(120.98599149,338.872)
\curveto(121.03399149,337.42)(121.84999149,336.724)(123.49399149,336.724)
\curveto(125.07799149,336.724)(126.08599149,337.504)(126.08599149,338.716)
\curveto(126.08599149,339.652)(125.55799149,340.168)(124.30999149,340.468)
\lineto(123.34999149,340.696)
\curveto(122.53399149,340.888)(122.18599149,341.152)(122.18599149,341.596)
\curveto(122.18599149,342.184)(122.70199149,342.544)(123.51799149,342.544)
\curveto(124.32199149,342.544)(124.75399149,342.196)(124.77799149,341.536)
\closepath
}
}
{
\newrgbcolor{curcolor}{0 0 0}
\pscustom[linestyle=none,fillstyle=solid,fillcolor=curcolor]
{
\newpath
\moveto(133.00996713,343.288)
\lineto(131.96596713,343.288)
\lineto(131.96596713,344.272)
\curveto(131.96596713,344.692)(132.19396713,344.908)(132.66196713,344.908)
\curveto(132.74596713,344.908)(132.78196713,344.908)(133.00996713,344.896)
\lineto(133.00996713,345.724)
\curveto(132.78196713,345.772)(132.64996713,345.784)(132.44596713,345.784)
\curveto(131.52196713,345.784)(130.96996713,345.256)(130.96996713,344.356)
\lineto(130.96996713,343.288)
\lineto(130.12996713,343.288)
\lineto(130.12996713,342.472)
\lineto(130.96996713,342.472)
\lineto(130.96996713,337)
\lineto(131.96596713,337)
\lineto(131.96596713,342.472)
\lineto(133.00996713,342.472)
\closepath
}
}
{
\newrgbcolor{curcolor}{0 0 0}
\pscustom[linestyle=none,fillstyle=solid,fillcolor=curcolor]
{
\newpath
\moveto(134.07796567,343.288)
\lineto(134.07796567,337)
\lineto(135.08596567,337)
\lineto(135.08596567,340.264)
\curveto(135.09796567,341.776)(135.72196567,342.448)(137.10196567,342.412)
\lineto(137.10196567,343.432)
\curveto(136.93396567,343.456)(136.83796567,343.468)(136.71796567,343.468)
\curveto(136.06996567,343.468)(135.57796567,343.084)(135.00196567,342.148)
\lineto(135.00196567,343.288)
\closepath
}
}
{
\newrgbcolor{curcolor}{0 0 0}
\pscustom[linestyle=none,fillstyle=solid,fillcolor=curcolor]
{
\newpath
\moveto(140.43793875,343.468)
\curveto(138.66193875,343.468)(137.60593875,342.208)(137.60593875,340.096)
\curveto(137.60593875,337.972)(138.66193875,336.724)(140.44993875,336.724)
\curveto(142.22593875,336.724)(143.29393875,337.984)(143.29393875,340.048)
\curveto(143.29393875,342.232)(142.26193875,343.468)(140.43793875,343.468)
\closepath
\moveto(140.44993875,342.544)
\curveto(141.57793875,342.544)(142.24993875,341.62)(142.24993875,340.06)
\curveto(142.24993875,338.572)(141.55393875,337.648)(140.44993875,337.648)
\curveto(139.33393875,337.648)(138.64993875,338.572)(138.64993875,340.096)
\curveto(138.64993875,341.62)(139.33393875,342.544)(140.44993875,342.544)
\closepath
}
}
{
\newrgbcolor{curcolor}{0 0 0}
\pscustom[linestyle=none,fillstyle=solid,fillcolor=curcolor]
{
\newpath
\moveto(144.68593582,343.288)
\lineto(144.68593582,337)
\lineto(145.69393582,337)
\lineto(145.69393582,340.948)
\curveto(145.69393582,341.86)(146.35393582,342.592)(147.16993582,342.592)
\curveto(147.91393582,342.592)(148.33393582,342.136)(148.33393582,341.332)
\lineto(148.33393582,337)
\lineto(149.34193582,337)
\lineto(149.34193582,340.948)
\curveto(149.34193582,341.86)(150.00193582,342.592)(150.81793582,342.592)
\curveto(151.54993582,342.592)(151.98193582,342.124)(151.98193582,341.332)
\lineto(151.98193582,337)
\lineto(152.98993582,337)
\lineto(152.98993582,341.716)
\curveto(152.98993582,342.844)(152.34193582,343.468)(151.16593582,343.468)
\curveto(150.32593582,343.468)(149.82193582,343.216)(149.23393582,342.508)
\curveto(148.86193582,343.18)(148.35793582,343.468)(147.54193582,343.468)
\curveto(146.70193582,343.468)(146.13793582,343.156)(145.60993582,342.4)
\lineto(145.60993582,343.288)
\closepath
}
}
{
\newrgbcolor{curcolor}{0 0 0}
\pscustom[linestyle=none,fillstyle=solid,fillcolor=curcolor]
{
\newpath
\moveto(162.49391367,343.288)
\lineto(157.80191367,343.288)
\lineto(157.80191367,342.412)
\lineto(161.30591367,342.412)
\lineto(157.54991367,337.9)
\lineto(157.54991367,337)
\lineto(162.66191367,337)
\lineto(162.66191367,337.876)
\lineto(158.76191367,337.876)
\lineto(162.49391367,342.4)
\closepath
}
}
{
\newrgbcolor{curcolor}{0 0 0}
\pscustom[linestyle=none,fillstyle=solid,fillcolor=curcolor]
{
\newpath
\moveto(165.15787906,338.248)
\lineto(163.90987906,338.248)
\lineto(163.90987906,337)
\lineto(165.15787906,337)
\closepath
}
}
{
\newrgbcolor{curcolor}{0 0 0}
\pscustom[linestyle=none,fillstyle=solid,fillcolor=curcolor]
{
\newpath
\moveto(168.49387759,338.248)
\lineto(167.24587759,338.248)
\lineto(167.24587759,337)
\lineto(168.49387759,337)
\closepath
}
}
{
\newrgbcolor{curcolor}{0 0 0}
\pscustom[linestyle=none,fillstyle=solid,fillcolor=curcolor]
{
\newpath
\moveto(171.82987613,338.248)
\lineto(170.58187613,338.248)
\lineto(170.58187613,337)
\lineto(171.82987613,337)
\closepath
}
}
{
\newrgbcolor{curcolor}{0 0 0}
\pscustom[linewidth=1,linecolor=curcolor]
{
\newpath
\moveto(90.53,210.5)
\lineto(90.53,190.5)
\lineto(90.53,200.5)
\lineto(90.53,186.87)
}
}
{
\newrgbcolor{curcolor}{0 0 0}
\pscustom[linestyle=none,fillstyle=solid,fillcolor=curcolor]
{
\newpath
\moveto(90.53,181.62)
\lineto(87.03,188.62)
\lineto(90.53,186.87)
\lineto(94.03,188.62)
\closepath
}
}
{
\newrgbcolor{curcolor}{0 0 0}
\pscustom[linewidth=1,linecolor=curcolor]
{
\newpath
\moveto(90.53,181.62)
\lineto(87.03,188.62)
\lineto(90.53,186.87)
\lineto(94.03,188.62)
\closepath
}
}
{
\newrgbcolor{curcolor}{1 1 1}
\pscustom[linestyle=none,fillstyle=solid,fillcolor=curcolor]
{
\newpath
\moveto(10.5,290.5)
\lineto(170.5,290.5)
\lineto(170.5,210.5)
\lineto(10.5,210.5)
\closepath
}
}
{
\newrgbcolor{curcolor}{0 0 0}
\pscustom[linewidth=1,linecolor=curcolor]
{
\newpath
\moveto(10.5,290.5)
\lineto(170.5,290.5)
\lineto(170.5,210.5)
\lineto(10.5,210.5)
\closepath
}
}
{
\newrgbcolor{curcolor}{0 0 0}
\pscustom[linestyle=none,fillstyle=solid,fillcolor=curcolor]
{
\newpath
\moveto(20.41815271,250.984)
\lineto(24.59415271,250.984)
\lineto(24.59415271,251.968)
\lineto(20.41815271,251.968)
\lineto(20.41815271,254.764)
\lineto(25.17015271,254.764)
\lineto(25.17015271,255.748)
\lineto(19.30215271,255.748)
\lineto(19.30215271,247)
\lineto(20.41815271,247)
\closepath
}
}
{
\newrgbcolor{curcolor}{0 0 0}
\pscustom[linestyle=none,fillstyle=solid,fillcolor=curcolor]
{
\newpath
\moveto(27.23412781,253.288)
\lineto(26.23812781,253.288)
\lineto(26.23812781,247)
\lineto(27.23412781,247)
\closepath
\moveto(27.23412781,255.748)
\lineto(26.22612781,255.748)
\lineto(26.22612781,254.488)
\lineto(27.23412781,254.488)
\closepath
}
}
{
\newrgbcolor{curcolor}{0 0 0}
\pscustom[linestyle=none,fillstyle=solid,fillcolor=curcolor]
{
\newpath
\moveto(28.93810638,253.288)
\lineto(28.93810638,247)
\lineto(29.94610638,247)
\lineto(29.94610638,250.468)
\curveto(29.94610638,251.752)(30.61810638,252.592)(31.65010638,252.592)
\curveto(32.44210638,252.592)(32.94610638,252.112)(32.94610638,251.356)
\lineto(32.94610638,247)
\lineto(33.94210638,247)
\lineto(33.94210638,251.752)
\curveto(33.94210638,252.796)(33.16210638,253.468)(31.95010638,253.468)
\curveto(31.01410638,253.468)(30.41410638,253.108)(29.86210638,252.232)
\lineto(29.86210638,253.288)
\closepath
}
}
{
\newrgbcolor{curcolor}{0 0 0}
\pscustom[linestyle=none,fillstyle=solid,fillcolor=curcolor]
{
\newpath
\moveto(40.71010345,255.748)
\lineto(39.71410345,255.748)
\lineto(39.71410345,252.496)
\curveto(39.29410345,253.132)(38.62210345,253.468)(37.78210345,253.468)
\curveto(36.15010345,253.468)(35.08210345,252.16)(35.08210345,250.156)
\curveto(35.08210345,248.032)(36.11410345,246.724)(37.81810345,246.724)
\curveto(38.68210345,246.724)(39.28210345,247.048)(39.82210345,247.828)
\lineto(39.82210345,247)
\lineto(40.71010345,247)
\closepath
\moveto(37.95010345,252.532)
\curveto(39.03010345,252.532)(39.71410345,251.584)(39.71410345,250.072)
\curveto(39.71410345,248.62)(39.01810345,247.66)(37.96210345,247.66)
\curveto(36.85810345,247.66)(36.12610345,248.632)(36.12610345,250.096)
\curveto(36.12610345,251.56)(36.85810345,252.532)(37.95010345,252.532)
\closepath
}
}
{
\newrgbcolor{curcolor}{0 0 0}
\pscustom[linestyle=none,fillstyle=solid,fillcolor=curcolor]
{
\newpath
\moveto(46.60209906,255.748)
\lineto(45.59409906,255.748)
\lineto(45.59409906,247)
\lineto(46.60209906,247)
\closepath
}
}
{
\newrgbcolor{curcolor}{0 0 0}
\pscustom[linestyle=none,fillstyle=solid,fillcolor=curcolor]
{
\newpath
\moveto(50.70607764,253.468)
\curveto(48.93007764,253.468)(47.87407764,252.208)(47.87407764,250.096)
\curveto(47.87407764,247.972)(48.93007764,246.724)(50.71807764,246.724)
\curveto(52.49407764,246.724)(53.56207764,247.984)(53.56207764,250.048)
\curveto(53.56207764,252.232)(52.53007764,253.468)(50.70607764,253.468)
\closepath
\moveto(50.71807764,252.544)
\curveto(51.84607764,252.544)(52.51807764,251.62)(52.51807764,250.06)
\curveto(52.51807764,248.572)(51.82207764,247.648)(50.71807764,247.648)
\curveto(49.60207764,247.648)(48.91807764,248.572)(48.91807764,250.096)
\curveto(48.91807764,251.62)(49.60207764,252.544)(50.71807764,252.544)
\closepath
}
}
{
\newrgbcolor{curcolor}{0 0 0}
\pscustom[linestyle=none,fillstyle=solid,fillcolor=curcolor]
{
\newpath
\moveto(59.76607471,251.176)
\curveto(59.71807471,251.788)(59.58607471,252.184)(59.34607471,252.532)
\curveto(58.91407471,253.12)(58.15807471,253.468)(57.28207471,253.468)
\curveto(55.57807471,253.468)(54.48607471,252.124)(54.48607471,250.036)
\curveto(54.48607471,248.008)(55.56607471,246.724)(57.27007471,246.724)
\curveto(58.77007471,246.724)(59.71807471,247.624)(59.83807471,249.16)
\lineto(58.83007471,249.16)
\curveto(58.66207471,248.152)(58.14607471,247.648)(57.29407471,247.648)
\curveto(56.19007471,247.648)(55.53007471,248.548)(55.53007471,250.036)
\curveto(55.53007471,251.608)(56.17807471,252.544)(57.27007471,252.544)
\curveto(58.11007471,252.544)(58.63807471,252.052)(58.75807471,251.176)
\closepath
}
}
{
\newrgbcolor{curcolor}{0 0 0}
\pscustom[linestyle=none,fillstyle=solid,fillcolor=curcolor]
{
\newpath
\moveto(66.61805145,247.588)
\curveto(66.51005145,247.564)(66.46205145,247.564)(66.40205145,247.564)
\curveto(66.05405145,247.564)(65.86205145,247.744)(65.86205145,248.056)
\lineto(65.86205145,251.752)
\curveto(65.86205145,252.868)(65.04605145,253.468)(63.49805145,253.468)
\curveto(62.57405145,253.468)(61.84205145,253.204)(61.41005145,252.736)
\curveto(61.12205145,252.412)(61.00205145,252.052)(60.97805145,251.428)
\lineto(61.98605145,251.428)
\curveto(62.07005145,252.196)(62.52605145,252.544)(63.46205145,252.544)
\curveto(64.37405145,252.544)(64.86605145,252.208)(64.86605145,251.608)
\lineto(64.86605145,251.344)
\curveto(64.85405145,250.912)(64.63805145,250.756)(63.82205145,250.648)
\curveto(62.40605145,250.468)(62.19005145,250.42)(61.80605145,250.264)
\curveto(61.07405145,249.952)(60.70205145,249.4)(60.70205145,248.584)
\curveto(60.70205145,247.444)(61.49405145,246.724)(62.76605145,246.724)
\curveto(63.55805145,246.724)(64.19405145,247)(64.90205145,247.648)
\curveto(64.97405145,247)(65.28605145,246.724)(65.93405145,246.724)
\curveto(66.15005145,246.724)(66.28205145,246.748)(66.61805145,246.832)
\closepath
\moveto(64.86605145,248.98)
\curveto(64.86605145,248.644)(64.77005145,248.44)(64.47005145,248.164)
\curveto(64.06205145,247.792)(63.57005145,247.6)(62.98205145,247.6)
\curveto(62.20205145,247.6)(61.74605145,247.972)(61.74605145,248.608)
\curveto(61.74605145,249.268)(62.17805145,249.604)(63.25805145,249.76)
\curveto(64.32605145,249.904)(64.53005145,249.952)(64.86605145,250.108)
\closepath
}
}
{
\newrgbcolor{curcolor}{0 0 0}
\pscustom[linestyle=none,fillstyle=solid,fillcolor=curcolor]
{
\newpath
\moveto(68.69404852,255.748)
\lineto(67.68604852,255.748)
\lineto(67.68604852,247)
\lineto(68.69404852,247)
\closepath
}
}
{
\newrgbcolor{curcolor}{0 0 0}
\pscustom[linestyle=none,fillstyle=solid,fillcolor=curcolor]
{
\newpath
\moveto(78.12602563,251.536)
\curveto(78.11402563,252.772)(77.29802563,253.468)(75.84602563,253.468)
\curveto(74.38202563,253.468)(73.43402563,252.712)(73.43402563,251.548)
\curveto(73.43402563,250.564)(73.93802563,250.096)(75.42602563,249.736)
\lineto(76.36202563,249.508)
\curveto(77.05802563,249.34)(77.33402563,249.088)(77.33402563,248.644)
\curveto(77.33402563,248.044)(76.74602563,247.648)(75.87002563,247.648)
\curveto(75.33002563,247.648)(74.87402563,247.804)(74.62202563,248.068)
\curveto(74.46602563,248.248)(74.39402563,248.428)(74.33402563,248.872)
\lineto(73.27802563,248.872)
\curveto(73.32602563,247.42)(74.14202563,246.724)(75.78602563,246.724)
\curveto(77.37002563,246.724)(78.37802563,247.504)(78.37802563,248.716)
\curveto(78.37802563,249.652)(77.85002563,250.168)(76.60202563,250.468)
\lineto(75.64202563,250.696)
\curveto(74.82602563,250.888)(74.47802563,251.152)(74.47802563,251.596)
\curveto(74.47802563,252.184)(74.99402563,252.544)(75.81002563,252.544)
\curveto(76.61402563,252.544)(77.04602563,252.196)(77.07002563,251.536)
\closepath
}
}
{
\newrgbcolor{curcolor}{0 0 0}
\pscustom[linestyle=none,fillstyle=solid,fillcolor=curcolor]
{
\newpath
\moveto(85.02600275,249.808)
\curveto(85.02600275,250.768)(84.95400275,251.344)(84.77400275,251.812)
\curveto(84.36600275,252.844)(83.40600275,253.468)(82.23000275,253.468)
\curveto(80.47800275,253.468)(79.35000275,252.136)(79.35000275,250.06)
\curveto(79.35000275,247.984)(80.43000275,246.724)(82.20600275,246.724)
\curveto(83.64600275,246.724)(84.64200275,247.54)(84.89400275,248.908)
\lineto(83.88600275,248.908)
\curveto(83.61000275,248.08)(83.04600275,247.648)(82.24200275,247.648)
\curveto(81.60600275,247.648)(81.06600275,247.936)(80.73000275,248.464)
\curveto(80.49000275,248.824)(80.40600275,249.184)(80.39400275,249.808)
\closepath
\moveto(80.41800275,250.624)
\curveto(80.50200275,251.788)(81.21000275,252.544)(82.21800275,252.544)
\curveto(83.23800275,252.544)(83.94600275,251.752)(83.94600275,250.624)
\closepath
}
}
{
\newrgbcolor{curcolor}{0 0 0}
\pscustom[linestyle=none,fillstyle=solid,fillcolor=curcolor]
{
\newpath
\moveto(91.87797729,247.588)
\curveto(91.76997729,247.564)(91.72197729,247.564)(91.66197729,247.564)
\curveto(91.31397729,247.564)(91.12197729,247.744)(91.12197729,248.056)
\lineto(91.12197729,251.752)
\curveto(91.12197729,252.868)(90.30597729,253.468)(88.75797729,253.468)
\curveto(87.83397729,253.468)(87.10197729,253.204)(86.66997729,252.736)
\curveto(86.38197729,252.412)(86.26197729,252.052)(86.23797729,251.428)
\lineto(87.24597729,251.428)
\curveto(87.32997729,252.196)(87.78597729,252.544)(88.72197729,252.544)
\curveto(89.63397729,252.544)(90.12597729,252.208)(90.12597729,251.608)
\lineto(90.12597729,251.344)
\curveto(90.11397729,250.912)(89.89797729,250.756)(89.08197729,250.648)
\curveto(87.66597729,250.468)(87.44997729,250.42)(87.06597729,250.264)
\curveto(86.33397729,249.952)(85.96197729,249.4)(85.96197729,248.584)
\curveto(85.96197729,247.444)(86.75397729,246.724)(88.02597729,246.724)
\curveto(88.81797729,246.724)(89.45397729,247)(90.16197729,247.648)
\curveto(90.23397729,247)(90.54597729,246.724)(91.19397729,246.724)
\curveto(91.40997729,246.724)(91.54197729,246.748)(91.87797729,246.832)
\closepath
\moveto(90.12597729,248.98)
\curveto(90.12597729,248.644)(90.02997729,248.44)(89.72997729,248.164)
\curveto(89.32197729,247.792)(88.82997729,247.6)(88.24197729,247.6)
\curveto(87.46197729,247.6)(87.00597729,247.972)(87.00597729,248.608)
\curveto(87.00597729,249.268)(87.43797729,249.604)(88.51797729,249.76)
\curveto(89.58597729,249.904)(89.78997729,249.952)(90.12597729,250.108)
\closepath
}
}
{
\newrgbcolor{curcolor}{0 0 0}
\pscustom[linestyle=none,fillstyle=solid,fillcolor=curcolor]
{
\newpath
\moveto(100.7219729,251.536)
\curveto(100.7099729,252.772)(99.8939729,253.468)(98.4419729,253.468)
\curveto(96.9779729,253.468)(96.0299729,252.712)(96.0299729,251.548)
\curveto(96.0299729,250.564)(96.5339729,250.096)(98.0219729,249.736)
\lineto(98.9579729,249.508)
\curveto(99.6539729,249.34)(99.9299729,249.088)(99.9299729,248.644)
\curveto(99.9299729,248.044)(99.3419729,247.648)(98.4659729,247.648)
\curveto(97.9259729,247.648)(97.4699729,247.804)(97.2179729,248.068)
\curveto(97.0619729,248.248)(96.9899729,248.428)(96.9299729,248.872)
\lineto(95.8739729,248.872)
\curveto(95.9219729,247.42)(96.7379729,246.724)(98.3819729,246.724)
\curveto(99.9659729,246.724)(100.9739729,247.504)(100.9739729,248.716)
\curveto(100.9739729,249.652)(100.4459729,250.168)(99.1979729,250.468)
\lineto(98.2379729,250.696)
\curveto(97.4219729,250.888)(97.0739729,251.152)(97.0739729,251.596)
\curveto(97.0739729,252.184)(97.5899729,252.544)(98.4059729,252.544)
\curveto(99.2099729,252.544)(99.6419729,252.196)(99.6659729,251.536)
\closepath
}
}
{
\newrgbcolor{curcolor}{0 0 0}
\pscustom[linestyle=none,fillstyle=solid,fillcolor=curcolor]
{
\newpath
\moveto(107.24995001,247)
\lineto(107.24995001,253.288)
\lineto(106.25395001,253.288)
\lineto(106.25395001,249.724)
\curveto(106.25395001,248.44)(105.58195001,247.6)(104.53795001,247.6)
\curveto(103.74595001,247.6)(103.24195001,248.08)(103.24195001,248.836)
\lineto(103.24195001,253.288)
\lineto(102.24595001,253.288)
\lineto(102.24595001,248.44)
\curveto(102.24595001,247.396)(103.02595001,246.724)(104.24995001,246.724)
\curveto(105.17395001,246.724)(105.76195001,247.048)(106.34995001,247.876)
\lineto(106.34995001,247)
\closepath
}
}
{
\newrgbcolor{curcolor}{0 0 0}
\pscustom[linestyle=none,fillstyle=solid,fillcolor=curcolor]
{
\newpath
\moveto(108.96594708,253.288)
\lineto(108.96594708,247)
\lineto(109.97394708,247)
\lineto(109.97394708,250.264)
\curveto(109.98594708,251.776)(110.60994708,252.448)(111.98994708,252.412)
\lineto(111.98994708,253.432)
\curveto(111.82194708,253.456)(111.72594708,253.468)(111.60594708,253.468)
\curveto(110.95794708,253.468)(110.46594708,253.084)(109.88994708,252.148)
\lineto(109.88994708,253.288)
\closepath
}
}
{
\newrgbcolor{curcolor}{0 0 0}
\pscustom[linestyle=none,fillstyle=solid,fillcolor=curcolor]
{
\newpath
\moveto(115.52992181,253.288)
\lineto(114.48592181,253.288)
\lineto(114.48592181,254.272)
\curveto(114.48592181,254.692)(114.71392181,254.908)(115.18192181,254.908)
\curveto(115.26592181,254.908)(115.30192181,254.908)(115.52992181,254.896)
\lineto(115.52992181,255.724)
\curveto(115.30192181,255.772)(115.16992181,255.784)(114.96592181,255.784)
\curveto(114.04192181,255.784)(113.48992181,255.256)(113.48992181,254.356)
\lineto(113.48992181,253.288)
\lineto(112.64992181,253.288)
\lineto(112.64992181,252.472)
\lineto(113.48992181,252.472)
\lineto(113.48992181,247)
\lineto(114.48592181,247)
\lineto(114.48592181,252.472)
\lineto(115.52992181,252.472)
\closepath
}
}
{
\newrgbcolor{curcolor}{0 0 0}
\pscustom[linestyle=none,fillstyle=solid,fillcolor=curcolor]
{
\newpath
\moveto(122.09390442,247.588)
\curveto(121.98590442,247.564)(121.93790442,247.564)(121.87790442,247.564)
\curveto(121.52990442,247.564)(121.33790442,247.744)(121.33790442,248.056)
\lineto(121.33790442,251.752)
\curveto(121.33790442,252.868)(120.52190442,253.468)(118.97390442,253.468)
\curveto(118.04990442,253.468)(117.31790442,253.204)(116.88590442,252.736)
\curveto(116.59790442,252.412)(116.47790442,252.052)(116.45390442,251.428)
\lineto(117.46190442,251.428)
\curveto(117.54590442,252.196)(118.00190442,252.544)(118.93790442,252.544)
\curveto(119.84990442,252.544)(120.34190442,252.208)(120.34190442,251.608)
\lineto(120.34190442,251.344)
\curveto(120.32990442,250.912)(120.11390442,250.756)(119.29790442,250.648)
\curveto(117.88190442,250.468)(117.66590442,250.42)(117.28190442,250.264)
\curveto(116.54990442,249.952)(116.17790442,249.4)(116.17790442,248.584)
\curveto(116.17790442,247.444)(116.96990442,246.724)(118.24190442,246.724)
\curveto(119.03390442,246.724)(119.66990442,247)(120.37790442,247.648)
\curveto(120.44990442,247)(120.76190442,246.724)(121.40990442,246.724)
\curveto(121.62590442,246.724)(121.75790442,246.748)(122.09390442,246.832)
\closepath
\moveto(120.34190442,248.98)
\curveto(120.34190442,248.644)(120.24590442,248.44)(119.94590442,248.164)
\curveto(119.53790442,247.792)(119.04590442,247.6)(118.45790442,247.6)
\curveto(117.67790442,247.6)(117.22190442,247.972)(117.22190442,248.608)
\curveto(117.22190442,249.268)(117.65390442,249.604)(118.73390442,249.76)
\curveto(119.80190442,249.904)(120.00590442,249.952)(120.34190442,250.108)
\closepath
}
}
{
\newrgbcolor{curcolor}{0 0 0}
\pscustom[linestyle=none,fillstyle=solid,fillcolor=curcolor]
{
\newpath
\moveto(127.99790149,251.176)
\curveto(127.94990149,251.788)(127.81790149,252.184)(127.57790149,252.532)
\curveto(127.14590149,253.12)(126.38990149,253.468)(125.51390149,253.468)
\curveto(123.80990149,253.468)(122.71790149,252.124)(122.71790149,250.036)
\curveto(122.71790149,248.008)(123.79790149,246.724)(125.50190149,246.724)
\curveto(127.00190149,246.724)(127.94990149,247.624)(128.06990149,249.16)
\lineto(127.06190149,249.16)
\curveto(126.89390149,248.152)(126.37790149,247.648)(125.52590149,247.648)
\curveto(124.42190149,247.648)(123.76190149,248.548)(123.76190149,250.036)
\curveto(123.76190149,251.608)(124.40990149,252.544)(125.50190149,252.544)
\curveto(126.34190149,252.544)(126.86990149,252.052)(126.98990149,251.176)
\closepath
}
}
{
\newrgbcolor{curcolor}{0 0 0}
\pscustom[linestyle=none,fillstyle=solid,fillcolor=curcolor]
{
\newpath
\moveto(134.57386194,249.808)
\curveto(134.57386194,250.768)(134.50186194,251.344)(134.32186194,251.812)
\curveto(133.91386194,252.844)(132.95386194,253.468)(131.77786194,253.468)
\curveto(130.02586194,253.468)(128.89786194,252.136)(128.89786194,250.06)
\curveto(128.89786194,247.984)(129.97786194,246.724)(131.75386194,246.724)
\curveto(133.19386194,246.724)(134.18986194,247.54)(134.44186194,248.908)
\lineto(133.43386194,248.908)
\curveto(133.15786194,248.08)(132.59386194,247.648)(131.78986194,247.648)
\curveto(131.15386194,247.648)(130.61386194,247.936)(130.27786194,248.464)
\curveto(130.03786194,248.824)(129.95386194,249.184)(129.94186194,249.808)
\closepath
\moveto(129.96586194,250.624)
\curveto(130.04986194,251.788)(130.75786194,252.544)(131.76586194,252.544)
\curveto(132.78586194,252.544)(133.49386194,251.752)(133.49386194,250.624)
\closepath
}
}
{
\newrgbcolor{curcolor}{0 0 0}
\pscustom[linestyle=none,fillstyle=solid,fillcolor=curcolor]
{
\newpath
\moveto(141.68985754,253.468)
\curveto(139.91385754,253.468)(138.85785754,252.208)(138.85785754,250.096)
\curveto(138.85785754,247.972)(139.91385754,246.724)(141.70185754,246.724)
\curveto(143.47785754,246.724)(144.54585754,247.984)(144.54585754,250.048)
\curveto(144.54585754,252.232)(143.51385754,253.468)(141.68985754,253.468)
\closepath
\moveto(141.70185754,252.544)
\curveto(142.82985754,252.544)(143.50185754,251.62)(143.50185754,250.06)
\curveto(143.50185754,248.572)(142.80585754,247.648)(141.70185754,247.648)
\curveto(140.58585754,247.648)(139.90185754,248.572)(139.90185754,250.096)
\curveto(139.90185754,251.62)(140.58585754,252.544)(141.70185754,252.544)
\closepath
}
}
{
\newrgbcolor{curcolor}{0 0 0}
\pscustom[linestyle=none,fillstyle=solid,fillcolor=curcolor]
{
\newpath
\moveto(145.93785461,253.288)
\lineto(145.93785461,247)
\lineto(146.94585461,247)
\lineto(146.94585461,250.468)
\curveto(146.94585461,251.752)(147.61785461,252.592)(148.64985461,252.592)
\curveto(149.44185461,252.592)(149.94585461,252.112)(149.94585461,251.356)
\lineto(149.94585461,247)
\lineto(150.94185461,247)
\lineto(150.94185461,251.752)
\curveto(150.94185461,252.796)(150.16185461,253.468)(148.94985461,253.468)
\curveto(148.01385461,253.468)(147.41385461,253.108)(146.86185461,252.232)
\lineto(146.86185461,253.288)
\closepath
}
}
{
\newrgbcolor{curcolor}{0 0 0}
\pscustom[linestyle=none,fillstyle=solid,fillcolor=curcolor]
{
\newpath
\moveto(154.06185168,248.248)
\lineto(152.81385168,248.248)
\lineto(152.81385168,247)
\lineto(154.06185168,247)
\closepath
}
}
{
\newrgbcolor{curcolor}{0 0 0}
\pscustom[linestyle=none,fillstyle=solid,fillcolor=curcolor]
{
\newpath
\moveto(157.39785022,248.248)
\lineto(156.14985022,248.248)
\lineto(156.14985022,247)
\lineto(157.39785022,247)
\closepath
}
}
{
\newrgbcolor{curcolor}{0 0 0}
\pscustom[linestyle=none,fillstyle=solid,fillcolor=curcolor]
{
\newpath
\moveto(160.73384875,248.248)
\lineto(159.48584875,248.248)
\lineto(159.48584875,247)
\lineto(160.73384875,247)
\closepath
}
}
{
\newrgbcolor{curcolor}{0 0 0}
\pscustom[linewidth=1,linecolor=curcolor]
{
\newpath
\moveto(260.53,310.5)
\lineto(260.53,290.5)
\lineto(260.53,310.5)
\lineto(260.53,296.87)
}
}
{
\newrgbcolor{curcolor}{0 0 0}
\pscustom[linestyle=none,fillstyle=solid,fillcolor=curcolor]
{
\newpath
\moveto(260.53,291.62)
\lineto(257.03,298.62)
\lineto(260.53,296.87)
\lineto(264.03,298.62)
\closepath
}
}
{
\newrgbcolor{curcolor}{0 0 0}
\pscustom[linewidth=1,linecolor=curcolor]
{
\newpath
\moveto(260.53,291.62)
\lineto(257.03,298.62)
\lineto(260.53,296.87)
\lineto(264.03,298.62)
\closepath
}
}
{
\newrgbcolor{curcolor}{1 1 1}
\pscustom[linestyle=none,fillstyle=solid,fillcolor=curcolor]
{
\newpath
\moveto(200.5,370.5)
\lineto(320.5,370.5)
\lineto(320.5,310.5)
\lineto(200.5,310.5)
\closepath
}
}
{
\newrgbcolor{curcolor}{0 0 0}
\pscustom[linewidth=1,linecolor=curcolor]
{
\newpath
\moveto(200.5,370.5)
\lineto(320.5,370.5)
\lineto(320.5,310.5)
\lineto(200.5,310.5)
\closepath
}
}
{
\newrgbcolor{curcolor}{0 0 0}
\pscustom[linestyle=none,fillstyle=solid,fillcolor=curcolor]
{
\newpath
\moveto(206.1740878,339.628)
\lineto(207.0740878,337)
\lineto(208.3220878,337)
\lineto(205.2500878,345.748)
\lineto(203.8100878,345.748)
\lineto(200.6900878,337)
\lineto(201.8780878,337)
\lineto(202.8020878,339.628)
\closepath
\moveto(205.8620878,340.564)
\lineto(203.0780878,340.564)
\lineto(204.5180878,344.548)
\closepath
}
}
{
\newrgbcolor{curcolor}{0 0 0}
\pscustom[linestyle=none,fillstyle=solid,fillcolor=curcolor]
{
\newpath
\moveto(213.71008249,341.536)
\curveto(213.69808249,342.772)(212.88208249,343.468)(211.43008249,343.468)
\curveto(209.96608249,343.468)(209.01808249,342.712)(209.01808249,341.548)
\curveto(209.01808249,340.564)(209.52208249,340.096)(211.01008249,339.736)
\lineto(211.94608249,339.508)
\curveto(212.64208249,339.34)(212.91808249,339.088)(212.91808249,338.644)
\curveto(212.91808249,338.044)(212.33008249,337.648)(211.45408249,337.648)
\curveto(210.91408249,337.648)(210.45808249,337.804)(210.20608249,338.068)
\curveto(210.05008249,338.248)(209.97808249,338.428)(209.91808249,338.872)
\lineto(208.86208249,338.872)
\curveto(208.91008249,337.42)(209.72608249,336.724)(211.37008249,336.724)
\curveto(212.95408249,336.724)(213.96208249,337.504)(213.96208249,338.716)
\curveto(213.96208249,339.652)(213.43408249,340.168)(212.18608249,340.468)
\lineto(211.22608249,340.696)
\curveto(210.41008249,340.888)(210.06208249,341.152)(210.06208249,341.596)
\curveto(210.06208249,342.184)(210.57808249,342.544)(211.39408249,342.544)
\curveto(212.19808249,342.544)(212.63008249,342.196)(212.65408249,341.536)
\closepath
}
}
{
\newrgbcolor{curcolor}{0 0 0}
\pscustom[linestyle=none,fillstyle=solid,fillcolor=curcolor]
{
\newpath
\moveto(219.73404642,341.536)
\curveto(219.72204642,342.772)(218.90604642,343.468)(217.45404642,343.468)
\curveto(215.99004642,343.468)(215.04204642,342.712)(215.04204642,341.548)
\curveto(215.04204642,340.564)(215.54604642,340.096)(217.03404642,339.736)
\lineto(217.97004642,339.508)
\curveto(218.66604642,339.34)(218.94204642,339.088)(218.94204642,338.644)
\curveto(218.94204642,338.044)(218.35404642,337.648)(217.47804642,337.648)
\curveto(216.93804642,337.648)(216.48204642,337.804)(216.23004642,338.068)
\curveto(216.07404642,338.248)(216.00204642,338.428)(215.94204642,338.872)
\lineto(214.88604642,338.872)
\curveto(214.93404642,337.42)(215.75004642,336.724)(217.39404642,336.724)
\curveto(218.97804642,336.724)(219.98604642,337.504)(219.98604642,338.716)
\curveto(219.98604642,339.652)(219.45804642,340.168)(218.21004642,340.468)
\lineto(217.25004642,340.696)
\curveto(216.43404642,340.888)(216.08604642,341.152)(216.08604642,341.596)
\curveto(216.08604642,342.184)(216.60204642,342.544)(217.41804642,342.544)
\curveto(218.22204642,342.544)(218.65404642,342.196)(218.67804642,341.536)
\closepath
}
}
{
\newrgbcolor{curcolor}{0 0 0}
\pscustom[linestyle=none,fillstyle=solid,fillcolor=curcolor]
{
\newpath
\moveto(222.27802353,343.288)
\lineto(221.28202353,343.288)
\lineto(221.28202353,337)
\lineto(222.27802353,337)
\closepath
\moveto(222.27802353,345.748)
\lineto(221.27002353,345.748)
\lineto(221.27002353,344.488)
\lineto(222.27802353,344.488)
\closepath
}
}
{
\newrgbcolor{curcolor}{0 0 0}
\pscustom[linestyle=none,fillstyle=solid,fillcolor=curcolor]
{
\newpath
\moveto(227.97798581,343.288)
\lineto(227.97798581,342.376)
\curveto(227.47398581,343.132)(226.92198581,343.468)(226.12998581,343.468)
\curveto(224.60598581,343.468)(223.54998581,342.052)(223.54998581,340.036)
\curveto(223.54998581,338.98)(223.80198581,338.212)(224.34198581,337.576)
\curveto(224.80998581,337.024)(225.40998581,336.724)(226.05798581,336.724)
\curveto(226.81398581,336.724)(227.35398581,337.06)(227.88198581,337.852)
\lineto(227.88198581,337.528)
\curveto(227.88198581,335.848)(227.41398581,335.224)(226.16598581,335.224)
\curveto(225.31398581,335.224)(224.86998581,335.56)(224.77398581,336.28)
\lineto(223.75398581,336.28)
\curveto(223.84998581,335.116)(224.77398581,334.384)(226.14198581,334.384)
\curveto(227.06598581,334.384)(227.83398581,334.684)(228.24198581,335.188)
\curveto(228.72198581,335.776)(228.90198581,336.556)(228.90198581,338.032)
\lineto(228.90198581,343.288)
\closepath
\moveto(226.22598581,342.544)
\curveto(227.28198581,342.544)(227.88198581,341.656)(227.88198581,340.06)
\curveto(227.88198581,338.536)(227.26998581,337.648)(226.22598581,337.648)
\curveto(225.19398581,337.648)(224.59398581,338.548)(224.59398581,340.096)
\curveto(224.59398581,341.632)(225.19398581,342.544)(226.22598581,342.544)
\closepath
}
}
{
\newrgbcolor{curcolor}{0 0 0}
\pscustom[linestyle=none,fillstyle=solid,fillcolor=curcolor]
{
\newpath
\moveto(230.64198288,343.288)
\lineto(230.64198288,337)
\lineto(231.64998288,337)
\lineto(231.64998288,340.468)
\curveto(231.64998288,341.752)(232.32198288,342.592)(233.35398288,342.592)
\curveto(234.14598288,342.592)(234.64998288,342.112)(234.64998288,341.356)
\lineto(234.64998288,337)
\lineto(235.64598288,337)
\lineto(235.64598288,341.752)
\curveto(235.64598288,342.796)(234.86598288,343.468)(233.65398288,343.468)
\curveto(232.71798288,343.468)(232.11798288,343.108)(231.56598288,342.232)
\lineto(231.56598288,343.288)
\closepath
}
}
{
\newrgbcolor{curcolor}{0 0 0}
\pscustom[linestyle=none,fillstyle=solid,fillcolor=curcolor]
{
\newpath
\moveto(240.45797849,334.384)
\lineto(241.46597849,334.384)
\lineto(241.46597849,337.66)
\curveto(241.99397849,337.012)(242.58197849,336.724)(243.39797849,336.724)
\curveto(245.02997849,336.724)(246.08597849,338.032)(246.08597849,340.036)
\curveto(246.08597849,342.148)(245.05397849,343.468)(243.38597849,343.468)
\curveto(242.53397849,343.468)(241.84997849,343.084)(241.38197849,342.34)
\lineto(241.38197849,343.288)
\lineto(240.45797849,343.288)
\closepath
\moveto(243.21797849,342.532)
\curveto(244.32197849,342.532)(245.04197849,341.56)(245.04197849,340.06)
\curveto(245.04197849,338.632)(244.30997849,337.66)(243.21797849,337.66)
\curveto(242.14997849,337.66)(241.46597849,338.62)(241.46597849,340.096)
\curveto(241.46597849,341.572)(242.14997849,342.532)(243.21797849,342.532)
\closepath
}
}
{
\newrgbcolor{curcolor}{0 0 0}
\pscustom[linestyle=none,fillstyle=solid,fillcolor=curcolor]
{
\newpath
\moveto(247.32197556,345.748)
\lineto(247.32197556,337)
\lineto(248.31797556,337)
\lineto(248.31797556,340.468)
\curveto(248.31797556,341.752)(248.98997556,342.592)(250.02197556,342.592)
\curveto(250.35797556,342.592)(250.66997556,342.496)(250.90997556,342.316)
\curveto(251.19797556,342.1)(251.31797556,341.8)(251.31797556,341.356)
\lineto(251.31797556,337)
\lineto(252.31397556,337)
\lineto(252.31397556,341.752)
\curveto(252.31397556,342.808)(251.55797556,343.468)(250.33397556,343.468)
\curveto(249.44597556,343.468)(248.90597556,343.192)(248.31797556,342.424)
\lineto(248.31797556,345.748)
\closepath
}
}
{
\newrgbcolor{curcolor}{0 0 0}
\pscustom[linestyle=none,fillstyle=solid,fillcolor=curcolor]
{
\newpath
\moveto(256.41797263,343.468)
\curveto(254.64197263,343.468)(253.58597263,342.208)(253.58597263,340.096)
\curveto(253.58597263,337.972)(254.64197263,336.724)(256.42997263,336.724)
\curveto(258.20597263,336.724)(259.27397263,337.984)(259.27397263,340.048)
\curveto(259.27397263,342.232)(258.24197263,343.468)(256.41797263,343.468)
\closepath
\moveto(256.42997263,342.544)
\curveto(257.55797263,342.544)(258.22997263,341.62)(258.22997263,340.06)
\curveto(258.22997263,338.572)(257.53397263,337.648)(256.42997263,337.648)
\curveto(255.31397263,337.648)(254.62997263,338.572)(254.62997263,340.096)
\curveto(254.62997263,341.62)(255.31397263,342.544)(256.42997263,342.544)
\closepath
}
}
{
\newrgbcolor{curcolor}{0 0 0}
\pscustom[linestyle=none,fillstyle=solid,fillcolor=curcolor]
{
\newpath
\moveto(262.76596036,343.288)
\lineto(261.73396036,343.288)
\lineto(261.73396036,345.016)
\lineto(260.73796036,345.016)
\lineto(260.73796036,343.288)
\lineto(259.88596036,343.288)
\lineto(259.88596036,342.472)
\lineto(260.73796036,342.472)
\lineto(260.73796036,337.72)
\curveto(260.73796036,337.072)(261.16996036,336.724)(261.94996036,336.724)
\curveto(262.21396036,336.724)(262.42996036,336.748)(262.76596036,336.808)
\lineto(262.76596036,337.648)
\curveto(262.62196036,337.612)(262.48996036,337.6)(262.28596036,337.6)
\curveto(261.85396036,337.6)(261.73396036,337.72)(261.73396036,338.164)
\lineto(261.73396036,342.472)
\lineto(262.76596036,342.472)
\closepath
}
}
{
\newrgbcolor{curcolor}{0 0 0}
\pscustom[linestyle=none,fillstyle=solid,fillcolor=curcolor]
{
\newpath
\moveto(266.17394644,343.468)
\curveto(264.39794644,343.468)(263.34194644,342.208)(263.34194644,340.096)
\curveto(263.34194644,337.972)(264.39794644,336.724)(266.18594644,336.724)
\curveto(267.96194644,336.724)(269.02994644,337.984)(269.02994644,340.048)
\curveto(269.02994644,342.232)(267.99794644,343.468)(266.17394644,343.468)
\closepath
\moveto(266.18594644,342.544)
\curveto(267.31394644,342.544)(267.98594644,341.62)(267.98594644,340.06)
\curveto(267.98594644,338.572)(267.28994644,337.648)(266.18594644,337.648)
\curveto(265.06994644,337.648)(264.38594644,338.572)(264.38594644,340.096)
\curveto(264.38594644,341.62)(265.06994644,342.544)(266.18594644,342.544)
\closepath
}
}
{
\newrgbcolor{curcolor}{0 0 0}
\pscustom[linestyle=none,fillstyle=solid,fillcolor=curcolor]
{
\newpath
\moveto(270.42194351,343.288)
\lineto(270.42194351,337)
\lineto(271.42994351,337)
\lineto(271.42994351,340.468)
\curveto(271.42994351,341.752)(272.10194351,342.592)(273.13394351,342.592)
\curveto(273.92594351,342.592)(274.42994351,342.112)(274.42994351,341.356)
\lineto(274.42994351,337)
\lineto(275.42594351,337)
\lineto(275.42594351,341.752)
\curveto(275.42594351,342.796)(274.64594351,343.468)(273.43394351,343.468)
\curveto(272.49794351,343.468)(271.89794351,343.108)(271.34594351,342.232)
\lineto(271.34594351,343.288)
\closepath
}
}
{
\newrgbcolor{curcolor}{0 0 0}
\pscustom[linestyle=none,fillstyle=solid,fillcolor=curcolor]
{
\newpath
\moveto(285.52993912,345.748)
\lineto(284.53393912,345.748)
\lineto(284.53393912,342.496)
\curveto(284.11393912,343.132)(283.44193912,343.468)(282.60193912,343.468)
\curveto(280.96993912,343.468)(279.90193912,342.16)(279.90193912,340.156)
\curveto(279.90193912,338.032)(280.93393912,336.724)(282.63793912,336.724)
\curveto(283.50193912,336.724)(284.10193912,337.048)(284.64193912,337.828)
\lineto(284.64193912,337)
\lineto(285.52993912,337)
\closepath
\moveto(282.76993912,342.532)
\curveto(283.84993912,342.532)(284.53393912,341.584)(284.53393912,340.072)
\curveto(284.53393912,338.62)(283.83793912,337.66)(282.78193912,337.66)
\curveto(281.67793912,337.66)(280.94593912,338.632)(280.94593912,340.096)
\curveto(280.94593912,341.56)(281.67793912,342.532)(282.76993912,342.532)
\closepath
}
}
{
\newrgbcolor{curcolor}{0 0 0}
\pscustom[linestyle=none,fillstyle=solid,fillcolor=curcolor]
{
\newpath
\moveto(292.41793619,339.808)
\curveto(292.41793619,340.768)(292.34593619,341.344)(292.16593619,341.812)
\curveto(291.75793619,342.844)(290.79793619,343.468)(289.62193619,343.468)
\curveto(287.86993619,343.468)(286.74193619,342.136)(286.74193619,340.06)
\curveto(286.74193619,337.984)(287.82193619,336.724)(289.59793619,336.724)
\curveto(291.03793619,336.724)(292.03393619,337.54)(292.28593619,338.908)
\lineto(291.27793619,338.908)
\curveto(291.00193619,338.08)(290.43793619,337.648)(289.63393619,337.648)
\curveto(288.99793619,337.648)(288.45793619,337.936)(288.12193619,338.464)
\curveto(287.88193619,338.824)(287.79793619,339.184)(287.78593619,339.808)
\closepath
\moveto(287.80993619,340.624)
\curveto(287.89393619,341.788)(288.60193619,342.544)(289.60993619,342.544)
\curveto(290.62993619,342.544)(291.33793619,341.752)(291.33793619,340.624)
\closepath
}
}
{
\newrgbcolor{curcolor}{0 0 0}
\pscustom[linestyle=none,fillstyle=solid,fillcolor=curcolor]
{
\newpath
\moveto(293.58193326,334.384)
\lineto(294.58993326,334.384)
\lineto(294.58993326,337.66)
\curveto(295.11793326,337.012)(295.70593326,336.724)(296.52193326,336.724)
\curveto(298.15393326,336.724)(299.20993326,338.032)(299.20993326,340.036)
\curveto(299.20993326,342.148)(298.17793326,343.468)(296.50993326,343.468)
\curveto(295.65793326,343.468)(294.97393326,343.084)(294.50593326,342.34)
\lineto(294.50593326,343.288)
\lineto(293.58193326,343.288)
\closepath
\moveto(296.34193326,342.532)
\curveto(297.44593326,342.532)(298.16593326,341.56)(298.16593326,340.06)
\curveto(298.16593326,338.632)(297.43393326,337.66)(296.34193326,337.66)
\curveto(295.27393326,337.66)(294.58993326,338.62)(294.58993326,340.096)
\curveto(294.58993326,341.572)(295.27393326,342.532)(296.34193326,342.532)
\closepath
}
}
{
\newrgbcolor{curcolor}{0 0 0}
\pscustom[linestyle=none,fillstyle=solid,fillcolor=curcolor]
{
\newpath
\moveto(302.61792722,343.288)
\lineto(301.58592722,343.288)
\lineto(301.58592722,345.016)
\lineto(300.58992722,345.016)
\lineto(300.58992722,343.288)
\lineto(299.73792722,343.288)
\lineto(299.73792722,342.472)
\lineto(300.58992722,342.472)
\lineto(300.58992722,337.72)
\curveto(300.58992722,337.072)(301.02192722,336.724)(301.80192722,336.724)
\curveto(302.06592722,336.724)(302.28192722,336.748)(302.61792722,336.808)
\lineto(302.61792722,337.648)
\curveto(302.47392722,337.612)(302.34192722,337.6)(302.13792722,337.6)
\curveto(301.70592722,337.6)(301.58592722,337.72)(301.58592722,338.164)
\lineto(301.58592722,342.472)
\lineto(302.61792722,342.472)
\closepath
}
}
{
\newrgbcolor{curcolor}{0 0 0}
\pscustom[linestyle=none,fillstyle=solid,fillcolor=curcolor]
{
\newpath
\moveto(303.67391953,345.748)
\lineto(303.67391953,337)
\lineto(304.66991953,337)
\lineto(304.66991953,340.468)
\curveto(304.66991953,341.752)(305.34191953,342.592)(306.37391953,342.592)
\curveto(306.70991953,342.592)(307.02191953,342.496)(307.26191953,342.316)
\curveto(307.54991953,342.1)(307.66991953,341.8)(307.66991953,341.356)
\lineto(307.66991953,337)
\lineto(308.66591953,337)
\lineto(308.66591953,341.752)
\curveto(308.66591953,342.808)(307.90991953,343.468)(306.68591953,343.468)
\curveto(305.79791953,343.468)(305.25791953,343.192)(304.66991953,342.424)
\lineto(304.66991953,345.748)
\closepath
}
}
{
\newrgbcolor{curcolor}{0 0 0}
\pscustom[linestyle=none,fillstyle=solid,fillcolor=curcolor]
{
\newpath
\moveto(311.7979166,338.248)
\lineto(310.5499166,338.248)
\lineto(310.5499166,337)
\lineto(311.7979166,337)
\closepath
}
}
{
\newrgbcolor{curcolor}{0 0 0}
\pscustom[linestyle=none,fillstyle=solid,fillcolor=curcolor]
{
\newpath
\moveto(315.13391513,338.248)
\lineto(313.88591513,338.248)
\lineto(313.88591513,337)
\lineto(315.13391513,337)
\closepath
}
}
{
\newrgbcolor{curcolor}{0 0 0}
\pscustom[linestyle=none,fillstyle=solid,fillcolor=curcolor]
{
\newpath
\moveto(318.46991367,338.248)
\lineto(317.22191367,338.248)
\lineto(317.22191367,337)
\lineto(318.46991367,337)
\closepath
}
}
{
\newrgbcolor{curcolor}{0 0 0}
\pscustom[linewidth=1,linecolor=curcolor]
{
\newpath
\moveto(200.5,265.47)
\lineto(180.53,265.47)
\lineto(180.53,250.5)
\lineto(176.87,250.5)
}
}
{
\newrgbcolor{curcolor}{0 0 0}
\pscustom[linestyle=none,fillstyle=solid,fillcolor=curcolor]
{
\newpath
\moveto(171.62,250.5)
\lineto(178.62,254)
\lineto(176.87,250.5)
\lineto(178.62,247)
\closepath
}
}
{
\newrgbcolor{curcolor}{0 0 0}
\pscustom[linewidth=1,linecolor=curcolor]
{
\newpath
\moveto(171.62,250.5)
\lineto(178.62,254)
\lineto(176.87,250.5)
\lineto(178.62,247)
\closepath
}
}
{
\newrgbcolor{curcolor}{1 1 1}
\pscustom[linestyle=none,fillstyle=solid,fillcolor=curcolor]
{
\newpath
\moveto(200.5,290.5)
\lineto(320.5,290.5)
\lineto(320.5,240.5)
\lineto(200.5,240.5)
\closepath
}
}
{
\newrgbcolor{curcolor}{0 0 0}
\pscustom[linewidth=1,linecolor=curcolor]
{
\newpath
\moveto(200.5,290.5)
\lineto(320.5,290.5)
\lineto(320.5,240.5)
\lineto(200.5,240.5)
\closepath
}
}
{
\newrgbcolor{curcolor}{0 0 0}
\pscustom[linestyle=none,fillstyle=solid,fillcolor=curcolor]
{
\newpath
\moveto(200.97210849,265.768)
\lineto(203.85210849,265.768)
\curveto(204.84810849,265.768)(205.29210849,265.288)(205.29210849,264.208)
\lineto(205.28010849,263.428)
\curveto(205.28010849,262.888)(205.37610849,262.36)(205.53210849,262)
\lineto(206.88810849,262)
\lineto(206.88810849,262.276)
\curveto(206.46810849,262.564)(206.38410849,262.876)(206.36010849,264.04)
\curveto(206.34810849,265.48)(206.12010849,265.912)(205.17210849,266.32)
\curveto(206.15610849,266.812)(206.55210849,267.4)(206.55210849,268.408)
\curveto(206.55210849,269.92)(205.61610849,270.748)(203.88810849,270.748)
\lineto(199.85610849,270.748)
\lineto(199.85610849,262)
\lineto(200.97210849,262)
\closepath
\moveto(200.97210849,266.752)
\lineto(200.97210849,269.764)
\lineto(203.67210849,269.764)
\curveto(204.29610849,269.764)(204.65610849,269.668)(204.93210849,269.428)
\curveto(205.23210849,269.176)(205.38810849,268.78)(205.38810849,268.264)
\curveto(205.38810849,267.22)(204.86010849,266.752)(203.67210849,266.752)
\closepath
}
}
{
\newrgbcolor{curcolor}{0 0 0}
\pscustom[linestyle=none,fillstyle=solid,fillcolor=curcolor]
{
\newpath
\moveto(213.44008432,264.808)
\curveto(213.44008432,265.768)(213.36808432,266.344)(213.18808432,266.812)
\curveto(212.78008432,267.844)(211.82008432,268.468)(210.64408432,268.468)
\curveto(208.89208432,268.468)(207.76408432,267.136)(207.76408432,265.06)
\curveto(207.76408432,262.984)(208.84408432,261.724)(210.62008432,261.724)
\curveto(212.06008432,261.724)(213.05608432,262.54)(213.30808432,263.908)
\lineto(212.30008432,263.908)
\curveto(212.02408432,263.08)(211.46008432,262.648)(210.65608432,262.648)
\curveto(210.02008432,262.648)(209.48008432,262.936)(209.14408432,263.464)
\curveto(208.90408432,263.824)(208.82008432,264.184)(208.80808432,264.808)
\closepath
\moveto(208.83208432,265.624)
\curveto(208.91608432,266.788)(209.62408432,267.544)(210.63208432,267.544)
\curveto(211.65208432,267.544)(212.36008432,266.752)(212.36008432,265.624)
\closepath
}
}
{
\newrgbcolor{curcolor}{0 0 0}
\pscustom[linestyle=none,fillstyle=solid,fillcolor=curcolor]
{
\newpath
\moveto(214.79608139,268.288)
\lineto(214.79608139,262)
\lineto(215.80408139,262)
\lineto(215.80408139,265.948)
\curveto(215.80408139,266.86)(216.46408139,267.592)(217.28008139,267.592)
\curveto(218.02408139,267.592)(218.44408139,267.136)(218.44408139,266.332)
\lineto(218.44408139,262)
\lineto(219.45208139,262)
\lineto(219.45208139,265.948)
\curveto(219.45208139,266.86)(220.11208139,267.592)(220.92808139,267.592)
\curveto(221.66008139,267.592)(222.09208139,267.124)(222.09208139,266.332)
\lineto(222.09208139,262)
\lineto(223.10008139,262)
\lineto(223.10008139,266.716)
\curveto(223.10008139,267.844)(222.45208139,268.468)(221.27608139,268.468)
\curveto(220.43608139,268.468)(219.93208139,268.216)(219.34408139,267.508)
\curveto(218.97208139,268.18)(218.46808139,268.468)(217.65208139,268.468)
\curveto(216.81208139,268.468)(216.24808139,268.156)(215.72008139,267.4)
\lineto(215.72008139,268.288)
\closepath
}
}
{
\newrgbcolor{curcolor}{0 0 0}
\pscustom[linestyle=none,fillstyle=solid,fillcolor=curcolor]
{
\newpath
\moveto(227.2160607,268.468)
\curveto(225.4400607,268.468)(224.3840607,267.208)(224.3840607,265.096)
\curveto(224.3840607,262.972)(225.4400607,261.724)(227.2280607,261.724)
\curveto(229.0040607,261.724)(230.0720607,262.984)(230.0720607,265.048)
\curveto(230.0720607,267.232)(229.0400607,268.468)(227.2160607,268.468)
\closepath
\moveto(227.2280607,267.544)
\curveto(228.3560607,267.544)(229.0280607,266.62)(229.0280607,265.06)
\curveto(229.0280607,263.572)(228.3320607,262.648)(227.2280607,262.648)
\curveto(226.1120607,262.648)(225.4280607,263.572)(225.4280607,265.096)
\curveto(225.4280607,266.62)(226.1120607,267.544)(227.2280607,267.544)
\closepath
}
}
{
\newrgbcolor{curcolor}{0 0 0}
\pscustom[linestyle=none,fillstyle=solid,fillcolor=curcolor]
{
\newpath
\moveto(233.82803909,262)
\lineto(236.24003909,268.288)
\lineto(235.11203909,268.288)
\lineto(233.33603909,263.188)
\lineto(231.65603909,268.288)
\lineto(230.52803909,268.288)
\lineto(232.73603909,262)
\closepath
}
}
{
\newrgbcolor{curcolor}{0 0 0}
\pscustom[linestyle=none,fillstyle=solid,fillcolor=curcolor]
{
\newpath
\moveto(242.32401071,264.808)
\curveto(242.32401071,265.768)(242.25201071,266.344)(242.07201071,266.812)
\curveto(241.66401071,267.844)(240.70401071,268.468)(239.52801071,268.468)
\curveto(237.77601071,268.468)(236.64801071,267.136)(236.64801071,265.06)
\curveto(236.64801071,262.984)(237.72801071,261.724)(239.50401071,261.724)
\curveto(240.94401071,261.724)(241.94001071,262.54)(242.19201071,263.908)
\lineto(241.18401071,263.908)
\curveto(240.90801071,263.08)(240.34401071,262.648)(239.54001071,262.648)
\curveto(238.90401071,262.648)(238.36401071,262.936)(238.02801071,263.464)
\curveto(237.78801071,263.824)(237.70401071,264.184)(237.69201071,264.808)
\closepath
\moveto(237.71601071,265.624)
\curveto(237.80001071,266.788)(238.50801071,267.544)(239.51601071,267.544)
\curveto(240.53601071,267.544)(241.24401071,266.752)(241.24401071,265.624)
\closepath
}
}
{
\newrgbcolor{curcolor}{0 0 0}
\pscustom[linestyle=none,fillstyle=solid,fillcolor=curcolor]
{
\newpath
\moveto(246.82400632,259.384)
\lineto(247.83200632,259.384)
\lineto(247.83200632,262.66)
\curveto(248.36000632,262.012)(248.94800632,261.724)(249.76400632,261.724)
\curveto(251.39600632,261.724)(252.45200632,263.032)(252.45200632,265.036)
\curveto(252.45200632,267.148)(251.42000632,268.468)(249.75200632,268.468)
\curveto(248.90000632,268.468)(248.21600632,268.084)(247.74800632,267.34)
\lineto(247.74800632,268.288)
\lineto(246.82400632,268.288)
\closepath
\moveto(249.58400632,267.532)
\curveto(250.68800632,267.532)(251.40800632,266.56)(251.40800632,265.06)
\curveto(251.40800632,263.632)(250.67600632,262.66)(249.58400632,262.66)
\curveto(248.51600632,262.66)(247.83200632,263.62)(247.83200632,265.096)
\curveto(247.83200632,266.572)(248.51600632,267.532)(249.58400632,267.532)
\closepath
}
}
{
\newrgbcolor{curcolor}{0 0 0}
\pscustom[linestyle=none,fillstyle=solid,fillcolor=curcolor]
{
\newpath
\moveto(256.11200339,268.468)
\curveto(254.33600339,268.468)(253.28000339,267.208)(253.28000339,265.096)
\curveto(253.28000339,262.972)(254.33600339,261.724)(256.12400339,261.724)
\curveto(257.90000339,261.724)(258.96800339,262.984)(258.96800339,265.048)
\curveto(258.96800339,267.232)(257.93600339,268.468)(256.11200339,268.468)
\closepath
\moveto(256.12400339,267.544)
\curveto(257.25200339,267.544)(257.92400339,266.62)(257.92400339,265.06)
\curveto(257.92400339,263.572)(257.22800339,262.648)(256.12400339,262.648)
\curveto(255.00800339,262.648)(254.32400339,263.572)(254.32400339,265.096)
\curveto(254.32400339,266.62)(255.00800339,267.544)(256.12400339,267.544)
\closepath
}
}
{
\newrgbcolor{curcolor}{0 0 0}
\pscustom[linestyle=none,fillstyle=solid,fillcolor=curcolor]
{
\newpath
\moveto(261.32000046,268.288)
\lineto(260.32400046,268.288)
\lineto(260.32400046,262)
\lineto(261.32000046,262)
\closepath
\moveto(261.32000046,270.748)
\lineto(260.31200046,270.748)
\lineto(260.31200046,269.488)
\lineto(261.32000046,269.488)
\closepath
}
}
{
\newrgbcolor{curcolor}{0 0 0}
\pscustom[linestyle=none,fillstyle=solid,fillcolor=curcolor]
{
\newpath
\moveto(263.02397903,268.288)
\lineto(263.02397903,262)
\lineto(264.03197903,262)
\lineto(264.03197903,265.468)
\curveto(264.03197903,266.752)(264.70397903,267.592)(265.73597903,267.592)
\curveto(266.52797903,267.592)(267.03197903,267.112)(267.03197903,266.356)
\lineto(267.03197903,262)
\lineto(268.02797903,262)
\lineto(268.02797903,266.752)
\curveto(268.02797903,267.796)(267.24797903,268.468)(266.03597903,268.468)
\curveto(265.09997903,268.468)(264.49997903,268.108)(263.94797903,267.232)
\lineto(263.94797903,268.288)
\closepath
}
}
{
\newrgbcolor{curcolor}{0 0 0}
\pscustom[linestyle=none,fillstyle=solid,fillcolor=curcolor]
{
\newpath
\moveto(271.9039761,268.288)
\lineto(270.8719761,268.288)
\lineto(270.8719761,270.016)
\lineto(269.8759761,270.016)
\lineto(269.8759761,268.288)
\lineto(269.0239761,268.288)
\lineto(269.0239761,267.472)
\lineto(269.8759761,267.472)
\lineto(269.8759761,262.72)
\curveto(269.8759761,262.072)(270.3079761,261.724)(271.0879761,261.724)
\curveto(271.3519761,261.724)(271.5679761,261.748)(271.9039761,261.808)
\lineto(271.9039761,262.648)
\curveto(271.7599761,262.612)(271.6279761,262.6)(271.4239761,262.6)
\curveto(270.9919761,262.6)(270.8719761,262.72)(270.8719761,263.164)
\lineto(270.8719761,267.472)
\lineto(271.9039761,267.472)
\closepath
}
}
{
\newrgbcolor{curcolor}{0 0 0}
\pscustom[linestyle=none,fillstyle=solid,fillcolor=curcolor]
{
\newpath
\moveto(277.44797464,266.536)
\curveto(277.43597464,267.772)(276.61997464,268.468)(275.16797464,268.468)
\curveto(273.70397464,268.468)(272.75597464,267.712)(272.75597464,266.548)
\curveto(272.75597464,265.564)(273.25997464,265.096)(274.74797464,264.736)
\lineto(275.68397464,264.508)
\curveto(276.37997464,264.34)(276.65597464,264.088)(276.65597464,263.644)
\curveto(276.65597464,263.044)(276.06797464,262.648)(275.19197464,262.648)
\curveto(274.65197464,262.648)(274.19597464,262.804)(273.94397464,263.068)
\curveto(273.78797464,263.248)(273.71597464,263.428)(273.65597464,263.872)
\lineto(272.59997464,263.872)
\curveto(272.64797464,262.42)(273.46397464,261.724)(275.10797464,261.724)
\curveto(276.69197464,261.724)(277.69997464,262.504)(277.69997464,263.716)
\curveto(277.69997464,264.652)(277.17197464,265.168)(275.92397464,265.468)
\lineto(274.96397464,265.696)
\curveto(274.14797464,265.888)(273.79997464,266.152)(273.79997464,266.596)
\curveto(273.79997464,267.184)(274.31597464,267.544)(275.13197464,267.544)
\curveto(275.93597464,267.544)(276.36797464,267.196)(276.39197464,266.536)
\closepath
}
}
{
\newrgbcolor{curcolor}{0 0 0}
\pscustom[linestyle=none,fillstyle=solid,fillcolor=curcolor]
{
\newpath
\moveto(284.57595029,268.288)
\lineto(283.54395029,268.288)
\lineto(283.54395029,270.016)
\lineto(282.54795029,270.016)
\lineto(282.54795029,268.288)
\lineto(281.69595029,268.288)
\lineto(281.69595029,267.472)
\lineto(282.54795029,267.472)
\lineto(282.54795029,262.72)
\curveto(282.54795029,262.072)(282.97995029,261.724)(283.75995029,261.724)
\curveto(284.02395029,261.724)(284.23995029,261.748)(284.57595029,261.808)
\lineto(284.57595029,262.648)
\curveto(284.43195029,262.612)(284.29995029,262.6)(284.09595029,262.6)
\curveto(283.66395029,262.6)(283.54395029,262.72)(283.54395029,263.164)
\lineto(283.54395029,267.472)
\lineto(284.57595029,267.472)
\closepath
}
}
{
\newrgbcolor{curcolor}{0 0 0}
\pscustom[linestyle=none,fillstyle=solid,fillcolor=curcolor]
{
\newpath
\moveto(285.6319426,270.748)
\lineto(285.6319426,262)
\lineto(286.6279426,262)
\lineto(286.6279426,265.468)
\curveto(286.6279426,266.752)(287.2999426,267.592)(288.3319426,267.592)
\curveto(288.6679426,267.592)(288.9799426,267.496)(289.2199426,267.316)
\curveto(289.5079426,267.1)(289.6279426,266.8)(289.6279426,266.356)
\lineto(289.6279426,262)
\lineto(290.6239426,262)
\lineto(290.6239426,266.752)
\curveto(290.6239426,267.808)(289.8679426,268.468)(288.6439426,268.468)
\curveto(287.7559426,268.468)(287.2159426,268.192)(286.6279426,267.424)
\lineto(286.6279426,270.748)
\closepath
}
}
{
\newrgbcolor{curcolor}{0 0 0}
\pscustom[linestyle=none,fillstyle=solid,fillcolor=curcolor]
{
\newpath
\moveto(297.88393967,262.588)
\curveto(297.77593967,262.564)(297.72793967,262.564)(297.66793967,262.564)
\curveto(297.31993967,262.564)(297.12793967,262.744)(297.12793967,263.056)
\lineto(297.12793967,266.752)
\curveto(297.12793967,267.868)(296.31193967,268.468)(294.76393967,268.468)
\curveto(293.83993967,268.468)(293.10793967,268.204)(292.67593967,267.736)
\curveto(292.38793967,267.412)(292.26793967,267.052)(292.24393967,266.428)
\lineto(293.25193967,266.428)
\curveto(293.33593967,267.196)(293.79193967,267.544)(294.72793967,267.544)
\curveto(295.63993967,267.544)(296.13193967,267.208)(296.13193967,266.608)
\lineto(296.13193967,266.344)
\curveto(296.11993967,265.912)(295.90393967,265.756)(295.08793967,265.648)
\curveto(293.67193967,265.468)(293.45593967,265.42)(293.07193967,265.264)
\curveto(292.33993967,264.952)(291.96793967,264.4)(291.96793967,263.584)
\curveto(291.96793967,262.444)(292.75993967,261.724)(294.03193967,261.724)
\curveto(294.82393967,261.724)(295.45993967,262)(296.16793967,262.648)
\curveto(296.23993967,262)(296.55193967,261.724)(297.19993967,261.724)
\curveto(297.41593967,261.724)(297.54793967,261.748)(297.88393967,261.832)
\closepath
\moveto(296.13193967,263.98)
\curveto(296.13193967,263.644)(296.03593967,263.44)(295.73593967,263.164)
\curveto(295.32793967,262.792)(294.83593967,262.6)(294.24793967,262.6)
\curveto(293.46793967,262.6)(293.01193967,262.972)(293.01193967,263.608)
\curveto(293.01193967,264.268)(293.44393967,264.604)(294.52393967,264.76)
\curveto(295.59193967,264.904)(295.79593967,264.952)(296.13193967,265.108)
\closepath
}
}
{
\newrgbcolor{curcolor}{0 0 0}
\pscustom[linestyle=none,fillstyle=solid,fillcolor=curcolor]
{
\newpath
\moveto(301.05191769,268.288)
\lineto(300.01991769,268.288)
\lineto(300.01991769,270.016)
\lineto(299.02391769,270.016)
\lineto(299.02391769,268.288)
\lineto(298.17191769,268.288)
\lineto(298.17191769,267.472)
\lineto(299.02391769,267.472)
\lineto(299.02391769,262.72)
\curveto(299.02391769,262.072)(299.45591769,261.724)(300.23591769,261.724)
\curveto(300.49991769,261.724)(300.71591769,261.748)(301.05191769,261.808)
\lineto(301.05191769,262.648)
\curveto(300.90791769,262.612)(300.77591769,262.6)(300.57191769,262.6)
\curveto(300.13991769,262.6)(300.01991769,262.72)(300.01991769,263.164)
\lineto(300.01991769,267.472)
\lineto(301.05191769,267.472)
\closepath
}
}
{
\newrgbcolor{curcolor}{0 0 0}
\pscustom[linestyle=none,fillstyle=solid,fillcolor=curcolor]
{
\newpath
\moveto(311.09591476,262.588)
\curveto(310.98791476,262.564)(310.93991476,262.564)(310.87991476,262.564)
\curveto(310.53191476,262.564)(310.33991476,262.744)(310.33991476,263.056)
\lineto(310.33991476,266.752)
\curveto(310.33991476,267.868)(309.52391476,268.468)(307.97591476,268.468)
\curveto(307.05191476,268.468)(306.31991476,268.204)(305.88791476,267.736)
\curveto(305.59991476,267.412)(305.47991476,267.052)(305.45591476,266.428)
\lineto(306.46391476,266.428)
\curveto(306.54791476,267.196)(307.00391476,267.544)(307.93991476,267.544)
\curveto(308.85191476,267.544)(309.34391476,267.208)(309.34391476,266.608)
\lineto(309.34391476,266.344)
\curveto(309.33191476,265.912)(309.11591476,265.756)(308.29991476,265.648)
\curveto(306.88391476,265.468)(306.66791476,265.42)(306.28391476,265.264)
\curveto(305.55191476,264.952)(305.17991476,264.4)(305.17991476,263.584)
\curveto(305.17991476,262.444)(305.97191476,261.724)(307.24391476,261.724)
\curveto(308.03591476,261.724)(308.67191476,262)(309.37991476,262.648)
\curveto(309.45191476,262)(309.76391476,261.724)(310.41191476,261.724)
\curveto(310.62791476,261.724)(310.75991476,261.748)(311.09591476,261.832)
\closepath
\moveto(309.34391476,263.98)
\curveto(309.34391476,263.644)(309.24791476,263.44)(308.94791476,263.164)
\curveto(308.53991476,262.792)(308.04791476,262.6)(307.45991476,262.6)
\curveto(306.67991476,262.6)(306.22391476,262.972)(306.22391476,263.608)
\curveto(306.22391476,264.268)(306.65591476,264.604)(307.73591476,264.76)
\curveto(308.80391476,264.904)(309.00791476,264.952)(309.34391476,265.108)
\closepath
}
}
{
\newrgbcolor{curcolor}{0 0 0}
\pscustom[linestyle=none,fillstyle=solid,fillcolor=curcolor]
{
\newpath
\moveto(313.5438959,263.248)
\lineto(312.2958959,263.248)
\lineto(312.2958959,262)
\lineto(313.5438959,262)
\closepath
}
}
{
\newrgbcolor{curcolor}{0 0 0}
\pscustom[linestyle=none,fillstyle=solid,fillcolor=curcolor]
{
\newpath
\moveto(316.87989444,263.248)
\lineto(315.63189444,263.248)
\lineto(315.63189444,262)
\lineto(316.87989444,262)
\closepath
}
}
{
\newrgbcolor{curcolor}{0 0 0}
\pscustom[linestyle=none,fillstyle=solid,fillcolor=curcolor]
{
\newpath
\moveto(320.21589297,263.248)
\lineto(318.96789297,263.248)
\lineto(318.96789297,262)
\lineto(320.21589297,262)
\closepath
}
}
{
\newrgbcolor{curcolor}{0 0 0}
\pscustom[linewidth=1,linecolor=curcolor]
{
\newpath
\moveto(165.5,145.47)
\lineto(185.5,145.47)
\lineto(180.53,145.47)
\lineto(194.13,145.47)
}
}
{
\newrgbcolor{curcolor}{0 0 0}
\pscustom[linestyle=none,fillstyle=solid,fillcolor=curcolor]
{
\newpath
\moveto(199.38,145.47)
\lineto(192.38,141.97)
\lineto(194.13,145.47)
\lineto(192.38,148.97)
\closepath
}
}
{
\newrgbcolor{curcolor}{0 0 0}
\pscustom[linewidth=1,linecolor=curcolor]
{
\newpath
\moveto(199.38,145.47)
\lineto(192.38,141.97)
\lineto(194.13,145.47)
\lineto(192.38,148.97)
\closepath
}
}
{
\newrgbcolor{curcolor}{1 1 1}
\pscustom[linestyle=none,fillstyle=solid,fillcolor=curcolor]
{
\newpath
\moveto(15.5,180.5)
\lineto(165.5,180.5)
\lineto(165.5,110.5)
\lineto(15.5,110.5)
\closepath
}
}
{
\newrgbcolor{curcolor}{0 0 0}
\pscustom[linewidth=1,linecolor=curcolor]
{
\newpath
\moveto(15.5,180.5)
\lineto(165.5,180.5)
\lineto(165.5,110.5)
\lineto(15.5,110.5)
\closepath
}
}
{
\newrgbcolor{curcolor}{0 0 0}
\pscustom[linestyle=none,fillstyle=solid,fillcolor=curcolor]
{
\newpath
\moveto(18.99612964,145.768)
\lineto(21.87612964,145.768)
\curveto(22.87212964,145.768)(23.31612964,145.288)(23.31612964,144.208)
\lineto(23.30412964,143.428)
\curveto(23.30412964,142.888)(23.40012964,142.36)(23.55612964,142)
\lineto(24.91212964,142)
\lineto(24.91212964,142.276)
\curveto(24.49212964,142.564)(24.40812964,142.876)(24.38412964,144.04)
\curveto(24.37212964,145.48)(24.14412964,145.912)(23.19612964,146.32)
\curveto(24.18012964,146.812)(24.57612964,147.4)(24.57612964,148.408)
\curveto(24.57612964,149.92)(23.64012964,150.748)(21.91212964,150.748)
\lineto(17.88012964,150.748)
\lineto(17.88012964,142)
\lineto(18.99612964,142)
\closepath
\moveto(18.99612964,146.752)
\lineto(18.99612964,149.764)
\lineto(21.69612964,149.764)
\curveto(22.32012964,149.764)(22.68012964,149.668)(22.95612964,149.428)
\curveto(23.25612964,149.176)(23.41212964,148.78)(23.41212964,148.264)
\curveto(23.41212964,147.22)(22.88412964,146.752)(21.69612964,146.752)
\closepath
}
}
{
\newrgbcolor{curcolor}{0 0 0}
\pscustom[linestyle=none,fillstyle=solid,fillcolor=curcolor]
{
\newpath
\moveto(31.46410547,144.808)
\curveto(31.46410547,145.768)(31.39210547,146.344)(31.21210547,146.812)
\curveto(30.80410547,147.844)(29.84410547,148.468)(28.66810547,148.468)
\curveto(26.91610547,148.468)(25.78810547,147.136)(25.78810547,145.06)
\curveto(25.78810547,142.984)(26.86810547,141.724)(28.64410547,141.724)
\curveto(30.08410547,141.724)(31.08010547,142.54)(31.33210547,143.908)
\lineto(30.32410547,143.908)
\curveto(30.04810547,143.08)(29.48410547,142.648)(28.68010547,142.648)
\curveto(28.04410547,142.648)(27.50410547,142.936)(27.16810547,143.464)
\curveto(26.92810547,143.824)(26.84410547,144.184)(26.83210547,144.808)
\closepath
\moveto(26.85610547,145.624)
\curveto(26.94010547,146.788)(27.64810547,147.544)(28.65610547,147.544)
\curveto(29.67610547,147.544)(30.38410547,146.752)(30.38410547,145.624)
\closepath
}
}
{
\newrgbcolor{curcolor}{0 0 0}
\pscustom[linestyle=none,fillstyle=solid,fillcolor=curcolor]
{
\newpath
\moveto(32.82010254,148.288)
\lineto(32.82010254,142)
\lineto(33.82810254,142)
\lineto(33.82810254,145.948)
\curveto(33.82810254,146.86)(34.48810254,147.592)(35.30410254,147.592)
\curveto(36.04810254,147.592)(36.46810254,147.136)(36.46810254,146.332)
\lineto(36.46810254,142)
\lineto(37.47610254,142)
\lineto(37.47610254,145.948)
\curveto(37.47610254,146.86)(38.13610254,147.592)(38.95210254,147.592)
\curveto(39.68410254,147.592)(40.11610254,147.124)(40.11610254,146.332)
\lineto(40.11610254,142)
\lineto(41.12410254,142)
\lineto(41.12410254,146.716)
\curveto(41.12410254,147.844)(40.47610254,148.468)(39.30010254,148.468)
\curveto(38.46010254,148.468)(37.95610254,148.216)(37.36810254,147.508)
\curveto(36.99610254,148.18)(36.49210254,148.468)(35.67610254,148.468)
\curveto(34.83610254,148.468)(34.27210254,148.156)(33.74410254,147.4)
\lineto(33.74410254,148.288)
\closepath
}
}
{
\newrgbcolor{curcolor}{0 0 0}
\pscustom[linestyle=none,fillstyle=solid,fillcolor=curcolor]
{
\newpath
\moveto(45.24008185,148.468)
\curveto(43.46408185,148.468)(42.40808185,147.208)(42.40808185,145.096)
\curveto(42.40808185,142.972)(43.46408185,141.724)(45.25208185,141.724)
\curveto(47.02808185,141.724)(48.09608185,142.984)(48.09608185,145.048)
\curveto(48.09608185,147.232)(47.06408185,148.468)(45.24008185,148.468)
\closepath
\moveto(45.25208185,147.544)
\curveto(46.38008185,147.544)(47.05208185,146.62)(47.05208185,145.06)
\curveto(47.05208185,143.572)(46.35608185,142.648)(45.25208185,142.648)
\curveto(44.13608185,142.648)(43.45208185,143.572)(43.45208185,145.096)
\curveto(43.45208185,146.62)(44.13608185,147.544)(45.25208185,147.544)
\closepath
}
}
{
\newrgbcolor{curcolor}{0 0 0}
\pscustom[linestyle=none,fillstyle=solid,fillcolor=curcolor]
{
\newpath
\moveto(51.85206024,142)
\lineto(54.26406024,148.288)
\lineto(53.13606024,148.288)
\lineto(51.36006024,143.188)
\lineto(49.68006024,148.288)
\lineto(48.55206024,148.288)
\lineto(50.76006024,142)
\closepath
}
}
{
\newrgbcolor{curcolor}{0 0 0}
\pscustom[linestyle=none,fillstyle=solid,fillcolor=curcolor]
{
\newpath
\moveto(60.34803186,144.808)
\curveto(60.34803186,145.768)(60.27603186,146.344)(60.09603186,146.812)
\curveto(59.68803186,147.844)(58.72803186,148.468)(57.55203186,148.468)
\curveto(55.80003186,148.468)(54.67203186,147.136)(54.67203186,145.06)
\curveto(54.67203186,142.984)(55.75203186,141.724)(57.52803186,141.724)
\curveto(58.96803186,141.724)(59.96403186,142.54)(60.21603186,143.908)
\lineto(59.20803186,143.908)
\curveto(58.93203186,143.08)(58.36803186,142.648)(57.56403186,142.648)
\curveto(56.92803186,142.648)(56.38803186,142.936)(56.05203186,143.464)
\curveto(55.81203186,143.824)(55.72803186,144.184)(55.71603186,144.808)
\closepath
\moveto(55.74003186,145.624)
\curveto(55.82403186,146.788)(56.53203186,147.544)(57.54003186,147.544)
\curveto(58.56003186,147.544)(59.26803186,146.752)(59.26803186,145.624)
\closepath
}
}
{
\newrgbcolor{curcolor}{0 0 0}
\pscustom[linestyle=none,fillstyle=solid,fillcolor=curcolor]
{
\newpath
\moveto(64.84802747,139.384)
\lineto(65.85602747,139.384)
\lineto(65.85602747,142.66)
\curveto(66.38402747,142.012)(66.97202747,141.724)(67.78802747,141.724)
\curveto(69.42002747,141.724)(70.47602747,143.032)(70.47602747,145.036)
\curveto(70.47602747,147.148)(69.44402747,148.468)(67.77602747,148.468)
\curveto(66.92402747,148.468)(66.24002747,148.084)(65.77202747,147.34)
\lineto(65.77202747,148.288)
\lineto(64.84802747,148.288)
\closepath
\moveto(67.60802747,147.532)
\curveto(68.71202747,147.532)(69.43202747,146.56)(69.43202747,145.06)
\curveto(69.43202747,143.632)(68.70002747,142.66)(67.60802747,142.66)
\curveto(66.54002747,142.66)(65.85602747,143.62)(65.85602747,145.096)
\curveto(65.85602747,146.572)(66.54002747,147.532)(67.60802747,147.532)
\closepath
}
}
{
\newrgbcolor{curcolor}{0 0 0}
\pscustom[linestyle=none,fillstyle=solid,fillcolor=curcolor]
{
\newpath
\moveto(74.13602454,148.468)
\curveto(72.36002454,148.468)(71.30402454,147.208)(71.30402454,145.096)
\curveto(71.30402454,142.972)(72.36002454,141.724)(74.14802454,141.724)
\curveto(75.92402454,141.724)(76.99202454,142.984)(76.99202454,145.048)
\curveto(76.99202454,147.232)(75.96002454,148.468)(74.13602454,148.468)
\closepath
\moveto(74.14802454,147.544)
\curveto(75.27602454,147.544)(75.94802454,146.62)(75.94802454,145.06)
\curveto(75.94802454,143.572)(75.25202454,142.648)(74.14802454,142.648)
\curveto(73.03202454,142.648)(72.34802454,143.572)(72.34802454,145.096)
\curveto(72.34802454,146.62)(73.03202454,147.544)(74.14802454,147.544)
\closepath
}
}
{
\newrgbcolor{curcolor}{0 0 0}
\pscustom[linestyle=none,fillstyle=solid,fillcolor=curcolor]
{
\newpath
\moveto(79.34402161,148.288)
\lineto(78.34802161,148.288)
\lineto(78.34802161,142)
\lineto(79.34402161,142)
\closepath
\moveto(79.34402161,150.748)
\lineto(78.33602161,150.748)
\lineto(78.33602161,149.488)
\lineto(79.34402161,149.488)
\closepath
}
}
{
\newrgbcolor{curcolor}{0 0 0}
\pscustom[linestyle=none,fillstyle=solid,fillcolor=curcolor]
{
\newpath
\moveto(81.04800018,148.288)
\lineto(81.04800018,142)
\lineto(82.05600018,142)
\lineto(82.05600018,145.468)
\curveto(82.05600018,146.752)(82.72800018,147.592)(83.76000018,147.592)
\curveto(84.55200018,147.592)(85.05600018,147.112)(85.05600018,146.356)
\lineto(85.05600018,142)
\lineto(86.05200018,142)
\lineto(86.05200018,146.752)
\curveto(86.05200018,147.796)(85.27200018,148.468)(84.06000018,148.468)
\curveto(83.12400018,148.468)(82.52400018,148.108)(81.97200018,147.232)
\lineto(81.97200018,148.288)
\closepath
}
}
{
\newrgbcolor{curcolor}{0 0 0}
\pscustom[linestyle=none,fillstyle=solid,fillcolor=curcolor]
{
\newpath
\moveto(89.92799725,148.288)
\lineto(88.89599725,148.288)
\lineto(88.89599725,150.016)
\lineto(87.89999725,150.016)
\lineto(87.89999725,148.288)
\lineto(87.04799725,148.288)
\lineto(87.04799725,147.472)
\lineto(87.89999725,147.472)
\lineto(87.89999725,142.72)
\curveto(87.89999725,142.072)(88.33199725,141.724)(89.11199725,141.724)
\curveto(89.37599725,141.724)(89.59199725,141.748)(89.92799725,141.808)
\lineto(89.92799725,142.648)
\curveto(89.78399725,142.612)(89.65199725,142.6)(89.44799725,142.6)
\curveto(89.01599725,142.6)(88.89599725,142.72)(88.89599725,143.164)
\lineto(88.89599725,147.472)
\lineto(89.92799725,147.472)
\closepath
}
}
{
\newrgbcolor{curcolor}{0 0 0}
\pscustom[linestyle=none,fillstyle=solid,fillcolor=curcolor]
{
\newpath
\moveto(95.47199579,146.536)
\curveto(95.45999579,147.772)(94.64399579,148.468)(93.19199579,148.468)
\curveto(91.72799579,148.468)(90.77999579,147.712)(90.77999579,146.548)
\curveto(90.77999579,145.564)(91.28399579,145.096)(92.77199579,144.736)
\lineto(93.70799579,144.508)
\curveto(94.40399579,144.34)(94.67999579,144.088)(94.67999579,143.644)
\curveto(94.67999579,143.044)(94.09199579,142.648)(93.21599579,142.648)
\curveto(92.67599579,142.648)(92.21999579,142.804)(91.96799579,143.068)
\curveto(91.81199579,143.248)(91.73999579,143.428)(91.67999579,143.872)
\lineto(90.62399579,143.872)
\curveto(90.67199579,142.42)(91.48799579,141.724)(93.13199579,141.724)
\curveto(94.71599579,141.724)(95.72399579,142.504)(95.72399579,143.716)
\curveto(95.72399579,144.652)(95.19599579,145.168)(93.94799579,145.468)
\lineto(92.98799579,145.696)
\curveto(92.17199579,145.888)(91.82399579,146.152)(91.82399579,146.596)
\curveto(91.82399579,147.184)(92.33999579,147.544)(93.15599579,147.544)
\curveto(93.95999579,147.544)(94.39199579,147.196)(94.41599579,146.536)
\closepath
}
}
{
\newrgbcolor{curcolor}{0 0 0}
\pscustom[linestyle=none,fillstyle=solid,fillcolor=curcolor]
{
\newpath
\moveto(106.19997144,142)
\lineto(108.04797144,148.288)
\lineto(106.91997144,148.288)
\lineto(105.67197144,143.392)
\lineto(104.43597144,148.288)
\lineto(103.21197144,148.288)
\lineto(102.01197144,143.392)
\lineto(100.72797144,148.288)
\lineto(99.62397144,148.288)
\lineto(101.44797144,142)
\lineto(102.57597144,142)
\lineto(103.78797144,146.932)
\lineto(105.05997144,142)
\closepath
}
}
{
\newrgbcolor{curcolor}{0 0 0}
\pscustom[linestyle=none,fillstyle=solid,fillcolor=curcolor]
{
\newpath
\moveto(110.01595001,148.288)
\lineto(109.01995001,148.288)
\lineto(109.01995001,142)
\lineto(110.01595001,142)
\closepath
\moveto(110.01595001,150.748)
\lineto(109.00795001,150.748)
\lineto(109.00795001,149.488)
\lineto(110.01595001,149.488)
\closepath
}
}
{
\newrgbcolor{curcolor}{0 0 0}
\pscustom[linestyle=none,fillstyle=solid,fillcolor=curcolor]
{
\newpath
\moveto(113.92792859,148.288)
\lineto(112.89592859,148.288)
\lineto(112.89592859,150.016)
\lineto(111.89992859,150.016)
\lineto(111.89992859,148.288)
\lineto(111.04792859,148.288)
\lineto(111.04792859,147.472)
\lineto(111.89992859,147.472)
\lineto(111.89992859,142.72)
\curveto(111.89992859,142.072)(112.33192859,141.724)(113.11192859,141.724)
\curveto(113.37592859,141.724)(113.59192859,141.748)(113.92792859,141.808)
\lineto(113.92792859,142.648)
\curveto(113.78392859,142.612)(113.65192859,142.6)(113.44792859,142.6)
\curveto(113.01592859,142.6)(112.89592859,142.72)(112.89592859,143.164)
\lineto(112.89592859,147.472)
\lineto(113.92792859,147.472)
\closepath
}
}
{
\newrgbcolor{curcolor}{0 0 0}
\pscustom[linestyle=none,fillstyle=solid,fillcolor=curcolor]
{
\newpath
\moveto(114.9839209,150.748)
\lineto(114.9839209,142)
\lineto(115.9799209,142)
\lineto(115.9799209,145.468)
\curveto(115.9799209,146.752)(116.6519209,147.592)(117.6839209,147.592)
\curveto(118.0199209,147.592)(118.3319209,147.496)(118.5719209,147.316)
\curveto(118.8599209,147.1)(118.9799209,146.8)(118.9799209,146.356)
\lineto(118.9799209,142)
\lineto(119.9759209,142)
\lineto(119.9759209,146.752)
\curveto(119.9759209,147.808)(119.2199209,148.468)(117.9959209,148.468)
\curveto(117.1079209,148.468)(116.5679209,148.192)(115.9799209,147.424)
\lineto(115.9799209,150.748)
\closepath
}
}
{
\newrgbcolor{curcolor}{0 0 0}
\pscustom[linestyle=none,fillstyle=solid,fillcolor=curcolor]
{
\newpath
\moveto(122.61591797,148.288)
\lineto(121.61991797,148.288)
\lineto(121.61991797,142)
\lineto(122.61591797,142)
\closepath
\moveto(122.61591797,150.748)
\lineto(121.60791797,150.748)
\lineto(121.60791797,149.488)
\lineto(122.61591797,149.488)
\closepath
}
}
{
\newrgbcolor{curcolor}{0 0 0}
\pscustom[linestyle=none,fillstyle=solid,fillcolor=curcolor]
{
\newpath
\moveto(124.31989655,148.288)
\lineto(124.31989655,142)
\lineto(125.32789655,142)
\lineto(125.32789655,145.468)
\curveto(125.32789655,146.752)(125.99989655,147.592)(127.03189655,147.592)
\curveto(127.82389655,147.592)(128.32789655,147.112)(128.32789655,146.356)
\lineto(128.32789655,142)
\lineto(129.32389655,142)
\lineto(129.32389655,146.752)
\curveto(129.32389655,147.796)(128.54389655,148.468)(127.33189655,148.468)
\curveto(126.39589655,148.468)(125.79589655,148.108)(125.24389655,147.232)
\lineto(125.24389655,148.288)
\closepath
}
}
{
\newrgbcolor{curcolor}{0 0 0}
\pscustom[linestyle=none,fillstyle=solid,fillcolor=curcolor]
{
\newpath
\moveto(136.75189215,148.468)
\curveto(134.97589215,148.468)(133.91989215,147.208)(133.91989215,145.096)
\curveto(133.91989215,142.972)(134.97589215,141.724)(136.76389215,141.724)
\curveto(138.53989215,141.724)(139.60789215,142.984)(139.60789215,145.048)
\curveto(139.60789215,147.232)(138.57589215,148.468)(136.75189215,148.468)
\closepath
\moveto(136.76389215,147.544)
\curveto(137.89189215,147.544)(138.56389215,146.62)(138.56389215,145.06)
\curveto(138.56389215,143.572)(137.86789215,142.648)(136.76389215,142.648)
\curveto(135.64789215,142.648)(134.96389215,143.572)(134.96389215,145.096)
\curveto(134.96389215,146.62)(135.64789215,147.544)(136.76389215,147.544)
\closepath
}
}
{
\newrgbcolor{curcolor}{0 0 0}
\pscustom[linestyle=none,fillstyle=solid,fillcolor=curcolor]
{
\newpath
\moveto(140.99988922,148.288)
\lineto(140.99988922,142)
\lineto(142.00788922,142)
\lineto(142.00788922,145.468)
\curveto(142.00788922,146.752)(142.67988922,147.592)(143.71188922,147.592)
\curveto(144.50388922,147.592)(145.00788922,147.112)(145.00788922,146.356)
\lineto(145.00788922,142)
\lineto(146.00388922,142)
\lineto(146.00388922,146.752)
\curveto(146.00388922,147.796)(145.22388922,148.468)(144.01188922,148.468)
\curveto(143.07588922,148.468)(142.47588922,148.108)(141.92388922,147.232)
\lineto(141.92388922,148.288)
\closepath
}
}
{
\newrgbcolor{curcolor}{0 0 0}
\pscustom[linestyle=none,fillstyle=solid,fillcolor=curcolor]
{
\newpath
\moveto(152.98787866,144.808)
\curveto(152.98787866,145.768)(152.91587866,146.344)(152.73587866,146.812)
\curveto(152.32787866,147.844)(151.36787866,148.468)(150.19187866,148.468)
\curveto(148.43987866,148.468)(147.31187866,147.136)(147.31187866,145.06)
\curveto(147.31187866,142.984)(148.39187866,141.724)(150.16787866,141.724)
\curveto(151.60787866,141.724)(152.60387866,142.54)(152.85587866,143.908)
\lineto(151.84787866,143.908)
\curveto(151.57187866,143.08)(151.00787866,142.648)(150.20387866,142.648)
\curveto(149.56787866,142.648)(149.02787866,142.936)(148.69187866,143.464)
\curveto(148.45187866,143.824)(148.36787866,144.184)(148.35587866,144.808)
\closepath
\moveto(148.37987866,145.624)
\curveto(148.46387866,146.788)(149.17187866,147.544)(150.17987866,147.544)
\curveto(151.19987866,147.544)(151.90787866,146.752)(151.90787866,145.624)
\closepath
}
}
{
\newrgbcolor{curcolor}{0 0 0}
\pscustom[linestyle=none,fillstyle=solid,fillcolor=curcolor]
{
\newpath
\moveto(155.51987476,143.248)
\lineto(154.27187476,143.248)
\lineto(154.27187476,142)
\lineto(155.51987476,142)
\closepath
}
}
{
\newrgbcolor{curcolor}{0 0 0}
\pscustom[linestyle=none,fillstyle=solid,fillcolor=curcolor]
{
\newpath
\moveto(158.85587329,143.248)
\lineto(157.60787329,143.248)
\lineto(157.60787329,142)
\lineto(158.85587329,142)
\closepath
}
}
{
\newrgbcolor{curcolor}{0 0 0}
\pscustom[linestyle=none,fillstyle=solid,fillcolor=curcolor]
{
\newpath
\moveto(162.19187183,143.248)
\lineto(160.94387183,143.248)
\lineto(160.94387183,142)
\lineto(162.19187183,142)
\closepath
}
}
{
\newrgbcolor{curcolor}{0 0 0}
\pscustom[linewidth=1,linecolor=curcolor]
{
\newpath
\moveto(295.5,100.5)
\lineto(295.5,80.5)
\lineto(217.39,80.5)
\lineto(217.42,71.41)
}
}
{
\newrgbcolor{curcolor}{0 0 0}
\pscustom[linestyle=none,fillstyle=solid,fillcolor=curcolor]
{
\newpath
\moveto(217.44,66.16)
\lineto(213.91,73.15)
\lineto(217.42,71.41)
\lineto(220.91,73.18)
\closepath
}
}
{
\newrgbcolor{curcolor}{0 0 0}
\pscustom[linewidth=1,linecolor=curcolor]
{
\newpath
\moveto(217.44,66.16)
\lineto(213.91,73.15)
\lineto(217.42,71.41)
\lineto(220.91,73.18)
\closepath
}
}
{
\newrgbcolor{curcolor}{1 1 1}
\pscustom[linestyle=none,fillstyle=solid,fillcolor=curcolor]
{
\newpath
\moveto(200.5,190.5)
\lineto(390.5,190.5)
\lineto(390.5,100.5)
\lineto(200.5,100.5)
\closepath
}
}
{
\newrgbcolor{curcolor}{0 0 0}
\pscustom[linewidth=1,linecolor=curcolor]
{
\newpath
\moveto(200.5,190.5)
\lineto(390.5,190.5)
\lineto(390.5,100.5)
\lineto(200.5,100.5)
\closepath
}
}
{
\newrgbcolor{curcolor}{0 0 0}
\pscustom[linestyle=none,fillstyle=solid,fillcolor=curcolor]
{
\newpath
\moveto(210.02217258,145.768)
\lineto(212.90217258,145.768)
\curveto(213.89817258,145.768)(214.34217258,145.288)(214.34217258,144.208)
\lineto(214.33017258,143.428)
\curveto(214.33017258,142.888)(214.42617258,142.36)(214.58217258,142)
\lineto(215.93817258,142)
\lineto(215.93817258,142.276)
\curveto(215.51817258,142.564)(215.43417258,142.876)(215.41017258,144.04)
\curveto(215.39817258,145.48)(215.17017258,145.912)(214.22217258,146.32)
\curveto(215.20617258,146.812)(215.60217258,147.4)(215.60217258,148.408)
\curveto(215.60217258,149.92)(214.66617258,150.748)(212.93817258,150.748)
\lineto(208.90617258,150.748)
\lineto(208.90617258,142)
\lineto(210.02217258,142)
\closepath
\moveto(210.02217258,146.752)
\lineto(210.02217258,149.764)
\lineto(212.72217258,149.764)
\curveto(213.34617258,149.764)(213.70617258,149.668)(213.98217258,149.428)
\curveto(214.28217258,149.176)(214.43817258,148.78)(214.43817258,148.264)
\curveto(214.43817258,147.22)(213.91017258,146.752)(212.72217258,146.752)
\closepath
}
}
{
\newrgbcolor{curcolor}{0 0 0}
\pscustom[linestyle=none,fillstyle=solid,fillcolor=curcolor]
{
\newpath
\moveto(222.49014841,144.808)
\curveto(222.49014841,145.768)(222.41814841,146.344)(222.23814841,146.812)
\curveto(221.83014841,147.844)(220.87014841,148.468)(219.69414841,148.468)
\curveto(217.94214841,148.468)(216.81414841,147.136)(216.81414841,145.06)
\curveto(216.81414841,142.984)(217.89414841,141.724)(219.67014841,141.724)
\curveto(221.11014841,141.724)(222.10614841,142.54)(222.35814841,143.908)
\lineto(221.35014841,143.908)
\curveto(221.07414841,143.08)(220.51014841,142.648)(219.70614841,142.648)
\curveto(219.07014841,142.648)(218.53014841,142.936)(218.19414841,143.464)
\curveto(217.95414841,143.824)(217.87014841,144.184)(217.85814841,144.808)
\closepath
\moveto(217.88214841,145.624)
\curveto(217.96614841,146.788)(218.67414841,147.544)(219.68214841,147.544)
\curveto(220.70214841,147.544)(221.41014841,146.752)(221.41014841,145.624)
\closepath
}
}
{
\newrgbcolor{curcolor}{0 0 0}
\pscustom[linestyle=none,fillstyle=solid,fillcolor=curcolor]
{
\newpath
\moveto(223.84614548,148.288)
\lineto(223.84614548,142)
\lineto(224.85414548,142)
\lineto(224.85414548,145.948)
\curveto(224.85414548,146.86)(225.51414548,147.592)(226.33014548,147.592)
\curveto(227.07414548,147.592)(227.49414548,147.136)(227.49414548,146.332)
\lineto(227.49414548,142)
\lineto(228.50214548,142)
\lineto(228.50214548,145.948)
\curveto(228.50214548,146.86)(229.16214548,147.592)(229.97814548,147.592)
\curveto(230.71014548,147.592)(231.14214548,147.124)(231.14214548,146.332)
\lineto(231.14214548,142)
\lineto(232.15014548,142)
\lineto(232.15014548,146.716)
\curveto(232.15014548,147.844)(231.50214548,148.468)(230.32614548,148.468)
\curveto(229.48614548,148.468)(228.98214548,148.216)(228.39414548,147.508)
\curveto(228.02214548,148.18)(227.51814548,148.468)(226.70214548,148.468)
\curveto(225.86214548,148.468)(225.29814548,148.156)(224.77014548,147.4)
\lineto(224.77014548,148.288)
\closepath
}
}
{
\newrgbcolor{curcolor}{0 0 0}
\pscustom[linestyle=none,fillstyle=solid,fillcolor=curcolor]
{
\newpath
\moveto(236.26612479,148.468)
\curveto(234.49012479,148.468)(233.43412479,147.208)(233.43412479,145.096)
\curveto(233.43412479,142.972)(234.49012479,141.724)(236.27812479,141.724)
\curveto(238.05412479,141.724)(239.12212479,142.984)(239.12212479,145.048)
\curveto(239.12212479,147.232)(238.09012479,148.468)(236.26612479,148.468)
\closepath
\moveto(236.27812479,147.544)
\curveto(237.40612479,147.544)(238.07812479,146.62)(238.07812479,145.06)
\curveto(238.07812479,143.572)(237.38212479,142.648)(236.27812479,142.648)
\curveto(235.16212479,142.648)(234.47812479,143.572)(234.47812479,145.096)
\curveto(234.47812479,146.62)(235.16212479,147.544)(236.27812479,147.544)
\closepath
}
}
{
\newrgbcolor{curcolor}{0 0 0}
\pscustom[linestyle=none,fillstyle=solid,fillcolor=curcolor]
{
\newpath
\moveto(242.87810318,142)
\lineto(245.29010318,148.288)
\lineto(244.16210318,148.288)
\lineto(242.38610318,143.188)
\lineto(240.70610318,148.288)
\lineto(239.57810318,148.288)
\lineto(241.78610318,142)
\closepath
}
}
{
\newrgbcolor{curcolor}{0 0 0}
\pscustom[linestyle=none,fillstyle=solid,fillcolor=curcolor]
{
\newpath
\moveto(251.3740748,144.808)
\curveto(251.3740748,145.768)(251.3020748,146.344)(251.1220748,146.812)
\curveto(250.7140748,147.844)(249.7540748,148.468)(248.5780748,148.468)
\curveto(246.8260748,148.468)(245.6980748,147.136)(245.6980748,145.06)
\curveto(245.6980748,142.984)(246.7780748,141.724)(248.5540748,141.724)
\curveto(249.9940748,141.724)(250.9900748,142.54)(251.2420748,143.908)
\lineto(250.2340748,143.908)
\curveto(249.9580748,143.08)(249.3940748,142.648)(248.5900748,142.648)
\curveto(247.9540748,142.648)(247.4140748,142.936)(247.0780748,143.464)
\curveto(246.8380748,143.824)(246.7540748,144.184)(246.7420748,144.808)
\closepath
\moveto(246.7660748,145.624)
\curveto(246.8500748,146.788)(247.5580748,147.544)(248.5660748,147.544)
\curveto(249.5860748,147.544)(250.2940748,146.752)(250.2940748,145.624)
\closepath
}
}
{
\newrgbcolor{curcolor}{0 0 0}
\pscustom[linestyle=none,fillstyle=solid,fillcolor=curcolor]
{
\newpath
\moveto(261.6460704,142.588)
\curveto(261.5380704,142.564)(261.4900704,142.564)(261.4300704,142.564)
\curveto(261.0820704,142.564)(260.8900704,142.744)(260.8900704,143.056)
\lineto(260.8900704,146.752)
\curveto(260.8900704,147.868)(260.0740704,148.468)(258.5260704,148.468)
\curveto(257.6020704,148.468)(256.8700704,148.204)(256.4380704,147.736)
\curveto(256.1500704,147.412)(256.0300704,147.052)(256.0060704,146.428)
\lineto(257.0140704,146.428)
\curveto(257.0980704,147.196)(257.5540704,147.544)(258.4900704,147.544)
\curveto(259.4020704,147.544)(259.8940704,147.208)(259.8940704,146.608)
\lineto(259.8940704,146.344)
\curveto(259.8820704,145.912)(259.6660704,145.756)(258.8500704,145.648)
\curveto(257.4340704,145.468)(257.2180704,145.42)(256.8340704,145.264)
\curveto(256.1020704,144.952)(255.7300704,144.4)(255.7300704,143.584)
\curveto(255.7300704,142.444)(256.5220704,141.724)(257.7940704,141.724)
\curveto(258.5860704,141.724)(259.2220704,142)(259.9300704,142.648)
\curveto(260.0020704,142)(260.3140704,141.724)(260.9620704,141.724)
\curveto(261.1780704,141.724)(261.3100704,141.748)(261.6460704,141.832)
\closepath
\moveto(259.8940704,143.98)
\curveto(259.8940704,143.644)(259.7980704,143.44)(259.4980704,143.164)
\curveto(259.0900704,142.792)(258.5980704,142.6)(258.0100704,142.6)
\curveto(257.2300704,142.6)(256.7740704,142.972)(256.7740704,143.608)
\curveto(256.7740704,144.268)(257.2060704,144.604)(258.2860704,144.76)
\curveto(259.3540704,144.904)(259.5580704,144.952)(259.8940704,145.108)
\closepath
}
}
{
\newrgbcolor{curcolor}{0 0 0}
\pscustom[linestyle=none,fillstyle=solid,fillcolor=curcolor]
{
\newpath
\moveto(263.72206747,150.748)
\lineto(262.71406747,150.748)
\lineto(262.71406747,142)
\lineto(263.72206747,142)
\closepath
}
}
{
\newrgbcolor{curcolor}{0 0 0}
\pscustom[linestyle=none,fillstyle=solid,fillcolor=curcolor]
{
\newpath
\moveto(266.38604605,150.748)
\lineto(265.37804605,150.748)
\lineto(265.37804605,142)
\lineto(266.38604605,142)
\closepath
}
}
{
\newrgbcolor{curcolor}{0 0 0}
\pscustom[linestyle=none,fillstyle=solid,fillcolor=curcolor]
{
\newpath
\moveto(271.21002316,139.384)
\lineto(272.21802316,139.384)
\lineto(272.21802316,142.66)
\curveto(272.74602316,142.012)(273.33402316,141.724)(274.15002316,141.724)
\curveto(275.78202316,141.724)(276.83802316,143.032)(276.83802316,145.036)
\curveto(276.83802316,147.148)(275.80602316,148.468)(274.13802316,148.468)
\curveto(273.28602316,148.468)(272.60202316,148.084)(272.13402316,147.34)
\lineto(272.13402316,148.288)
\lineto(271.21002316,148.288)
\closepath
\moveto(273.97002316,147.532)
\curveto(275.07402316,147.532)(275.79402316,146.56)(275.79402316,145.06)
\curveto(275.79402316,143.632)(275.06202316,142.66)(273.97002316,142.66)
\curveto(272.90202316,142.66)(272.21802316,143.62)(272.21802316,145.096)
\curveto(272.21802316,146.572)(272.90202316,147.532)(273.97002316,147.532)
\closepath
}
}
{
\newrgbcolor{curcolor}{0 0 0}
\pscustom[linestyle=none,fillstyle=solid,fillcolor=curcolor]
{
\newpath
\moveto(280.49802023,148.468)
\curveto(278.72202023,148.468)(277.66602023,147.208)(277.66602023,145.096)
\curveto(277.66602023,142.972)(278.72202023,141.724)(280.51002023,141.724)
\curveto(282.28602023,141.724)(283.35402023,142.984)(283.35402023,145.048)
\curveto(283.35402023,147.232)(282.32202023,148.468)(280.49802023,148.468)
\closepath
\moveto(280.51002023,147.544)
\curveto(281.63802023,147.544)(282.31002023,146.62)(282.31002023,145.06)
\curveto(282.31002023,143.572)(281.61402023,142.648)(280.51002023,142.648)
\curveto(279.39402023,142.648)(278.71002023,143.572)(278.71002023,145.096)
\curveto(278.71002023,146.62)(279.39402023,147.544)(280.51002023,147.544)
\closepath
}
}
{
\newrgbcolor{curcolor}{0 0 0}
\pscustom[linestyle=none,fillstyle=solid,fillcolor=curcolor]
{
\newpath
\moveto(285.7060173,148.288)
\lineto(284.7100173,148.288)
\lineto(284.7100173,142)
\lineto(285.7060173,142)
\closepath
\moveto(285.7060173,150.748)
\lineto(284.6980173,150.748)
\lineto(284.6980173,149.488)
\lineto(285.7060173,149.488)
\closepath
}
}
{
\newrgbcolor{curcolor}{0 0 0}
\pscustom[linestyle=none,fillstyle=solid,fillcolor=curcolor]
{
\newpath
\moveto(287.40999588,148.288)
\lineto(287.40999588,142)
\lineto(288.41799588,142)
\lineto(288.41799588,145.468)
\curveto(288.41799588,146.752)(289.08999588,147.592)(290.12199588,147.592)
\curveto(290.91399588,147.592)(291.41799588,147.112)(291.41799588,146.356)
\lineto(291.41799588,142)
\lineto(292.41399588,142)
\lineto(292.41399588,146.752)
\curveto(292.41399588,147.796)(291.63399588,148.468)(290.42199588,148.468)
\curveto(289.48599588,148.468)(288.88599588,148.108)(288.33399588,147.232)
\lineto(288.33399588,148.288)
\closepath
}
}
{
\newrgbcolor{curcolor}{0 0 0}
\pscustom[linestyle=none,fillstyle=solid,fillcolor=curcolor]
{
\newpath
\moveto(296.28999295,148.288)
\lineto(295.25799295,148.288)
\lineto(295.25799295,150.016)
\lineto(294.26199295,150.016)
\lineto(294.26199295,148.288)
\lineto(293.40999295,148.288)
\lineto(293.40999295,147.472)
\lineto(294.26199295,147.472)
\lineto(294.26199295,142.72)
\curveto(294.26199295,142.072)(294.69399295,141.724)(295.47399295,141.724)
\curveto(295.73799295,141.724)(295.95399295,141.748)(296.28999295,141.808)
\lineto(296.28999295,142.648)
\curveto(296.14599295,142.612)(296.01399295,142.6)(295.80999295,142.6)
\curveto(295.37799295,142.6)(295.25799295,142.72)(295.25799295,143.164)
\lineto(295.25799295,147.472)
\lineto(296.28999295,147.472)
\closepath
}
}
{
\newrgbcolor{curcolor}{0 0 0}
\pscustom[linestyle=none,fillstyle=solid,fillcolor=curcolor]
{
\newpath
\moveto(301.83399149,146.536)
\curveto(301.82199149,147.772)(301.00599149,148.468)(299.55399149,148.468)
\curveto(298.08999149,148.468)(297.14199149,147.712)(297.14199149,146.548)
\curveto(297.14199149,145.564)(297.64599149,145.096)(299.13399149,144.736)
\lineto(300.06999149,144.508)
\curveto(300.76599149,144.34)(301.04199149,144.088)(301.04199149,143.644)
\curveto(301.04199149,143.044)(300.45399149,142.648)(299.57799149,142.648)
\curveto(299.03799149,142.648)(298.58199149,142.804)(298.32999149,143.068)
\curveto(298.17399149,143.248)(298.10199149,143.428)(298.04199149,143.872)
\lineto(296.98599149,143.872)
\curveto(297.03399149,142.42)(297.84999149,141.724)(299.49399149,141.724)
\curveto(301.07799149,141.724)(302.08599149,142.504)(302.08599149,143.716)
\curveto(302.08599149,144.652)(301.55799149,145.168)(300.30999149,145.468)
\lineto(299.34999149,145.696)
\curveto(298.53399149,145.888)(298.18599149,146.152)(298.18599149,146.596)
\curveto(298.18599149,147.184)(298.70199149,147.544)(299.51799149,147.544)
\curveto(300.32199149,147.544)(300.75399149,147.196)(300.77799149,146.536)
\closepath
}
}
{
\newrgbcolor{curcolor}{0 0 0}
\pscustom[linestyle=none,fillstyle=solid,fillcolor=curcolor]
{
\newpath
\moveto(308.96196713,148.288)
\lineto(307.92996713,148.288)
\lineto(307.92996713,150.016)
\lineto(306.93396713,150.016)
\lineto(306.93396713,148.288)
\lineto(306.08196713,148.288)
\lineto(306.08196713,147.472)
\lineto(306.93396713,147.472)
\lineto(306.93396713,142.72)
\curveto(306.93396713,142.072)(307.36596713,141.724)(308.14596713,141.724)
\curveto(308.40996713,141.724)(308.62596713,141.748)(308.96196713,141.808)
\lineto(308.96196713,142.648)
\curveto(308.81796713,142.612)(308.68596713,142.6)(308.48196713,142.6)
\curveto(308.04996713,142.6)(307.92996713,142.72)(307.92996713,143.164)
\lineto(307.92996713,147.472)
\lineto(308.96196713,147.472)
\closepath
}
}
{
\newrgbcolor{curcolor}{0 0 0}
\pscustom[linestyle=none,fillstyle=solid,fillcolor=curcolor]
{
\newpath
\moveto(310.01795944,150.748)
\lineto(310.01795944,142)
\lineto(311.01395944,142)
\lineto(311.01395944,145.468)
\curveto(311.01395944,146.752)(311.68595944,147.592)(312.71795944,147.592)
\curveto(313.05395944,147.592)(313.36595944,147.496)(313.60595944,147.316)
\curveto(313.89395944,147.1)(314.01395944,146.8)(314.01395944,146.356)
\lineto(314.01395944,142)
\lineto(315.00995944,142)
\lineto(315.00995944,146.752)
\curveto(315.00995944,147.808)(314.25395944,148.468)(313.02995944,148.468)
\curveto(312.14195944,148.468)(311.60195944,148.192)(311.01395944,147.424)
\lineto(311.01395944,150.748)
\closepath
}
}
{
\newrgbcolor{curcolor}{0 0 0}
\pscustom[linestyle=none,fillstyle=solid,fillcolor=curcolor]
{
\newpath
\moveto(322.26995651,142.588)
\curveto(322.16195651,142.564)(322.11395651,142.564)(322.05395651,142.564)
\curveto(321.70595651,142.564)(321.51395651,142.744)(321.51395651,143.056)
\lineto(321.51395651,146.752)
\curveto(321.51395651,147.868)(320.69795651,148.468)(319.14995651,148.468)
\curveto(318.22595651,148.468)(317.49395651,148.204)(317.06195651,147.736)
\curveto(316.77395651,147.412)(316.65395651,147.052)(316.62995651,146.428)
\lineto(317.63795651,146.428)
\curveto(317.72195651,147.196)(318.17795651,147.544)(319.11395651,147.544)
\curveto(320.02595651,147.544)(320.51795651,147.208)(320.51795651,146.608)
\lineto(320.51795651,146.344)
\curveto(320.50595651,145.912)(320.28995651,145.756)(319.47395651,145.648)
\curveto(318.05795651,145.468)(317.84195651,145.42)(317.45795651,145.264)
\curveto(316.72595651,144.952)(316.35395651,144.4)(316.35395651,143.584)
\curveto(316.35395651,142.444)(317.14595651,141.724)(318.41795651,141.724)
\curveto(319.20995651,141.724)(319.84595651,142)(320.55395651,142.648)
\curveto(320.62595651,142)(320.93795651,141.724)(321.58595651,141.724)
\curveto(321.80195651,141.724)(321.93395651,141.748)(322.26995651,141.832)
\closepath
\moveto(320.51795651,143.98)
\curveto(320.51795651,143.644)(320.42195651,143.44)(320.12195651,143.164)
\curveto(319.71395651,142.792)(319.22195651,142.6)(318.63395651,142.6)
\curveto(317.85395651,142.6)(317.39795651,142.972)(317.39795651,143.608)
\curveto(317.39795651,144.268)(317.82995651,144.604)(318.90995651,144.76)
\curveto(319.97795651,144.904)(320.18195651,144.952)(320.51795651,145.108)
\closepath
}
}
{
\newrgbcolor{curcolor}{0 0 0}
\pscustom[linestyle=none,fillstyle=solid,fillcolor=curcolor]
{
\newpath
\moveto(325.43793454,148.288)
\lineto(324.40593454,148.288)
\lineto(324.40593454,150.016)
\lineto(323.40993454,150.016)
\lineto(323.40993454,148.288)
\lineto(322.55793454,148.288)
\lineto(322.55793454,147.472)
\lineto(323.40993454,147.472)
\lineto(323.40993454,142.72)
\curveto(323.40993454,142.072)(323.84193454,141.724)(324.62193454,141.724)
\curveto(324.88593454,141.724)(325.10193454,141.748)(325.43793454,141.808)
\lineto(325.43793454,142.648)
\curveto(325.29393454,142.612)(325.16193454,142.6)(324.95793454,142.6)
\curveto(324.52593454,142.6)(324.40593454,142.72)(324.40593454,143.164)
\lineto(324.40593454,147.472)
\lineto(325.43793454,147.472)
\closepath
}
}
{
\newrgbcolor{curcolor}{0 0 0}
\pscustom[linestyle=none,fillstyle=solid,fillcolor=curcolor]
{
\newpath
\moveto(335.48193161,142.588)
\curveto(335.37393161,142.564)(335.32593161,142.564)(335.26593161,142.564)
\curveto(334.91793161,142.564)(334.72593161,142.744)(334.72593161,143.056)
\lineto(334.72593161,146.752)
\curveto(334.72593161,147.868)(333.90993161,148.468)(332.36193161,148.468)
\curveto(331.43793161,148.468)(330.70593161,148.204)(330.27393161,147.736)
\curveto(329.98593161,147.412)(329.86593161,147.052)(329.84193161,146.428)
\lineto(330.84993161,146.428)
\curveto(330.93393161,147.196)(331.38993161,147.544)(332.32593161,147.544)
\curveto(333.23793161,147.544)(333.72993161,147.208)(333.72993161,146.608)
\lineto(333.72993161,146.344)
\curveto(333.71793161,145.912)(333.50193161,145.756)(332.68593161,145.648)
\curveto(331.26993161,145.468)(331.05393161,145.42)(330.66993161,145.264)
\curveto(329.93793161,144.952)(329.56593161,144.4)(329.56593161,143.584)
\curveto(329.56593161,142.444)(330.35793161,141.724)(331.62993161,141.724)
\curveto(332.42193161,141.724)(333.05793161,142)(333.76593161,142.648)
\curveto(333.83793161,142)(334.14993161,141.724)(334.79793161,141.724)
\curveto(335.01393161,141.724)(335.14593161,141.748)(335.48193161,141.832)
\closepath
\moveto(333.72993161,143.98)
\curveto(333.72993161,143.644)(333.63393161,143.44)(333.33393161,143.164)
\curveto(332.92593161,142.792)(332.43393161,142.6)(331.84593161,142.6)
\curveto(331.06593161,142.6)(330.60993161,142.972)(330.60993161,143.608)
\curveto(330.60993161,144.268)(331.04193161,144.604)(332.12193161,144.76)
\curveto(333.18993161,144.904)(333.39393161,144.952)(333.72993161,145.108)
\closepath
}
}
{
\newrgbcolor{curcolor}{0 0 0}
\pscustom[linestyle=none,fillstyle=solid,fillcolor=curcolor]
{
\newpath
\moveto(336.56192868,148.288)
\lineto(336.56192868,142)
\lineto(337.56992868,142)
\lineto(337.56992868,145.264)
\curveto(337.58192868,146.776)(338.20592868,147.448)(339.58592868,147.412)
\lineto(339.58592868,148.432)
\curveto(339.41792868,148.456)(339.32192868,148.468)(339.20192868,148.468)
\curveto(338.55392868,148.468)(338.06192868,148.084)(337.48592868,147.148)
\lineto(337.48592868,148.288)
\closepath
}
}
{
\newrgbcolor{curcolor}{0 0 0}
\pscustom[linestyle=none,fillstyle=solid,fillcolor=curcolor]
{
\newpath
\moveto(345.76590524,144.808)
\curveto(345.76590524,145.768)(345.69390524,146.344)(345.51390524,146.812)
\curveto(345.10590524,147.844)(344.14590524,148.468)(342.96990524,148.468)
\curveto(341.21790524,148.468)(340.08990524,147.136)(340.08990524,145.06)
\curveto(340.08990524,142.984)(341.16990524,141.724)(342.94590524,141.724)
\curveto(344.38590524,141.724)(345.38190524,142.54)(345.63390524,143.908)
\lineto(344.62590524,143.908)
\curveto(344.34990524,143.08)(343.78590524,142.648)(342.98190524,142.648)
\curveto(342.34590524,142.648)(341.80590524,142.936)(341.46990524,143.464)
\curveto(341.22990524,143.824)(341.14590524,144.184)(341.13390524,144.808)
\closepath
\moveto(341.15790524,145.624)
\curveto(341.24190524,146.788)(341.94990524,147.544)(342.95790524,147.544)
\curveto(343.97790524,147.544)(344.68590524,146.752)(344.68590524,145.624)
\closepath
}
}
{
\newrgbcolor{curcolor}{0 0 0}
\pscustom[linestyle=none,fillstyle=solid,fillcolor=curcolor]
{
\newpath
\moveto(350.45790085,150.748)
\lineto(350.45790085,142)
\lineto(351.45390085,142)
\lineto(351.45390085,145.468)
\curveto(351.45390085,146.752)(352.12590085,147.592)(353.15790085,147.592)
\curveto(353.49390085,147.592)(353.80590085,147.496)(354.04590085,147.316)
\curveto(354.33390085,147.1)(354.45390085,146.8)(354.45390085,146.356)
\lineto(354.45390085,142)
\lineto(355.44990085,142)
\lineto(355.44990085,146.752)
\curveto(355.44990085,147.808)(354.69390085,148.468)(353.46990085,148.468)
\curveto(352.58190085,148.468)(352.04190085,148.192)(351.45390085,147.424)
\lineto(351.45390085,150.748)
\closepath
}
}
{
\newrgbcolor{curcolor}{0 0 0}
\pscustom[linestyle=none,fillstyle=solid,fillcolor=curcolor]
{
\newpath
\moveto(358.08989792,148.288)
\lineto(357.09389792,148.288)
\lineto(357.09389792,142)
\lineto(358.08989792,142)
\closepath
\moveto(358.08989792,150.748)
\lineto(357.08189792,150.748)
\lineto(357.08189792,149.488)
\lineto(358.08989792,149.488)
\closepath
}
}
{
\newrgbcolor{curcolor}{0 0 0}
\pscustom[linestyle=none,fillstyle=solid,fillcolor=curcolor]
{
\newpath
\moveto(363.7898602,148.288)
\lineto(363.7898602,147.376)
\curveto(363.2858602,148.132)(362.7338602,148.468)(361.9418602,148.468)
\curveto(360.4178602,148.468)(359.3618602,147.052)(359.3618602,145.036)
\curveto(359.3618602,143.98)(359.6138602,143.212)(360.1538602,142.576)
\curveto(360.6218602,142.024)(361.2218602,141.724)(361.8698602,141.724)
\curveto(362.6258602,141.724)(363.1658602,142.06)(363.6938602,142.852)
\lineto(363.6938602,142.528)
\curveto(363.6938602,140.848)(363.2258602,140.224)(361.9778602,140.224)
\curveto(361.1258602,140.224)(360.6818602,140.56)(360.5858602,141.28)
\lineto(359.5658602,141.28)
\curveto(359.6618602,140.116)(360.5858602,139.384)(361.9538602,139.384)
\curveto(362.8778602,139.384)(363.6458602,139.684)(364.0538602,140.188)
\curveto(364.5338602,140.776)(364.7138602,141.556)(364.7138602,143.032)
\lineto(364.7138602,148.288)
\closepath
\moveto(362.0378602,147.544)
\curveto(363.0938602,147.544)(363.6938602,146.656)(363.6938602,145.06)
\curveto(363.6938602,143.536)(363.0818602,142.648)(362.0378602,142.648)
\curveto(361.0058602,142.648)(360.4058602,143.548)(360.4058602,145.096)
\curveto(360.4058602,146.632)(361.0058602,147.544)(362.0378602,147.544)
\closepath
}
}
{
\newrgbcolor{curcolor}{0 0 0}
\pscustom[linestyle=none,fillstyle=solid,fillcolor=curcolor]
{
\newpath
\moveto(366.36982712,150.748)
\lineto(366.36982712,142)
\lineto(367.36582712,142)
\lineto(367.36582712,145.468)
\curveto(367.36582712,146.752)(368.03782712,147.592)(369.06982712,147.592)
\curveto(369.40582712,147.592)(369.71782712,147.496)(369.95782712,147.316)
\curveto(370.24582712,147.1)(370.36582712,146.8)(370.36582712,146.356)
\lineto(370.36582712,142)
\lineto(371.36182712,142)
\lineto(371.36182712,146.752)
\curveto(371.36182712,147.808)(370.60582712,148.468)(369.38182712,148.468)
\curveto(368.49382712,148.468)(367.95382712,148.192)(367.36582712,147.424)
\lineto(367.36582712,150.748)
\closepath
}
}
{
\newrgbcolor{curcolor}{0 0 0}
\pscustom[linestyle=none,fillstyle=solid,fillcolor=curcolor]
{
\newpath
\moveto(374.49382419,143.248)
\lineto(373.24582419,143.248)
\lineto(373.24582419,142)
\lineto(374.49382419,142)
\closepath
}
}
{
\newrgbcolor{curcolor}{0 0 0}
\pscustom[linestyle=none,fillstyle=solid,fillcolor=curcolor]
{
\newpath
\moveto(377.82982272,143.248)
\lineto(376.58182272,143.248)
\lineto(376.58182272,142)
\lineto(377.82982272,142)
\closepath
}
}
{
\newrgbcolor{curcolor}{0 0 0}
\pscustom[linestyle=none,fillstyle=solid,fillcolor=curcolor]
{
\newpath
\moveto(381.16582126,143.248)
\lineto(379.91782126,143.248)
\lineto(379.91782126,142)
\lineto(381.16582126,142)
\closepath
}
}
{
\newrgbcolor{curcolor}{1 1 1}
\pscustom[linestyle=none,fillstyle=solid,fillcolor=curcolor]
{
\newpath
\moveto(130.5,0.5)
\lineto(150.5,65.5)
\lineto(270.5,65.5)
\lineto(250.5,0.5)
\closepath
}
}
{
\newrgbcolor{curcolor}{0 0 0}
\pscustom[linewidth=1,linecolor=curcolor]
{
\newpath
\moveto(130.5,0.5)
\lineto(150.5,65.5)
\lineto(270.5,65.5)
\lineto(250.5,0.5)
\closepath
}
}
{
\newrgbcolor{curcolor}{0 0 0}
\pscustom[linestyle=none,fillstyle=solid,fillcolor=curcolor]
{
\newpath
\moveto(138.3261333,36.18)
\curveto(138.3261333,36.78)(138.2901333,36.948)(138.0981333,37.356)
\curveto(137.6181333,38.364)(136.5981333,38.892)(135.1221333,38.892)
\curveto(133.2021333,38.892)(132.0141333,37.908)(132.0141333,36.324)
\curveto(132.0141333,35.256)(132.5781333,34.584)(133.7301333,34.284)
\lineto(135.9021333,33.708)
\curveto(137.0181333,33.42)(137.5101333,32.976)(137.5101333,32.292)
\curveto(137.5101333,31.824)(137.2581333,31.344)(136.8861333,31.08)
\curveto(136.5381333,30.828)(135.9861333,30.708)(135.2781333,30.708)
\curveto(134.3181333,30.708)(133.6821333,30.936)(133.2621333,31.44)
\curveto(132.9381333,31.824)(132.7941333,32.244)(132.8061333,32.784)
\lineto(131.7501333,32.784)
\curveto(131.7621333,31.98)(131.9181333,31.452)(132.2661333,30.972)
\curveto(132.8661333,30.144)(133.8741333,29.724)(135.2061333,29.724)
\curveto(136.2501333,29.724)(137.1021333,29.964)(137.6661333,30.396)
\curveto(138.2541333,30.864)(138.6261333,31.644)(138.6261333,32.4)
\curveto(138.6261333,33.48)(137.9541333,34.272)(136.7661333,34.596)
\lineto(134.5701333,35.184)
\curveto(133.5141333,35.472)(133.1301333,35.808)(133.1301333,36.48)
\curveto(133.1301333,37.368)(133.9101333,37.956)(135.0861333,37.956)
\curveto(136.4781333,37.956)(137.2581333,37.332)(137.2701333,36.18)
\closepath
}
}
{
\newrgbcolor{curcolor}{0 0 0}
\pscustom[linestyle=none,fillstyle=solid,fillcolor=curcolor]
{
\newpath
\moveto(144.9621311,30)
\lineto(144.9621311,36.288)
\lineto(143.9661311,36.288)
\lineto(143.9661311,32.724)
\curveto(143.9661311,31.44)(143.2941311,30.6)(142.2501311,30.6)
\curveto(141.4581311,30.6)(140.9541311,31.08)(140.9541311,31.836)
\lineto(140.9541311,36.288)
\lineto(139.9581311,36.288)
\lineto(139.9581311,31.44)
\curveto(139.9581311,30.396)(140.7381311,29.724)(141.9621311,29.724)
\curveto(142.8861311,29.724)(143.4741311,30.048)(144.0621311,30.876)
\lineto(144.0621311,30)
\closepath
}
}
{
\newrgbcolor{curcolor}{0 0 0}
\pscustom[linestyle=none,fillstyle=solid,fillcolor=curcolor]
{
\newpath
\moveto(146.49812817,38.748)
\lineto(146.49812817,30)
\lineto(147.39812817,30)
\lineto(147.39812817,30.804)
\curveto(147.87812817,30.072)(148.51412817,29.724)(149.39012817,29.724)
\curveto(151.04612817,29.724)(152.12612817,31.08)(152.12612817,33.168)
\curveto(152.12612817,35.208)(151.10612817,36.468)(149.43812817,36.468)
\curveto(148.57412817,36.468)(147.96212817,36.144)(147.49412817,35.436)
\lineto(147.49412817,38.748)
\closepath
\moveto(149.24612817,35.532)
\curveto(150.36212817,35.532)(151.08212817,34.56)(151.08212817,33.06)
\curveto(151.08212817,31.632)(150.33812817,30.66)(149.24612817,30.66)
\curveto(148.17812817,30.66)(147.49412817,31.62)(147.49412817,33.096)
\curveto(147.49412817,34.572)(148.17812817,35.532)(149.24612817,35.532)
\closepath
}
}
{
\newrgbcolor{curcolor}{0 0 0}
\pscustom[linestyle=none,fillstyle=solid,fillcolor=curcolor]
{
\newpath
\moveto(157.77812524,34.536)
\curveto(157.76612524,35.772)(156.95012524,36.468)(155.49812524,36.468)
\curveto(154.03412524,36.468)(153.08612524,35.712)(153.08612524,34.548)
\curveto(153.08612524,33.564)(153.59012524,33.096)(155.07812524,32.736)
\lineto(156.01412524,32.508)
\curveto(156.71012524,32.34)(156.98612524,32.088)(156.98612524,31.644)
\curveto(156.98612524,31.044)(156.39812524,30.648)(155.52212524,30.648)
\curveto(154.98212524,30.648)(154.52612524,30.804)(154.27412524,31.068)
\curveto(154.11812524,31.248)(154.04612524,31.428)(153.98612524,31.872)
\lineto(152.93012524,31.872)
\curveto(152.97812524,30.42)(153.79412524,29.724)(155.43812524,29.724)
\curveto(157.02212524,29.724)(158.03012524,30.504)(158.03012524,31.716)
\curveto(158.03012524,32.652)(157.50212524,33.168)(156.25412524,33.468)
\lineto(155.29412524,33.696)
\curveto(154.47812524,33.888)(154.13012524,34.152)(154.13012524,34.596)
\curveto(154.13012524,35.184)(154.64612524,35.544)(155.46212524,35.544)
\curveto(156.26612524,35.544)(156.69812524,35.196)(156.72212524,34.536)
\closepath
}
}
{
\newrgbcolor{curcolor}{0 0 0}
\pscustom[linestyle=none,fillstyle=solid,fillcolor=curcolor]
{
\newpath
\moveto(164.30610236,30)
\lineto(164.30610236,36.288)
\lineto(163.31010236,36.288)
\lineto(163.31010236,32.724)
\curveto(163.31010236,31.44)(162.63810236,30.6)(161.59410236,30.6)
\curveto(160.80210236,30.6)(160.29810236,31.08)(160.29810236,31.836)
\lineto(160.29810236,36.288)
\lineto(159.30210236,36.288)
\lineto(159.30210236,31.44)
\curveto(159.30210236,30.396)(160.08210236,29.724)(161.30610236,29.724)
\curveto(162.23010236,29.724)(162.81810236,30.048)(163.40610236,30.876)
\lineto(163.40610236,30)
\closepath
}
}
{
\newrgbcolor{curcolor}{0 0 0}
\pscustom[linestyle=none,fillstyle=solid,fillcolor=curcolor]
{
\newpath
\moveto(166.02209943,36.288)
\lineto(166.02209943,30)
\lineto(167.03009943,30)
\lineto(167.03009943,33.264)
\curveto(167.04209943,34.776)(167.66609943,35.448)(169.04609943,35.412)
\lineto(169.04609943,36.432)
\curveto(168.87809943,36.456)(168.78209943,36.468)(168.66209943,36.468)
\curveto(168.01409943,36.468)(167.52209943,36.084)(166.94609943,35.148)
\lineto(166.94609943,36.288)
\closepath
}
}
{
\newrgbcolor{curcolor}{0 0 0}
\pscustom[linestyle=none,fillstyle=solid,fillcolor=curcolor]
{
\newpath
\moveto(172.58607416,36.288)
\lineto(171.54207416,36.288)
\lineto(171.54207416,37.272)
\curveto(171.54207416,37.692)(171.77007416,37.908)(172.23807416,37.908)
\curveto(172.32207416,37.908)(172.35807416,37.908)(172.58607416,37.896)
\lineto(172.58607416,38.724)
\curveto(172.35807416,38.772)(172.22607416,38.784)(172.02207416,38.784)
\curveto(171.09807416,38.784)(170.54607416,38.256)(170.54607416,37.356)
\lineto(170.54607416,36.288)
\lineto(169.70607416,36.288)
\lineto(169.70607416,35.472)
\lineto(170.54607416,35.472)
\lineto(170.54607416,30)
\lineto(171.54207416,30)
\lineto(171.54207416,35.472)
\lineto(172.58607416,35.472)
\closepath
}
}
{
\newrgbcolor{curcolor}{0 0 0}
\pscustom[linestyle=none,fillstyle=solid,fillcolor=curcolor]
{
\newpath
\moveto(179.15005676,30.588)
\curveto(179.04205676,30.564)(178.99405676,30.564)(178.93405676,30.564)
\curveto(178.58605676,30.564)(178.39405676,30.744)(178.39405676,31.056)
\lineto(178.39405676,34.752)
\curveto(178.39405676,35.868)(177.57805676,36.468)(176.03005676,36.468)
\curveto(175.10605676,36.468)(174.37405676,36.204)(173.94205676,35.736)
\curveto(173.65405676,35.412)(173.53405676,35.052)(173.51005676,34.428)
\lineto(174.51805676,34.428)
\curveto(174.60205676,35.196)(175.05805676,35.544)(175.99405676,35.544)
\curveto(176.90605676,35.544)(177.39805676,35.208)(177.39805676,34.608)
\lineto(177.39805676,34.344)
\curveto(177.38605676,33.912)(177.17005676,33.756)(176.35405676,33.648)
\curveto(174.93805676,33.468)(174.72205676,33.42)(174.33805676,33.264)
\curveto(173.60605676,32.952)(173.23405676,32.4)(173.23405676,31.584)
\curveto(173.23405676,30.444)(174.02605676,29.724)(175.29805676,29.724)
\curveto(176.09005676,29.724)(176.72605676,30)(177.43405676,30.648)
\curveto(177.50605676,30)(177.81805676,29.724)(178.46605676,29.724)
\curveto(178.68205676,29.724)(178.81405676,29.748)(179.15005676,29.832)
\closepath
\moveto(177.39805676,31.98)
\curveto(177.39805676,31.644)(177.30205676,31.44)(177.00205676,31.164)
\curveto(176.59405676,30.792)(176.10205676,30.6)(175.51405676,30.6)
\curveto(174.73405676,30.6)(174.27805676,30.972)(174.27805676,31.608)
\curveto(174.27805676,32.268)(174.71005676,32.604)(175.79005676,32.76)
\curveto(176.85805676,32.904)(177.06205676,32.952)(177.39805676,33.108)
\closepath
}
}
{
\newrgbcolor{curcolor}{0 0 0}
\pscustom[linestyle=none,fillstyle=solid,fillcolor=curcolor]
{
\newpath
\moveto(185.05405383,34.176)
\curveto(185.00605383,34.788)(184.87405383,35.184)(184.63405383,35.532)
\curveto(184.20205383,36.12)(183.44605383,36.468)(182.57005383,36.468)
\curveto(180.86605383,36.468)(179.77405383,35.124)(179.77405383,33.036)
\curveto(179.77405383,31.008)(180.85405383,29.724)(182.55805383,29.724)
\curveto(184.05805383,29.724)(185.00605383,30.624)(185.12605383,32.16)
\lineto(184.11805383,32.16)
\curveto(183.95005383,31.152)(183.43405383,30.648)(182.58205383,30.648)
\curveto(181.47805383,30.648)(180.81805383,31.548)(180.81805383,33.036)
\curveto(180.81805383,34.608)(181.46605383,35.544)(182.55805383,35.544)
\curveto(183.39805383,35.544)(183.92605383,35.052)(184.04605383,34.176)
\closepath
}
}
{
\newrgbcolor{curcolor}{0 0 0}
\pscustom[linestyle=none,fillstyle=solid,fillcolor=curcolor]
{
\newpath
\moveto(191.63001428,32.808)
\curveto(191.63001428,33.768)(191.55801428,34.344)(191.37801428,34.812)
\curveto(190.97001428,35.844)(190.01001428,36.468)(188.83401428,36.468)
\curveto(187.08201428,36.468)(185.95401428,35.136)(185.95401428,33.06)
\curveto(185.95401428,30.984)(187.03401428,29.724)(188.81001428,29.724)
\curveto(190.25001428,29.724)(191.24601428,30.54)(191.49801428,31.908)
\lineto(190.49001428,31.908)
\curveto(190.21401428,31.08)(189.65001428,30.648)(188.84601428,30.648)
\curveto(188.21001428,30.648)(187.67001428,30.936)(187.33401428,31.464)
\curveto(187.09401428,31.824)(187.01001428,32.184)(186.99801428,32.808)
\closepath
\moveto(187.02201428,33.624)
\curveto(187.10601428,34.788)(187.81401428,35.544)(188.82201428,35.544)
\curveto(189.84201428,35.544)(190.55001428,34.752)(190.55001428,33.624)
\closepath
}
}
{
\newrgbcolor{curcolor}{0 0 0}
\pscustom[linestyle=none,fillstyle=solid,fillcolor=curcolor]
{
\newpath
\moveto(196.13000989,27.384)
\lineto(197.13800989,27.384)
\lineto(197.13800989,30.66)
\curveto(197.66600989,30.012)(198.25400989,29.724)(199.07000989,29.724)
\curveto(200.70200989,29.724)(201.75800989,31.032)(201.75800989,33.036)
\curveto(201.75800989,35.148)(200.72600989,36.468)(199.05800989,36.468)
\curveto(198.20600989,36.468)(197.52200989,36.084)(197.05400989,35.34)
\lineto(197.05400989,36.288)
\lineto(196.13000989,36.288)
\closepath
\moveto(198.89000989,35.532)
\curveto(199.99400989,35.532)(200.71400989,34.56)(200.71400989,33.06)
\curveto(200.71400989,31.632)(199.98200989,30.66)(198.89000989,30.66)
\curveto(197.82200989,30.66)(197.13800989,31.62)(197.13800989,33.096)
\curveto(197.13800989,34.572)(197.82200989,35.532)(198.89000989,35.532)
\closepath
}
}
{
\newrgbcolor{curcolor}{0 0 0}
\pscustom[linestyle=none,fillstyle=solid,fillcolor=curcolor]
{
\newpath
\moveto(202.99400696,38.748)
\lineto(202.99400696,30)
\lineto(203.99000696,30)
\lineto(203.99000696,33.468)
\curveto(203.99000696,34.752)(204.66200696,35.592)(205.69400696,35.592)
\curveto(206.03000696,35.592)(206.34200696,35.496)(206.58200696,35.316)
\curveto(206.87000696,35.1)(206.99000696,34.8)(206.99000696,34.356)
\lineto(206.99000696,30)
\lineto(207.98600696,30)
\lineto(207.98600696,34.752)
\curveto(207.98600696,35.808)(207.23000696,36.468)(206.00600696,36.468)
\curveto(205.11800696,36.468)(204.57800696,36.192)(203.99000696,35.424)
\lineto(203.99000696,38.748)
\closepath
}
}
{
\newrgbcolor{curcolor}{0 0 0}
\pscustom[linestyle=none,fillstyle=solid,fillcolor=curcolor]
{
\newpath
\moveto(212.09000403,36.468)
\curveto(210.31400403,36.468)(209.25800403,35.208)(209.25800403,33.096)
\curveto(209.25800403,30.972)(210.31400403,29.724)(212.10200403,29.724)
\curveto(213.87800403,29.724)(214.94600403,30.984)(214.94600403,33.048)
\curveto(214.94600403,35.232)(213.91400403,36.468)(212.09000403,36.468)
\closepath
\moveto(212.10200403,35.544)
\curveto(213.23000403,35.544)(213.90200403,34.62)(213.90200403,33.06)
\curveto(213.90200403,31.572)(213.20600403,30.648)(212.10200403,30.648)
\curveto(210.98600403,30.648)(210.30200403,31.572)(210.30200403,33.096)
\curveto(210.30200403,34.62)(210.98600403,35.544)(212.10200403,35.544)
\closepath
}
}
{
\newrgbcolor{curcolor}{0 0 0}
\pscustom[linestyle=none,fillstyle=solid,fillcolor=curcolor]
{
\newpath
\moveto(218.43799176,36.288)
\lineto(217.40599176,36.288)
\lineto(217.40599176,38.016)
\lineto(216.40999176,38.016)
\lineto(216.40999176,36.288)
\lineto(215.55799176,36.288)
\lineto(215.55799176,35.472)
\lineto(216.40999176,35.472)
\lineto(216.40999176,30.72)
\curveto(216.40999176,30.072)(216.84199176,29.724)(217.62199176,29.724)
\curveto(217.88599176,29.724)(218.10199176,29.748)(218.43799176,29.808)
\lineto(218.43799176,30.648)
\curveto(218.29399176,30.612)(218.16199176,30.6)(217.95799176,30.6)
\curveto(217.52599176,30.6)(217.40599176,30.72)(217.40599176,31.164)
\lineto(217.40599176,35.472)
\lineto(218.43799176,35.472)
\closepath
}
}
{
\newrgbcolor{curcolor}{0 0 0}
\pscustom[linestyle=none,fillstyle=solid,fillcolor=curcolor]
{
\newpath
\moveto(221.84597784,36.468)
\curveto(220.06997784,36.468)(219.01397784,35.208)(219.01397784,33.096)
\curveto(219.01397784,30.972)(220.06997784,29.724)(221.85797784,29.724)
\curveto(223.63397784,29.724)(224.70197784,30.984)(224.70197784,33.048)
\curveto(224.70197784,35.232)(223.66997784,36.468)(221.84597784,36.468)
\closepath
\moveto(221.85797784,35.544)
\curveto(222.98597784,35.544)(223.65797784,34.62)(223.65797784,33.06)
\curveto(223.65797784,31.572)(222.96197784,30.648)(221.85797784,30.648)
\curveto(220.74197784,30.648)(220.05797784,31.572)(220.05797784,33.096)
\curveto(220.05797784,34.62)(220.74197784,35.544)(221.85797784,35.544)
\closepath
}
}
{
\newrgbcolor{curcolor}{0 0 0}
\pscustom[linestyle=none,fillstyle=solid,fillcolor=curcolor]
{
\newpath
\moveto(226.09397491,36.288)
\lineto(226.09397491,30)
\lineto(227.10197491,30)
\lineto(227.10197491,33.468)
\curveto(227.10197491,34.752)(227.77397491,35.592)(228.80597491,35.592)
\curveto(229.59797491,35.592)(230.10197491,35.112)(230.10197491,34.356)
\lineto(230.10197491,30)
\lineto(231.09797491,30)
\lineto(231.09797491,34.752)
\curveto(231.09797491,35.796)(230.31797491,36.468)(229.10597491,36.468)
\curveto(228.16997491,36.468)(227.56997491,36.108)(227.01797491,35.232)
\lineto(227.01797491,36.288)
\closepath
}
}
{
\newrgbcolor{curcolor}{0 0 0}
\pscustom[linestyle=none,fillstyle=solid,fillcolor=curcolor]
{
\newpath
\moveto(236.08997052,36.288)
\lineto(236.08997052,30)
\lineto(237.09797052,30)
\lineto(237.09797052,33.264)
\curveto(237.10997052,34.776)(237.73397052,35.448)(239.11397052,35.412)
\lineto(239.11397052,36.432)
\curveto(238.94597052,36.456)(238.84997052,36.468)(238.72997052,36.468)
\curveto(238.08197052,36.468)(237.58997052,36.084)(237.01397052,35.148)
\lineto(237.01397052,36.288)
\closepath
}
}
{
\newrgbcolor{curcolor}{0 0 0}
\pscustom[linestyle=none,fillstyle=solid,fillcolor=curcolor]
{
\newpath
\moveto(245.29394708,32.808)
\curveto(245.29394708,33.768)(245.22194708,34.344)(245.04194708,34.812)
\curveto(244.63394708,35.844)(243.67394708,36.468)(242.49794708,36.468)
\curveto(240.74594708,36.468)(239.61794708,35.136)(239.61794708,33.06)
\curveto(239.61794708,30.984)(240.69794708,29.724)(242.47394708,29.724)
\curveto(243.91394708,29.724)(244.90994708,30.54)(245.16194708,31.908)
\lineto(244.15394708,31.908)
\curveto(243.87794708,31.08)(243.31394708,30.648)(242.50994708,30.648)
\curveto(241.87394708,30.648)(241.33394708,30.936)(240.99794708,31.464)
\curveto(240.75794708,31.824)(240.67394708,32.184)(240.66194708,32.808)
\closepath
\moveto(240.68594708,33.624)
\curveto(240.76994708,34.788)(241.47794708,35.544)(242.48594708,35.544)
\curveto(243.50594708,35.544)(244.21394708,34.752)(244.21394708,33.624)
\closepath
}
}
{
\newrgbcolor{curcolor}{0 0 0}
\pscustom[linestyle=none,fillstyle=solid,fillcolor=curcolor]
{
\newpath
\moveto(248.74993481,36.288)
\lineto(247.71793481,36.288)
\lineto(247.71793481,38.016)
\lineto(246.72193481,38.016)
\lineto(246.72193481,36.288)
\lineto(245.86993481,36.288)
\lineto(245.86993481,35.472)
\lineto(246.72193481,35.472)
\lineto(246.72193481,30.72)
\curveto(246.72193481,30.072)(247.15393481,29.724)(247.93393481,29.724)
\curveto(248.19793481,29.724)(248.41393481,29.748)(248.74993481,29.808)
\lineto(248.74993481,30.648)
\curveto(248.60593481,30.612)(248.47393481,30.6)(248.26993481,30.6)
\curveto(247.83793481,30.6)(247.71793481,30.72)(247.71793481,31.164)
\lineto(247.71793481,35.472)
\lineto(248.74993481,35.472)
\closepath
}
}
{
\newrgbcolor{curcolor}{0 0 0}
\pscustom[linestyle=none,fillstyle=solid,fillcolor=curcolor]
{
\newpath
\moveto(254.77391394,30)
\lineto(254.77391394,36.288)
\lineto(253.77791394,36.288)
\lineto(253.77791394,32.724)
\curveto(253.77791394,31.44)(253.10591394,30.6)(252.06191394,30.6)
\curveto(251.26991394,30.6)(250.76591394,31.08)(250.76591394,31.836)
\lineto(250.76591394,36.288)
\lineto(249.76991394,36.288)
\lineto(249.76991394,31.44)
\curveto(249.76991394,30.396)(250.54991394,29.724)(251.77391394,29.724)
\curveto(252.69791394,29.724)(253.28591394,30.048)(253.87391394,30.876)
\lineto(253.87391394,30)
\closepath
}
}
{
\newrgbcolor{curcolor}{0 0 0}
\pscustom[linestyle=none,fillstyle=solid,fillcolor=curcolor]
{
\newpath
\moveto(256.48991101,36.288)
\lineto(256.48991101,30)
\lineto(257.49791101,30)
\lineto(257.49791101,33.264)
\curveto(257.50991101,34.776)(258.13391101,35.448)(259.51391101,35.412)
\lineto(259.51391101,36.432)
\curveto(259.34591101,36.456)(259.24991101,36.468)(259.12991101,36.468)
\curveto(258.48191101,36.468)(257.98991101,36.084)(257.41391101,35.148)
\lineto(257.41391101,36.288)
\closepath
}
}
{
\newrgbcolor{curcolor}{0 0 0}
\pscustom[linestyle=none,fillstyle=solid,fillcolor=curcolor]
{
\newpath
\moveto(261.10987109,31.248)
\lineto(259.86187109,31.248)
\lineto(259.86187109,30)
\lineto(261.10987109,30)
\closepath
}
}
{
\newrgbcolor{curcolor}{0 0 0}
\pscustom[linestyle=none,fillstyle=solid,fillcolor=curcolor]
{
\newpath
\moveto(264.44586963,31.248)
\lineto(263.19786963,31.248)
\lineto(263.19786963,30)
\lineto(264.44586963,30)
\closepath
}
}
{
\newrgbcolor{curcolor}{0 0 0}
\pscustom[linestyle=none,fillstyle=solid,fillcolor=curcolor]
{
\newpath
\moveto(267.78186816,31.248)
\lineto(266.53386816,31.248)
\lineto(266.53386816,30)
\lineto(267.78186816,30)
\closepath
}
}
\end{pspicture}

    \import{figures/drawio}{Filtering_algo.drawio.svg.pdf_tex}
    \caption{Filtering Process}
    \label{fig:filtering-flowchart}
\end{figure}

Finally, before the signal-finding is applied, the Parrish method of refraction correction is applied to all subsurface photons.

\begin{figure}[htbp]
    \centering
    \includegraphics[width=\textwidth]{figures/methodology_refraction.jpg}
    \caption{The refraction correction applied to the remaining photons}
    \label{fig:refraction-photons}
\end{figure}

\subsection{Bathymetric Signal Extraction}\label{sec:kdesignalfinding}

The filtering steps reduce the dataset to just photon that are in the subsurface zone. To determine if there is bathymetric signal present, further processing is required. Some proposed methods for separating bathymetric signal photons from noise are explained in section \ref{subsec:denoising}. For this project\pdfcomment{reword?}, a new method is proposed based on a Gaussian Kernel Density Estimation (KDE) function. A function is created that returns the maximum kernel density, and the Z location at which it occurs. $$ f(\hat{z}_{window}) \rightarrow kde_{max},z_{kde_{max}} $$ Figure \ref{fig:kdefunc} shows the KDE function as applied to an example window, and the resulting kernel density plot. The KDE function is highly influenced by the \emph{Bandwidth} parameter. For this implementation, the Scott method \parencite{Scott2015} is used to estimate the required bandwidth based on the distribution of the data. 

This function is applied on a rolling basis to a window of 100 adjacent photons. This function returns a value for every single point along the transect, including in areas that do not have any noticeable signal. The kernel density value gives an indication of the strength of the peak. To reject the locations where the signal is weak, any points with a KDE value of less than the median value $$ kde_{50} $$ \pdfcomment{maybe just less than median? that decreases RMS error at the florida site} are assigned an NaN value and are dropped from the analysis.

\begin{figure}[htbp]
    \centering
    \includegraphics[width=\textwidth]{figures/2d_kde_plot.png}
    \caption{KDE function as applied to single window}
    \label{fig:kdefunc}
\end{figure}

The input parameters to the signal finding function are:

\begin{enumerate}
    \item The size of the window in \emph{number of points}
    \item the cutoff value for the Kernel Density required for point to be considered signal
\end{enumerate}

\subsection{Interpolation to a 2D grid}

After the bathymetric signal points are identified per the method in \ref{sec:kdesignalfinding}, the resulting bathymetry points are densely spaced along satellite tracklines, but are absent between them. A number of interpolation techniques have been applied to create bathymetry grids from point data. Commonly, inverse-distance weighting (IDW), tension splines, or loess interpolators have been used \parencite{gebcocookbook,Ferreira2017,}.

To convert these densely-spaced vector point locations to a raster of elevation data, and a raster of uncertainty data, geostatistical techniques are used.



\subsubsection{Subsampling of Bathymetric points using Poisson Disk Sampling} \label{subsec:poissonsubsampling}
The bathymetric points are extremely densely spaced, and kriging algorithms are computationally expensive. To reduce the number of points fed into the algorithm, a subsample of the points is taken using the poisson disk sampling technique. \pdfmargincomment{Need to expand on this quite a bit}
\pdfcomment{add figure showing points selected vs all points}

\subsubsection{Kriging interpolation}
Using the subsample of the points from the poisson disk sampling, they are converted to a bathymetry raster using Universal Kriging. This geostatistical technique results in both a raster of the estimated depth as well as the estimated uncertainty. 


\subsection{Bayesian Data Assimilation using Kalman Filtering}
The Kalman Filter is a mathematical technique to predict the state of systems based on uncertain measurements. It consist of a loop of two steps, an \emph{Time update} step which updates the position based on a measurement and a known measurement uncertainty, and a \emph{measurement update} step which predicts the state based on the dynamic equations of the system. The kalman filter equations, for a state $x_k$ and a vector of measurements of the state $z_k$:

Time Update:

\begin{equation}
    \hat{x}_{\bar{k}} = A\hat{x}_{k-1} + B\hat{u_{k-1}}
\end{equation}

\begin{equation}
    P_{\bar{k}} = A P_{k-1} A^T + Q
\end{equation}

Measurement Update:
\begin{equation}
    K = P_{\bar{k}} H^T(H P_{\bar{k}} H^T + R) ^{-1}
\end{equation}

\begin{equation}
    \hat{x}_k = \hat{x}_{\bar{k}} + K(\hat{z}_k - H \hat{x}_{\bar{k}})
\end{equation}

\begin{equation}
    P_k = (I - KH)P_{\bar{k}}
\end{equation}


The coastal zone is a highly dynamic system. However, for the purposes of this project is is assumed that the temporal variations over the time scale being studied are within the margin of error of the measurements, so the bathymetry of the nearshore zone is assumed to be a static system and the time update step can be ignored. It is also assumed all measurements are measurements of the same underlying physical depth, and that differences between measurements are due to normally distributed measurement error, with magnitude of the error varying depending on the method. To combine multiple measurements, the \emph{measurement update} step is applied recursively for each available measurement, producing a bayesian estimate of the bathymetry.  

\subsection{Error Evaluation}
\pdfcomment{add RMSE error derivation}
\subsection{Evaluating transect-level variables which predict bathymetry}
\pdfcomment{This could potentially be interesting and also help flesh out my answer to my first research question}
\subsection{Case Studies}
