\chapter{Introduction}
39\% of humans live within 100km of a coast, and many of the world's largest cities are located in coastal areas \parencite{Magdalena2021}. Further, 10\% of all human beings live in zones less than 10m above sea level, an area known as the Low Elevation Coastal Zone (LECZ) \parencite{Neumann2015,Lichter2011}. These economically and socially important areas are under significant stress due to anthropogenic climate change and the consequent sea level rise (SLR). Changing climate patterns will likely create more intense storms, which combined with a higher sea level boundary condition due to SLR, puts these populations and infrastructure investments at significant risk. Additionally, climate change can increase water stress and result in increases in groundwater abstraction, leading to subsidence of coastal areas. Climate change presents severe,even existential risks for coastal zones and the humans who live in them.

To mitigate these risks economically, substantial research into coastal protection strategies and climate adaptation measures is required. Scientific research will help fully quantify the risks, allow better prioritization of mitigation efforts, and help develop strategies for climate adaptation and coastal protection that work in harmony with nature. A critical variable for this research is nearshore bathymetric data \parencite{Holman2013}, which allows mapping of the spatial extents of flooding and better quantification of the wave transformation on the coasts. 

However, this critical data is often missing in most of the world, due to the complication and expense of surveying it. This project aims to capitalize on recent advances in remote sensing technology to explore ways of improving current bathymetry estimates without requiring any in-situ data. 

\section{Motivation and Relevance}

\subsection{Importance of Nearshore Bathymetric Data}

Bathymetric data is essential for many aspects of the blue economy, including aquaculture, marine energy, submarine cables, dredging operations, design of sea defenses, navigation, ecosystem preservation, and as a boundary condition for numerical studies of wave transformation. \parencite{Cesbron2021,Ashphaq2021}. However, the nearshore zone is notoriously difficult to survey. There is currently a global lack of data in the 5-10m zone \parencite{Albright2021} and up to 50\% of the world's shallow coastal zones remain unsurveyed \parencite{IHO/OHI2022}. Where surveys do exist, they can be decades out of date, especially in the 40\% of the world's coasts that are sandy and highly dynamic \parencite{Almar2021e}.

There have been attempts at global bathymetric datasets, most notably the General Bathymetric Chart of the Ocean (GEBCO), which is an annually-updated global bathymetric and topographic dataset. The bathymetry in GEBCO grids is derived by assimilation many different types of historic data, including  acoustic soundings provided by ships and gravimetric bathymetry measurements \parencite{Cesbron2021}. While it provides useful information about deep oceans, GEBCO accuracy in the nearshore zone is limited because sonar data is limited in many shallow nearshore zones. \parencite{Monteys2015}. Additionally, the horizontal and vertical resolution is not sufficient for accurate numerical studies of wave modeling.

\section{Identified Knowledge Gaps}




\section{Research Question}
The primary question that this project intends to answer is:

\emph{How can spaceborne remote sensing data be combined with existing global datasets to improve estimates of nearshore bathymetry?}

To answer this question, the following subquestions will be investigated:

\pdfcomment{make sure to link the subquestions to the methodology}

\begin{itemize}
      \item How can spaceborne lidar transects that contain bathymetry be identified algorithmically?
      \item Once transects with bathymetry are found, how can the seafloor elevation data be extracted?
      \item How can lidar photon return locations reflecting the seafloor be separated from background noise?
      \item How can spaceborne remote sensing sources be used to improve existing global bathymetry datasets?
      \item Under what conditions can remote sensing techniques provide useful improvement on bathymetric data estimates?
\end{itemize}
