\chapter{A Comparision of Geostatistic Interpolation methods}
% https://www.neonscience.org/resources/learning-hub/tutorials/spatial-interpolation-basics
It is a common problem in the geosciences to transform points observations into a continuous raster format, and operations called \emph{gridding}. To fill grid cells where no observations are present, interpolation techniques are used. These can be broadly divided into stochastic and deterministic methods \parencite{}, based on how they fill the empty grid cells, and how the uncertainty in the grid is considered 

% # TODO make some nice visuals
\section{Deterministic Interpolation}
Because the ICESat-2 data is very densely spaced along lines
\subsection{Inverse Distance Weighted}
\subsection{Spline}
\subsection{Nearest Neighbor}

\subsection{Natural Neighbor}
Standard Voroni diagrams

\section{Probabilistic Interpolation}
'random walk through known points' per wikipedia
\subsection{Ordinary Kriging}
\pdfcomment{replace esri site with an academic refernce material}
assumes the model $Z(s) = \mu + \epsilon(s)$ \parencite{esri site}, where the $\mu$ is an unknown constant mean

\subsection{Simple Kriging}
Like ordinary kriging, but $\mu$ is a known constant. Assumes the wide-sense stationarity, a zero mean, and a known covariance function (covariance in 2d space)
\subsection{Universal Kriging}
$\mu(s)$ is a deterministic function. Equivalent to polynomial regression. 
\subsection{Bayesian Kriging}