\chapter*{Abstract}
\noindent\emph{Nearshore bathymetry importance}

Knowledge of nearshore bathymetry is crucial for every aspect of the blue economy, especially for research on successful adaption to anthropogenic climate change. Currently, it is expensive to obtain nearshore bathymetry at regional scales because the shallow water area cannot be surveyed by boat. Therefore, nearshore bathymetry data is not available in many regions of the world, especially in the global south and big ocean states that are at the highest risk from climate change effects. Currently the only global bathymetric dataset, GEBCO, provides depth data at approximately a 500m resolution. This data is of limited accuracy in the nearshore zone, because the source data is especially sparse in nearshore areas. This thesis proposes a method of using GEBCO data as a starting point, and incorporating the spaceborne lidar data via a Kalman filter. This results in a product with an upscaled spatial resolution that provides bathymetry without requiring any in-situ data.

\noindent\emph{Lidar Satellite-derived bathymetry}

Recent developments in lidar remote sensing have shown that spaceborne lidar is capable of capturing nearshore bathymetry at depth of 0-20 meters if atmospheric conditions are good and the water is sufficiently clear. This provides a source of reliable, high-resolution bathymetric depth profiles that vary from 0-3 kilometers apart. This project proposes an automated method of extracting bathymetry points from the lidar data based on the density of the photon points in the underwater zone.

\noindent\emph{Data Assimilation via Kalman Updating}

To implement the Kalman upscaling, global data from GEBCO is clipped to the area of interest, and then resampled bilinearly to 50m resolution. Then, the ICESat-2 photon data for the area is processed generate point measurements of bathymetric depth. To fill in the gaps between these point measurements, the bathymetric points are subsampled and interpolated to the same resolution as the GEBCO data using a universal kriging interpolator. This interpolator results in a raster of the estimated depth, and a raster of the estimated uncertainty. To update the interpolated GEBCO grid, the Kalman gain is calculated for each raster cell in an elementwise manner, and using the Kalman state equation a new bathymetry grid is produced. If other data is available, the process can be applied recursively with other depth and uncertainty grids, allowing combination of any number of bathymetry datasets if the measurement uncertainty is known. 

\noindent\emph{Validation}

To validate the method, the RMSE error is calculated between the resulting bathymetry grid and previously validated, high-accuracy survey data. The validation will be applied at several global test sites to verify that the method generalizable to other regions.

\noindent\emph{Outlook} 

The results of this research could allow easier methods of characterizing nearshore bathymetry in remote areas, and could also be extended to include temporal variation – as more bathymetric data becomes available, it could be used measure the dynamic changes of coastal systems. Data with this temporal dimension would provide valuable validation data for coastal dynamics models. The method could also be feasibly be extended on a global basis, to produce a global dataset of bathymetry on coasts with clear water.
