\chapter{Results}

\section{Florida Keys test site}
\subsection{ICESat-2 Transects within AOI}
The AOI used the validate the method is shown in figure \ref{fig:keys_transects} below. 
\begin{figure}[h!]
    \centering
    \includegraphics[width=\textwidth]{figures/study_site_tracklines.jpg}
    % \input{figures/study_site_tracklines.pgf}
    \caption{Location of ICESat-2 Transects in Florida keys Study Area}
    \label{fig:keys_transects}
\end{figure}
\subsection{Comparison Data}

\begin{figure}[h!]
    \centering
    \includegraphics[width=\textwidth]{figures/florida_keys_ras.jpg}
    \caption{Ground Truth topobathymetric Survey taken after Hurricane Irma}
    \label{fig:truebathy}
\end{figure}
\subsection{Error in lidar photon data}

% \begin{table}
\caption{Atmospheric Profile vs error}
\begin{tabular}{lr}
 & 0 \\
atm_profile &  \\
profile_1 & 0.524872 \\
profile_2 & 0.666805 \\
profile_3 & 0.551757 \\
\end{tabular}
\end{table}


% \begin{table}
\caption{Beam Strength vs Error}
\begin{tabular}{lr}
 & RMS error \\
beamtype &  \\
strong & 11.206860 \\
weak & 9.363623 \\
\end{tabular}
\end{table}


\subsection{Bathymetric photons}
\ref{fig:bathyphotonmap} shows the location and estimated bathymetric depths of individual ICESat-2 photon returns.
\pdfcomment{change colorbars to match}
\begin{figure}[h!]
    \centering
    \includegraphics[width=\textwidth]{figures/florida_test_site_photon_map.jpg}
    \caption{Geographic distribution of Bathymetric photon data}
    \label{fig:bathyphotonmap}
\end{figure}

\subsection{Kalman Update with Lidar vs. simple bilinear}
\begin{table}[h!]
\caption{Comparison of the error metrics between the Kalman updating and a simple bilinear interpolaton of GEBCO data}
\label{raster_rmse_comparison}
\begin{tabular}{lrr}
\toprule
 & RMSE & MAE \\
\midrule
Naive Bilinear Interpolation & 2.086102 & 0.986900 \\
Kalman Updated Raster & 1.345899 & 0.836557 \\
\bottomrule
\end{tabular}
\end{table}

\section{Geographic Distribution of Results}
\section{Distribution of error at test sites}