\chapter{Results}

\section{Florida Keys Test Site}

\subsection{Site Conditions Summary}
\pdfcomment{Make this in pandas for consistency}


\begin{table}[htbp]
    \begin{minipage}{0.5\textwidth}
        \centering\begin{tabular}{r l }
            Parameter                                                      & \textbf{Value}                                 \\
            \hline
            Location                                                       & -81.04472,24.72868                             \\
            Tidal Range \footcite{tidal_data_reanalysis2022}               & \qty{0.51}{m}                                  \\
            Average Secchi Depth \footcite{ACRI-STGlobColourTeam2020}      & \qty{0.0}{m}                                   \\
            Validation Data vertical RMS error \footcite{Keys2019Lidar}    & \qty{0.056}{m}                                 \\
            Validation Data Horizontal Resolution \footcite{Keys2019Lidar} & \qty{9.4e-6}{ \degree} $\approx$ \qty{0.95}{m} \\
            Vertical Datum \footcite{Keys2019Lidar}                        & local MSL                                      \\
        \end{tabular}
    \end{minipage}
    \caption{Site conditions for the Florida Keys site}
    \label{table:floridasitestats}
\end{table}


\subsection{ICESat-2 Transects within AOI}
The AOI used the validate the method is shown in figure \ref{fig:keys_transects} below.
\begin{figure}[h!]
    \centering
    \includegraphics[width=0.5\textwidth]{figures/study_site_tracklines.jpg}
    % \input{figures/study_site_tracklines.pgf}
    \caption{Location of ICESat-2 Transects in Florida keys Study Area}
    \label{fig:keys_transects}
\end{figure}
\subsection{Validation Data}
The validation data used for this site is from a detailed lidar survey of the area performed in 2018 and 2019 by Quantum Spatial, Inc. \parencite{Keys2019Lidar}. The survey was performed using a Riegl VQ-880-G hydrographic airborne laser scanner. The instrument is designed for high bathymetric accuracy in shallow water.
\begin{figure}[h!]
    \centering
    \includegraphics[width=\textwidth]{figures/florida_keys_ras.jpg}
    \caption{Ground Truth topobathymetric Survey taken after Hurricane Irma}
    \label{fig:truebathy}
\end{figure}
\subsection{Error in lidar photon data}
After extracting the bathymetry from the ICESat-2 transects over the study site, the RMS error between the measured data from the USGS and the ICESat-2 data is
\begin{figure}[htbp]
    \centering
    \includegraphics[width=0.5\textwidth]{figures/florida_estimated_vs_truth.jpg}
    \caption{Comparison between depth calculated with ICESat-2 and true bathty}
    \label{fig:fl_truth_vs_measured_points}
\end{figure}
% \begin{table}
\caption{Atmospheric Profile vs error}
\begin{tabular}{lr}
 & 0 \\
atm_profile &  \\
profile_1 & 0.524872 \\
profile_2 & 0.666805 \\
profile_3 & 0.551757 \\
\end{tabular}
\end{table}


% \begin{table}
\caption{Beam Strength vs Error}
\begin{tabular}{lr}
beamtype & RMS error  \\
strong & 11.206860 \\
weak & 9.363623 \\
\end{tabular}
\end{table}


\subsection{Bathymetric photons}
Figure \ref{fig:bathyphotonmap} shows the location and estimated bathymetric depths of individual ICESat-2 photon returns.
\pdfcomment{change colorbars to match}
\begin{figure}[h!]
    \centering
    \includegraphics[width=\textwidth]{figures/florida_test_site_photon_map.jpg}
    \caption{Geographic distribution of Bathymetric photon data}
    \label{fig:bathyphotonmap}
\end{figure}

\subsection{Lidar updated GEBCO vs. simple bilinear Interpolation}
The results when comparing a raw interpolation of GEBCO to 
\begin{table}[h!]
\caption{Comparison of the error metrics between the Kalman updating and a simple bilinear interpolaton of GEBCO data}
\label{raster_rmse_comparison}
\begin{tabular}{lrr}
\toprule
 & RMSE & MAE \\
\midrule
Naive Bilinear Interpolation & 2.086102 & 0.986900 \\
Kalman Updated Raster & 1.345899 & 0.836557 \\
\bottomrule
\end{tabular}
\end{table}

\section{Geographic Distribution of Results}
\section{Distribution of error at test sites}