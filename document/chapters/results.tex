\chapter{Results}


\section{results summary}
\pdfcomment{Compare the effect of other physical or atmospheric variables on the error, like include an entire section}
\pdfcomment{Add a point density per sqaure meter for each site}
\pdfcomment{average depth for each site?}


\section{Florida Keys Test Site}
\subsection{Site Conditions Summary}
\pdfcomment{Make this in pandas for consistency}


\begin{table}[htbp]
    \begin{minipage}{0.5\textwidth}
        \centering\begin{tabular}{r l }
            Parameter                                                      & \textbf{Value}                                 \\
            \hline
            Location                                                       & -81.04472,24.72868                             \\
            Tidal Range \footcite{tidal_data_reanalysis2022}               & \qty{0.51}{m}                                  \\
            Average Secchi Depth \footcite{ACRI-STGlobColourTeam2020}      & \qty{0.0}{m}                                   \\
            Validation Data Vertical RMS error \footcite{Keys2019Lidar}    & \qty{0.056}{m}                                 \\
            Validation Data Horizontal Resolution \footcite{Keys2019Lidar} & \qty{9.4e-6}{ \degree} $\approx$ \qty{0.95}{m} \\
            Vertical Datum \footcite{Keys2019Lidar}                        & local MSL                                      \\
        \end{tabular}
    \end{minipage}
    \caption{Site conditions for the Florida Keys site}
    \label{table:floridasitestats}
\end{table}


\subsection{ICESat-2 Transects within AOI}
The AOI used the validate the method is shown in figure \ref{fig:keys_transects} below.
\begin{figure}[h!]
    \centering
    \includegraphics[width=0.5\textwidth]{figures/study_site_tracklines.jpg}
    % \input{figures/study_site_tracklines.pgf}
    \caption{Location of ICESat-2 Transects in Florida keys Study Area}
    \label{fig:keys_transects}
\end{figure}
\subsection{Validation Data}
The validation data used for this site is from a detailed lidar survey of the area performed in 2018 and 2019 by Quantum Spatial, Inc. \parencite{Keys2019Lidar}. The survey was performed using a Riegl VQ-880-G hydrographic airborne laser scanner. The instrument is designed for high bathymetric accuracy in shallow water.
\begin{figure}[h!]
    \centering
    \includegraphics[width=\textwidth]{figures/florida_keys_ras.jpg}
    \caption{Ground Truth topobathymetric Survey taken after Hurricane Irma}
    \label{fig:truebathy}
\end{figure}
\subsection{Error in lidar photon data}
After extracting the bathymetry from the ICESat-2 transects over the study site, the RMS error between the measured data from the USGS and the ICESat-2 data is
\begin{figure}[htbp]
    \centering
    \includegraphics[width=0.7\textwidth]{figures/Florida_keys_lidar_estimated_vs_truth.jpg}
    \caption{Comparison between depth calculated with ICESat-2 and true bathy}
    \label{fig:fl_truth_vs_measured_points}
\end{figure}
% \begin{table}
\caption{Atmospheric Profile vs error}
\begin{tabular}{lr}
 & 0 \\
atm_profile &  \\
profile_1 & 0.524872 \\
profile_2 & 0.666805 \\
profile_3 & 0.551757 \\
\end{tabular}
\end{table}


% \begin{table}
\caption{Beam Strength vs Error}
\begin{tabular}{lr}
 & RMS error \\
beamtype &  \\
strong & 11.206860 \\
weak & 9.363623 \\
\end{tabular}
\end{table}


\subsection{Bathymetric photons}
Figure \ref{fig:bathyphotonmap} shows the location and estimated bathymetric depths of individual ICESat-2 photon returns.
\pdfcomment{change colorbars to match}
\begin{figure}[h!]
    \centering
    \includegraphics[width=\textwidth]{figures/Florida_keys_photon_map.jpg}
    \caption{Geographic distribution of Bathymetric photon data}
    \label{fig:bathyphotonmap}
\end{figure}

\subsection{Lidar updated GEBCO vs. Simple bilinear Interpolation}
The results when comparing a raw interpolation of GEBCO to
\begin{table}[h!]
\caption{Comparison of the error metrics between the Kalman updating and a simple bilinear interpolaton of GEBCO data}
\label{raster_rmse_comparison}
\begin{tabular}{lrr}
\toprule
 & RMSE & MAE \\
\midrule
Naive Bilinear Interpolation & 2.086102 & 0.986900 \\
Kalman Updated Raster & 1.345899 & 0.836557 \\
\bottomrule
\end{tabular}
\end{table}

\
\section{Petten Test Site}
Most research results show that spaceborne lidar is not effective at retrieving bathymetry from very turbid waters. To test the upper limits of this, ICESat-2 data for the coast of Petten, NL was downloaded. As expected, there was no bathymetric signal found, likely due to the inability of the laser to penetrate the turbid water.

However, The Dutch government provides a dataset of point surveys of the Dutch coast. This data was used as input to the kriging and Kalman updating code.

\subsection{Site Conditions Summary}
\begin{table}[htbp]
    \begin{minipage}{0.5\textwidth}
        \centering\begin{tabular}{r l }
            Parameter                                                 & \textbf{Value}                  \\
            \hline
            Location                                                  &                                 \\
            Tidal Range \footcite{tidal_data_reanalysis2022}          & \qty{2.24}{m}                   \\
            Average Secchi Depth \footcite{ACRI-STGlobColourTeam2020} & \qty{3.86}{m}                   \\
            Validation Data vertical RMS error                        & \qty{}{m} \pdfcomment{look up}  \\
            Validation Data Horizontal Resolution                     & \qty{}{m} \pdfcomment{look up?} \\
            Vertical Datum                                            & Normaal Amsterdams Peil (NAP)   \\
        \end{tabular}
    \end{minipage}
    \caption{Site conditions for the Florida Keys site}
    \label{table:Pettensitestats}
\end{table}

\subsection{Validation data}
The validation data for this site was from a 2021 survey performed by Van Oord.
\subsection{Error between Jarkus Vs. Survey}
There was some error between the Jarkus survey points and the Van Oord 1m surveyed grid. The error metrics are shown in table \ref{tab:jarkus_vs_survey_error}, and the distribution of the error is shown in figure
\begin{table}[h!]
\caption{Error between Surveyed Bathymetry grid and Jarkus}
\label{tab:jarkus_vs_survey_error}
\begin{tabular}{lrr}
\toprule
 & MAE & RMSE \\
\midrule
Petten & 0.267152 & 0.439663 \\
\bottomrule
\end{tabular}
\end{table}


\begin{figure}[h!]
    \centering
    \includegraphics[width=0.5\textwidth]{figures/Petten_lidar_estimated_vs_truth.jpg}
    \caption{Jarkus Survey points vs same location on Van Oord Survey data}
    \label{fig:jarkus_vs_survey}
\end{figure}

\subsection{Jarkus Data kriging and kalman Update}
\begin{table}[h!]
\caption{Improvement in Error when incorporating Kriged Jarkus data}
\label{tab:jarkus_kriging_kalman_error}
\begin{tabular}{lrr}
\toprule
 & RMSE & MAE \\
\midrule
Naive Bilinear Interpolation & 1.473726 & 1.311282 \\
Kalman Updated Raster & 0.722669 & 0.483035 \\
Kriged Raster & 0.743263 & 0.496904 \\
\bottomrule
\end{tabular}
\end{table}


\section{St. Croix}
\section{Charlotte Amalie}
Another test site is the island of Charlotte Amalie in the US Virgin Islands. Basic details about the test site are in \ref{table:charlotteamalie_datatable}
\begin{table}[htbp]
    \begin{minipage}{0.5\textwidth}\pdfcomment{fill in table}
        \centering\begin{tabular}{r l }
            Parameter                                                 & \textbf{Value} \\
            \hline
            Location                                                  &                \\
            Tidal Range \footcite{tidal_data_reanalysis2022}          & \qty{}{m}      \\
            Average Secchi Depth \footcite{ACRI-STGlobColourTeam2020} & \qty{}{m}      \\
            Validation Data vertical 95\% confidence                  & $x$ m          \\
            Validation Data Horizontal Resolution                     & \qty{}{m}      \\
            Vertical Datum                                            & lookup         \\
        \end{tabular}
    \end{minipage}
    \caption{Site conditions for the Charlotte Amalie}
    \label{table:charlotteamalie_datatable}
\end{table}
Overall there where X \pdfcomment{lookup number of transects} transects of ICESat-2 data available for the site, and their distribution is shown in figure \ref{}.

\begin{figure}[htbp]
    \centering
    \includegraphics[width=\textwidth]{figures/Charlotteamalie_tracklines.jpg}
    \caption{ICESat-2 tracks in the Charlotte Amalie Test site}
    \label{fig:charlotteamalie-tracklines}
\end{figure}

\subsection{Lidar Points found for site}
Figure \ref{fig:point-map-charlotteamalie} shows the geographic distribution of the points

\begin{figure}[htbp]
    \centering
    \includegraphics[width=\textwidth]{figures/Charlotteamalie_photon_map.jpg}
    \caption{Charlotte Amalie points}
    \label{fig:point-map-charlotteamalie}
\end{figure}

The error for the site is shown in \ref{fig:charlotteamalie-lidar-bias}

\begin{figure}[htbp]
    \centering
    \includegraphics[width=\textwidth]{figures/Charlotteamalie_lidar_estimated_vs_truth.jpg}
    \caption{Bias plot showing the agreement between the validation data and the lidar points for Charlotte Amalie}
    \label{fig:charlotteamalie-lidar-bias}
\end{figure}


\section{Oahu}

\subsection{Site Conditions Summary}
\begin{table}[htbp]
    \begin{minipage}{0.5\textwidth}
        \centering\begin{tabular}{r l }
            Parameter                                                 & \textbf{Value}                      \\
            \hline
            Location                                                  &                                     \\
            Tidal Range \footcite{tidal_data_reanalysis2022}          & \qty{0.59}{m}                       \\
            Average Secchi Depth \footcite{ACRI-STGlobColourTeam2020} & \qty{}{m}                           \\
            Validation Data vertical 95\% confidence                  & $\sqrt[]{0.20^2 + 0.013 * depth}$ m \\
            Validation Data Horizontal Resolution                     & \qty{1}{m}                          \\
            Vertical Datum                                            & Normaal Amsterdams Peil (NAP)       \\
        \end{tabular}
    \end{minipage}
    \caption{Site conditions for the Petten test site}
    \label{table:Oahusitestats}
\end{table}

\section{Random Sampling vs linear sampling along lines}