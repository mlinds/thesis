\chapter*{Summary}

Bathymetric data is valuable because it is essential for nearly every aspect of coastal management and monitoring but is difficult to obtain, especially in the nearshore zone. Traditional techniques like acoustic sounding provide reasonably high-accuracy data, but survey campaigns are expensive and the survey vessels cannot operate in the shallowest areas of the nearshore zone. Airborne lidar surveying provides high resolution, high accuracy data in the shallowest waters, but the associated survey campaigns are extremely expensive and are very sensitive to environmental conditions. Because of the limitations of conventional survey techniques in the nearshore zone, there is much interest in ways of estimating bathymetry using only spaceborne remote sensing data. Spaceborne remote sensing data is advantageous because of wide global coverage and because the data is often publicly available.

Traditionally, there are two main techniques for extracting bathymetric estimates from satellite data. One approach, called optical satellite derived bathymetry, estimates the bathymetry based on the relationship between the attenuation of different parts of the visible light spectrum in the water column. These techniques require in-situ data to calibrate, and require the site to have water that is clear down to the seabed. The second approach is to use satellite imagery to estimate the wave field in a region, and then back-calculate the bathymetry based on the evolution of the wave field in space. This is called the wave-kinematic approach. This approach provides less accurate and lower resolution bathymetry estimates than optical SDB, and the deepest depth that can be estimated is limited by the maximum wavelength at the time of the satellite data acquisition. The advantage of wave-kinematic approaches is that the deepest estimation depth is not limited by water clarity.

NASA's ICESat-2 satellite is a lidar satellite that provides high-accuracy point elevations along lines all over the world. It was originally intended to study the elevation of ice in the polar regions, but researchers have discovered that it can also accurately capture the elevation of the seabed in coastal areas with very clear water. This has led to a number of papers that extract bathymetric point data from ICESat-2 and use it in combination with optical SDB methods to produce a bathymetric estimate without requiring any in-situ data.  

Most of the studies that try to extract point data from ICESat-2 look at a few (3-10) satellite passes, and either use manual extraction of the seabed signal based on the researchers judgement, or use other semi-automated techniques that estimate the seabed based on the density of photons in the vertical direction, then are manually checked before use. The goal of this project is to evaluate the potential to automate the signal finding approach to extract bathymetric point measurements at a larger scale (up hundreds of transects), to evaluate how these sparse point measurements might be interpolated into a continuous gridded product, and to evaluate if this interpolated data can be combined with existing global datasets to add additional value by increasing the resolution and decreasing the error. 

The methodology proposed is split into 3 major parts. First, the data processing and signal extraction from the satellite data: For a given area of interest, the ICESat-2 data is downloaded and processed to extract only photons that are located in the nearshore zone. It is then corrected for the effect of light refraction in the water column. Then, points that are likely to be sea surface signal are extracted based on the density of photon returns in the vertical direction along a moving window of adjacent photons. The second part of the method is to interpolate these points into a continuous bathymetry surface universal kriging, a geostatistical approach that takes into account the elevation at each point and the distance between them to produce a gridded output of the estimated interpolated surface and the the confidence in the interpolation at every point. The third step is to combine the interpolated surface with a prior estimate, in this case the global bathymetry dataset GEBCO via a Kalman filter. The Kalman filter is a bayesian approach that takes into account the new estimate and the uncertainty. Areas of the interpolated data with high certainty (i.e., nearer to ICESat-2 point measurements) will have a larger impact on the prior estimate, whereas areas with a lower certainty will have a correspondingly lower magnitude of impact on the prior depth estimate.

To test the feasibility of the kriging and Kalman updating approach, a synthetic experiment is created using a one of validation data sets. Random point samples from the validation data are taken, and these point samples are used as input to the kriging and kalman updating process. For the study site evaluated here, it is found that by kriging these input points and then combining them with GEBCO data, the error of the combined product is lower than either the error of GEBCO or the kriging surface. 

The technique is also validated 