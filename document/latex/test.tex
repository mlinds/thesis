\hypertarget{test-document}{%
\section{Test document}\label{test-document}}

testing the capabilities of my current planned workflow to write in
markdown, convert that to latex using pandoc, and compile the latex
against texlive.

\hypertarget{why-so-complicated}{%
\subsection{why so complicated?}\label{why-so-complicated}}

my hope is that by investing all this time setting up the workflow i can
stay organized throughout the project. By working in markdown for the
main writing I take advantage of the simple but powerful capabilities of
markdown combined with \href{https://pandoc.org/}{pandoc}, while also
easily compiling a nice looking latex document against the texlive
docker image

\hypertarget{method}{%
\subsection{method}\label{method}}

\begin{itemize}
\tightlist
\item
  all documents added to zotero will automatically be added to one large
  bibtex file which will be exported to my document directory
\item
  writing will be down in markdown with citations as needed
\item
  when compiling, the markdown files will be converted to .tex against a
  pandoc docker image
\item
  these .tex files will be compiled to pdf against the texlive docker
  image \autocite{PandocPandoc}
\end{itemize}

\(\text{the answer to life, the universe, and everything} = 42\)

\(\sum{69}\)

\hypertarget{table-testing}{%
\subsubsection{table testing}\label{table-testing}}

\input{../tables/table_test.tex}

hopefully this works

\hypertarget{image-testing}{%
\subsubsection{image testing}\label{image-testing}}

\begin{figure}
\centering
\includegraphics{../figures/testimage.png}
\caption{test image caption?}
\end{figure}

\hypertarget{another-image}{%
\subsubsection{another image}\label{another-image}}

\begin{figure}
\centering
\includegraphics{../figures/testimage2.png}
\caption{test image}
\end{figure}
